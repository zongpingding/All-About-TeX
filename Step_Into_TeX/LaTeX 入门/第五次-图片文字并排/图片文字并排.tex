\documentclass[fontset=windows]{article}
\usepackage{ctex}
\usepackage{wrapfig}
\usepackage{graphicx}
\usepackage{amsmath}
\usepackage{mathtools}


\begin{document}
%	『h』当前位置。将图形放置在正文文本中给出该图形环境的地方。如果本页所剩的页面不够,这一参数将不起作用。
%	『t』顶部。将图形放置在页面的顶部。
%	『b』底部。将图形放置在页面的底部。
%	『p』浮动页。将图形放置在一只允许有浮动对象的页面上。
%	 第一种方式	
	\begin{wrapfigure}{l}{0pt}
		\centering
		\includegraphics[scale=0.2]{C:/数学建模/微分方程建模/源码【微分方程建模】/Logistic模型.pdf}
		\caption{左排的图片}
	\end{wrapfigure}
	Wikipedia began as a related project for Nupedia.Nupedia was a free English-language online encyclopedia project.
	Nupedia's articles were written and owned by Bomis, Inc which was a web portal company. 
	The main people of the company were Jimmy Wales, the guy 
	% 第二种方式
	
	\vspace{8em}
	\begin{minipage}{0.5\linewidth}
		Wikipedia began as a related project for Nupedia. Nupedia was a free English-language online encyclopedia project.
		Nupedia's articles were written and owned by Bomis, Inc which was a web portal company. 
		The main people of the company were Jimmy Wales, the guy 
	\end{minipage}
	\hfill % \hfill 可以自动填充一定长度的空白	% 两个minipage之间不能空行
	\begin{minipage}{0.5\linewidth}
		\includegraphics[scale=0.2]{C:/数学建模/微分方程建模/源码【微分方程建模】/Logistic模型.pdf}
		
		右排的图片
	\end{minipage}

	$$\frac{x^2}{\lim_{x=1}^{\sum_{i=1}^{n}{f(\xi_i)}}}$$



	
\end{document}


