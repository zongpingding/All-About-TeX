\documentclass[UTF8]{ctexbook}
\usepackage{xcolor}
\usepackage{tikz}
\newcommand{\piyu}[4]{%
\raisebox{-0.3em}{\begin{tikzpicture}
    \draw[lightgray] (0,0pt) rectangle (20pt, 10pt);%矩形
    \fill[fill = lightgray] (0,0pt) rectangle (10pt, 10pt);%填色,还可以设置线宽
    \node [below,rotate= 0] at (5pt, 6pt) {\smash{\hbox{\color{white}\fangsong\fontsize{8}{10}\selectfont{#1}}\hss}};
    \node [below,rotate= 0] at (15pt, 6pt) {\smash{\hbox{\color{black}\fangsong\fontsize{8}{10}\selectfont{#2}}\hss}};
\end{tikzpicture}%
}% end of raisebox
{\textcolor{#3}{{\fangsong{\fontsize{8}{10}\selectfont#4}}}}}%


\begin{document}
虽是如此说,只这酒色财气四件中,惟有“财色”二者更为利害。怎见得他的利害?假如一个人到了那穷苦的田地,
受尽无限凄凉,耐尽无端懊恼,晚来摸一摸米瓮,苦无隔宿之炊,早起看一看厨前,愧没半星烟火,
\piyu{绣}{眉}{green}{情景逼真。酸俫读此,能不雪涕!}妻子饥寒,一身冻馁,就是那粥饭尚且艰难,
那讨馀钱沽酒?\piyu{绣}{旁}{green}{酒因财缺。}更有一种可恨处,亲朋白眼,面目寒酸,便是凌云志气,
分外消磨,怎能勾与人争气!\piyu{绣}{旁}{green}{气以财弱。}\piyu{张}{夹}{red}{以上反起财。}
正是:\piyu{张}{夹}{red}{这一个“正是”是冷。}

\end{document}