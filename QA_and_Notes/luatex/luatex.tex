\documentclass{article}
\usepackage{luacode}
\usepackage[margin=2.5cm]{geometry}
\usepackage{comment}


% plot function
\usepackage{tikz}
\usepackage{pgfplots}
\pgfplotsset{compat=1.18}

% ==> way 1:tcolorbox
\usepackage{xparse}
\usepackage{minted}
\usepackage{tcolorbox}
\tcbuselibrary{listings, minted, breakable, skins}
\tcbset{listing engine=minted}
\definecolor{bg}{RGB}{242, 242, 242}
\DeclareTCBListing{code}{!O{tex}}{
  % ==> basic settings
  enhanced,
  breakable,
  colback=bg, 
  frame hidden, 
  left=1mm, right=1mm,
  top=0mm, bottom=0mm, 
  %
  % 
  % ==> source and ouput
  sidebyside, 
  listing and text,
  % listing only, 
  % text and listing,
  % listing outside text,
  %
  % 
  % ==> minted options
  minted language=#1, 
  minted style=manni,
  minted options = {  
    autogobble,
    tabsize=2,
    breaklines=true, 
    breakanywhere=true,
    breakanywheresymbolpre=,
    breaksymbolleft=,
    fontsize=\footnotesize
  }
}


% ==> way 2: 'tkzexample' package
% \usepackage{tkzexample} 
% \usepackage[protrusion = true,
%             expansion, final,
%             verbose = false, babel   = true]{microtype}
% \DisableLigatures{encoding = T1, family = tt*}  
% \colorlet{graphicbackground}{blue!10!white}
% \colorlet{codebackground}{red!10}




\title{Lua\TeX{}}
\author{Eureka}
\date{\today} 
\begin{document}
\maketitle


% \begin{comment}
\section{Lua\TeX{} Primitive}
\begin{code}
% run TeX command in Lua
\newcommand{\texcmd}{some text}
\directlua{
  local var = "\texcmd"
  tex.print(var)
}
\end{code}



\section{Luacode package}
\subsection{baisc uasge}

\begin{code}
\begin{luacode}
  tex.sprint(math.random())
\end{luacode}
\end{code}


% this code will work
\begin{code}
\begin{luacode*}
  tex.print("\\begin{tabular}{|l|l|l|}")
  tex.print("\\hline ")
  tex.print("1 & a & Test A \\\\")
  tex.print("2 & b & Test B \\\\")
  tex.print("\\hline ")
  tex.print("\\end{tabular}")
\end{luacode*}
\end{code}


\subsection{define function}
using luacode* because there are percentage signs
\begin{code}
% declare the sqrt function in Lua
\begin{luacode*}
  function compute_sqrt(v)
    local s = math.sqrt(tonumber(v))
    local s_f = math.floor(s)
    local o
    if (math.abs(s_f * s_f - v) < 1.0e-5) then
        o = string.format("%.1f", s)
    else
        o = string.format("%.6f", s)
    end
    tex.print(o)
  end
\end{luacode*}

% declare a wrapper in TeX
\newcommand{\luasqrt}[1]{\directlua{compute_sqrt(#1)}}
\luasqrt{2}
\end{code}

\section{LuaTEX for Plot}
Consider function $f(x) = (\pi - x)/2$ in the interval $(0, 2\pi)$.

\begin{code}
\begin{luacode*}
-- Fourier series
function partial_sum(n,x)
  partial = 0;
  for k = 1, n, 1 do
    partial = partial + math.sin(k*x)/k
  end;
  return partial
end

-- Code to write PGFplots data as coordinates
function print_partial_sum(n,xMin,xMax,npoints,option)
  local delta = (xMax-xMin)/(npoints-1)
  local x = xMin
  if option~=[[]] then
    tex.sprint("\\addplot["..option.."] coordinates{")
  else
    tex.sprint("\\addplot coordinates{")
  end
  for i=1, npoints do
    y = partial_sum(n,x)
    tex.sprint("("..x..","..y..")")
    x = x+delta
  end
  tex.sprint("}")
end
\end{luacode*}

\newcommand\addLUADEDplot[5][]{%
  \directlua{print_partial_sum(#2,#3,#4,#5,[[#1]])}%
}


\pgfplotsset{width=15cm, height=7cm}
\begin{tikzpicture}[scale=.5]
\begin{axis}[xmin=-0.2, xmax=31.6, ymin=-1.85, ymax=1.85,
    xtick={0,5,10,15,20,25,30},
    ytick={-1.5,-1.0,-0.5,0.5,1.0,1.5},
    minor x tick num=4,
    minor y tick num=4,
    axis lines=middle,
    axis line style={-}
  ]
  % SYNTAX: Partial sum 30, from x = 0 to 10*pi, sampled in 1000 points
  \addLUADEDplot[color=blue,smooth]{30}{0}{10*math.pi}{1000};
\end{axis}
\end{tikzpicture}
\end{code}
% \end{comment}

\end{document}