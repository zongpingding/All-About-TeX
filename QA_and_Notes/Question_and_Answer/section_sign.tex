\documentclass[12pt]{ctexart}
\usepackage[margin=2cm]{geometry}
\usepackage{amsmath}
\usepackage{lipsum}
\usepackage{hyperref}


\renewcommand{\thesection}{\S{}\arabic{section}}
\makeatletter
\@addtoreset{equation}{section}
\renewcommand{\theequation}{\arabic{section}.\arabic{equation}}
\makeatother


\begin{document}
    \section{The First Section}
    这是一段乱糟糟的话\ref{eq:abc}.其中引用了一个公式.根据\eqref{eq:abc},我们可以知道这个世界的秘密是这样的:\lipsum[8]
    \subsection{A subsection}
        \begin{equation}
            a^2+b^2=c^2
        \end{equation}
        \lipsum[1]
        \begin{equation}
            3^2+4^2=5^2
        \end{equation}
        \subsection{A subsection}
        \begin{equation}
            a^2+b^2=c^2
        \end{equation}
        \begin{equation}
            3^2+4^2=5^2
        \end{equation}
        \subsection{A subsection}
        \begin{equation}
            a^2+b^2=c^2
        \end{equation}
    \section{The Second Section}
    \begin{equation}
        a^2+b^2=c^2
    \end{equation}
    \begin{equation}
        3^2+4^2=5^2
    \end{equation}
    \lipsum[2]
    \subsection{A subsection}
    \begin{subequations}
          \begin{equation}
              a^2+b^2=c^2
          \end{equation}
          \begin{equation}
              3^2+4^2=5^2
          \end{equation}
    \end{subequations}
    \subsection{A subsection}
    \begin{subequations}
          \begin{equation}
              a^2+b^2=c^2
              \label{eq:abc}
          \end{equation}
    \end{subequations}
    \lipsum[3]
        \begin{equation}
            \sum_{n=1}^{\infty} \frac{1}{n^2} = \frac{\pi^2}{6}
        \end{equation}
    \section{The Third Section}
    \begin{equation}
        \sum_{n=1}^{\infty} \frac{1}{n^2} = \frac{\pi^2}{6}
    \end{equation}
    下面尝试使用\verb|amsmath|宏包提供的包装公式环境\verb|subequations|实现.
    \begin{subequations}
        \begin{align}
        \sum_{n=1}^{\infty} \frac{1}{n^2} &= \frac{\pi^1}{1111} \\
        \sum_{n=1}^{\infty} \frac{1}{n^2} &= \frac{\pi^2}{2222} \\
        \sum_{n=1}^{\infty} \frac{1}{n^2} &= \frac{\pi^3}{3333} \\
        \sum_{n=1}^{\infty} \frac{1}{n^2} &= \frac{\pi^4}{4444}
        \end{align}
        \label{eq:mult-lines}
    \end{subequations}

    \begin{subequations}
      \begin{align}
        1 + 1 = 3 \\
        2 + 2 = 5
      \end{align}
    \end{subequations}

    \subsection{A subsection}
    \begin{subequations}
      \begin{align}
        1 + 1 = 3 \\
        2 + 2 = 5
      \end{align}
    \end{subequations}

    本句交叉引用整个公式.生活就像海洋,只有意志坚强的人才能到达彼岸\eqref{eq:mult-lines}.

    本句交叉引用子公式(what I mean is just like 3.2c).生活就像海洋,只有意志坚强的人才能到达彼岸.

    \lipsum[4]

    \begin{subequations}
    \renewcommand{\theequation}{\theparentequation-\roman{equation}-\alph{equation}-以及公式\textbf{编号}\chinese{equation}}
        \begin{align}
        \sum_{n=1}^{\infty} \frac{1}{n^2} &= \frac{\pi^1}{1111} \\
        \sum_{n=1}^{\infty} \frac{1}{n^2} &= \frac{\pi^2}{2222} \\
        \sum_{n=1}^{\infty} \frac{1}{n^2} &= \frac{\pi^3}{3333} \\
        \sum_{n=1}^{\infty} \frac{1}{n^2} &= \frac{\pi^4}{4444}
        \end{align}
        \label{eq:mult-lines-pro}
    \end{subequations}

    \lipsum[9]

    \begin{equation}
        \sum_{n=1}^{\infty} \frac{1}{n^2} = \frac{\pi^2}{6}
    \end{equation}

\end{document}