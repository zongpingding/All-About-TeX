\documentclass{article}


\begin{document}
% 1.声明盒子
% 盒子只能赋值为一个盒子hbox, vbox,不能写为\setbox\boxa={boxa}这种
% way 1.1
\newbox\boxa
\newbox\boxb
\setbox\boxa=\hbox{boxa}
\setbox\boxb=\hbox{boxb}
\the\boxa \copy\boxa

\the\boxb \box\boxb

% 为节省内存,盒子寄存器在使用后会被清空所以下面的boxb中就没有东西了
% 用copy可以重复使用盒子寄存器的内容,不清空
\box\boxa 
\box\boxb 

% way 1.2
\setbox0=\hbox{hbox}
\setbox1=\vbox{vbox}
\box0
\copy1
% \box1
% 盒子寄存器的编号由 \newbox 命令分配,所以下面的\boxa,\boxb其实可以看做整数
% 所以使用\box\boxa <=> \box52


% 3.盒子解包
% \unhbox , \unvbox, \unhcopy, \unvcopy 
% 解包盒子的操作只能用在恰当的模式下 
% \ifvoid 判断盒子寄存器中是否有东西
% \ifhbox,\ifvbox判断盒子寄存器中盒子的类型 
\ifvoid0 box-0 is empty\fi

\ifvbox1 box-1 has a vbox\fi

% 4.last box
% 5. domension
a\hbox{\kern-1em b}--


% 6. a macro
% \def\smash#1{{\setbox0=\hbox{#1}\dp0=0pt \ht0=0pt \box0\relax}}
% \smash{hello} world

% 7. dimen
\par
% \setbox8=\hbox{world}
% hello {\dimen0=\wd8 \box8 \kern-\dimen0} haha

% 8. underfull and overfull
\hbox to 10pt{hello}
\hbox to 20.555pt{hello}
\hbox to 30pt{hello}

% 9. \hss command: \hskip 0pt plus 1fil minus 1fil(盒子没被填满就撑满,盒子尺寸不够就吸收)
\hbox to 20pt{\hss hello}
\hbox to 30pt{\hss hello}
\setbox0=\hbox{hello}
% \showbox0
\end{document}
