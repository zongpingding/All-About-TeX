\documentclass{article}
\usepackage[margin=3.5cm]{geometry}
\usepackage[T1]{fontenc}
\usepackage{xcolor}


\title{\LaTeX{} 3 basic knowledge}
\author{Eureka}
\date{\today}
\begin{document}
\maketitle
\ExplSyntaxOn
\newcommand{\cmd}[1]{
  \texttt{\tl_to_str:n {#1}}
}
\section{cs_new:Npn~ VS~ cs_new:Nn}
% 1.Npn <=> Nn
% 也就是新定义的命令有几个 'n' 那么你就需要几个参数,
% 不然就会把你的其他东西给吃进去
\cs_new:Nn \mytest_foo:nn
  {Hello~ \textbf{#1}, #2~ world}
\mytest_foo:nn {\TeX{}}{\par}\par

\section{_nopar~ command}
% 2..._nopar: 意味着你不能在这个命令的传输中传入\par 
\cs_new_nopar:Nn \mytest_bar:nn
  {Hello~ \textbf{#1, #2}~ world\par}
\mytest_foo:nn {\TeX{}}{}\par

\section{expand~ type}
% 3. argument specifier: M, n, e, f, x, etc. 
% see more at: "Interface3.pdf" Chapter 1-1.1 'Naming functions and variables'
\subsection{e/x/o-type}
\cs_set:Nn \test:n {
  \tl_set:Nn \l_tmpb_tl {#1}
  \cs_meaning:N \l_tmpb_tl\par
}
\tl_set:Nn \l_tmpa_tl {hello}
\test:n {\l_tmpa_tl}
\cs_generate_variant:Nn \test:n {o}
\test:o {\l_tmpa_tl}
% how 'exp_args:Nx' will be help for ? see below
\cs_generate_variant:Nn \test:n {v}
\def\tlname{l_tmpa_tl}
\exp_args:Nx\test:v {\tlname}
% or 
% \test:o \l_tmpa_tl ==> working
% \test:V \l_tmpa_tl ==> working
% \test:v {\tlname}  ==> cause error.

\subsection{f-type}
% 'f'-type expansion
\tl_new:N \l_tmpaa_tl
\tl_set:Nn \l_tmpa_tl { A }
\tl_set:Nn \l_tmpaa_tl { \l_tmpa_tl }
\tl_set:Nn \l_tmpb_tl { B }
% 'x' type
\tl_set:Nx \l_tmpc_tl { \l_tmpaa_tl \l_tmpb_tl }
\quad\;(`x'-expand)\cs_meaning:N \l_tmpc_tl\par

% 'f' type expand
\tl_set:Nf \l_tmpc_tl { \l_tmpaa_tl \l_tmpb_tl }
(`f'-expand)\cs_meaning:N \l_tmpc_tl\par

% expans step by step
% expand once
\tl_set:No \l_tmpc_tl { \l_tmpaa_tl \l_tmpb_tl }
(`o'-type~ once)\cs_meaning:N \l_tmpc_tl\par
% expand twice
\tl_set:No \l_tmpc_tl { \l_tmpa_tl \l_tmpb_tl }
(`o'-type~ twice)\cs_meaning:N \l_tmpc_tl\par
% expand third
\tl_set:No \l_tmpc_tl { \l_tmpa_tl \l_tmpb_tl }
(`o'-type~ third)\cs_meaning:N \l_tmpc_tl
\ExplSyntaxOff


\section{argument specifier}
Every function must include an \textbf{argument specifier}. For functions which take no
arguments, this will be blank and the function name will end `:'.

\def\c#1{\textcolor{red}{\texttt{#1}}}
\def\diff#1{\textcolor{blue}{\texttt{#1}}}
\begin{itemize}
  \item \c{N} and \c{n}:means \diff{no manipulation}, pass the argument through \textbf{exactly as given}
  \item \c{c}: means \diff{csname}, argument be turned into a csname. like \verb|\csname mycmd \csname|.
  \item \c{V} and \c{v}:means \diff{value of variable}, get the content of a variable
    A \c{V} argument will be a single token (similar to \c{N}), using \c{v} a csname is
    \textbf{constructed first}, and then the \textbf{value} is recovered. So \verb|\foo:V \MyVariable| 
    = \verb|\foo:v {MyVariable}|. You can only define a \textbf{base function} argument is \c{n} instead 
    of \c{v, x, e, o} etc. Then generiant the variant(only \c{n} can be variant).
  \item \c{o}:means \diff{expansion once}(Is it equvilant to expand only the first macro once?), while that \c{V,v} are prefered.
  \item \c{x}:means \diff{exhaustive expansion}, fully expand every macro(until only unexpandable ones remain). 
    Functions which feature an \c{x}-type argument \textbf{are not expandable}.
  \item \c{e}:means \diff{expandable functions}, The \c{e} specifier is in many respects identical to \c{x},
    Functions which feature an \c{e}-type argument \textbf{may be expandable}. Besides, Parameter character 
    (usually \#) in the argument need not be doubled (when in nest loop). 
  \item \c{f}:means \diff{full(first) expansion}. The \c{f} specifier stands for full expansion, and in contrast to \c{x}, this type 
    \textbf{stops at the first non-expandable token}(<space> is gobbled) (reading the argument from left to right) without trying 
    to expand it.
  \item \c{T, F}: \diff{Ture/False} in logic test.
  \item \c{p}: means \diff{\TeX{} parameters}, like \verb|\cs_set:Npn \foo:n #1{\textbf{#1}}|
  \item \c{w}: means \diff{weird arguments}, This covers everything else, but mainly applies to delimited values
\end{itemize}
\end{document}
