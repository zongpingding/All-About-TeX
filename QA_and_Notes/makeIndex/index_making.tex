\documentclass{article}
\usepackage[margin=2.5in]{geometry}
\usepackage{makeidx,showidx,multicol,microtype}


\newcommand\bs{\symbol{’134}}
% print backslash in either OT1 or T1
\newcommand\Com[1]{\texttt{\bs#1}\index{#1@\texttt{\bs#1}}}
\newcommand\Prog[1]{\texttt{#1}\index{#1@\texttt{#1} program}}
\newcommand\MakeIndex{\texttt{makeindex}}


\renewenvironment{theindex}
{\begin{multicols}{2}[\section*{\indexname}][5\baselineskip]%
\addcontentsline{toc}{section}{\indexname}%
\setlength\parindent{0pt}\pagestyle{plain}%
\ExpandArgs{Nc}\RenewCommandCopy{\item}{@idxitem}}%
{\end{multicols}}


\makeindex
\begin{document}
\section{Generating an Index}
Using the \textsf{showidx} package, users can see where they define
index entries.

Entries are entered into the index by the \Com{index} command. More
precisely, the argument of the \Com{index} command is written
literally into the auxiliary file \texttt{idx}. Note, however, that
information is actually written into that file only when the
\Com{makeindex} command was given in the document preamble.

\section{Preparing the Index}
In order to prepare the index for printing, the \texttt{idx} file has
to be transformed by an external program, like \MakeIndex{}.
This program writes the \texttt{ind} file.
\begin{verbatim}
makeindex filename
\end{verbatim}

\section{Printing the Index}\index{Final production run}
During the final production run of a document, the index can be
\index{include index}included by putting a \Com{printindex} command at
the position in the text where you want the index to appear (normally
at the end).This command will input the \texttt{ind} file prepared by
\MakeIndex{}, and \LaTeX{} will typeset the information.
\printindex
\end{document}