\documentclass[10pt,varwidth]{standalone}
\usepackage{geometry}
\geometry{a4paper,left=1in,right=.3in,top=1in,bottom=1.2in}
\usepackage{pgfplots}
\usepackage{pgfplotstable}
\pgfplotsset{compat=1.16}
\usetikzlibrary {backgrounds, mindmap, shadows, calendar}

% First comes the definition of the \lecture command
\def\lecture#1#2#3#4#5#6{
    % As before:
    \node [annotation, #3, scale=0.65, text width=4cm, inner sep=2mm, fill=white] at (#4) {
        Lecture #1: \textcolor{orange}{\textbf{#2}}
        \list{--}{\topsep=2pt\itemsep=0pt\parsep=0pt
            \parskip=0pt\labelwidth=8pt\leftmargin=8pt
            \itemindent=0pt\labelsep=2pt}
        #5
        \endlist
    };
    % New:
    \node [anchor=base west] at (cal-#6.base east) {\textcolor{orange}{\textbf{#2}}};
}


\begin{document}
% This is followed by the main mindmap setup
\noindent
\begin{tikzpicture}
    \begin{scope}[
            mindmap,
            every node/.style={concept, circular drop shadow,execute at begin node=\hskip0pt},
            root concept/.append style={
            concept color=black,
            fill=white, line width=1ex,
            text=black, font=\large\scshape},
            text=white,
            computational problems/.style={concept color=red,faded/.style={concept color=red!50}},
            computational models/.style={concept color=blue,faded/.style={concept color=blue!50}},
            measuring complexity/.style={concept color=orange,faded/.style={concept color=orange!50}},
            solving problems/.style={concept color=green!50!black,faded/.style={concept color=green!50!black!50}},
            grow cyclic,
            level 1/.append style={level distance=4.5cm,sibling angle=90,font=\scshape},
            level 2/.append style={level distance=3cm,sibling angle=45,font=\scriptsize}
        ]
        % ... and contents
        \node [root concept] (Computational Complexity) {Computational Complexity} % root
            child [computational problems] { node [yshift=-1cm] (Computational Problems) {Computational Problems}
                child { node (Problem Measures) {Problem Measures} }
                child { node (Problem Aspects) {Problem Aspects} }
                child [faded] { node (problem Domains) {Problem Domains} }
                child { node (Key Problems) {Key Problems} }
        }
        child [computational models] { node [yshift=-1cm] (Computational Models) {Computational Models}
            child { node (Turing Machines) {Turing Machines} }
            child [faded] { node (Random-Access Machines) {Random-Access Machines} }
            child { node (Circuits) {Circuits} }
            child [faded] { node (Binary Decision Diagrams) {Binary Decision Diagrams} }
            child { node (Oracle Machines) {Oracle Machines} }
            child { node (Programming in Logic) {Programming in Logic} }
        }
        child [measuring complexity] { node [yshift=1cm] (Measuring Complexity) {Measuring Complexity}
            child { node (Complexity Measures) {Complexity Measures} }
            child { node (Classifying Complexity) {Classifying Complexity} }
            child { node (Comparing Complexity) {Comparing Complexity} }
            child [faded] { node (Describing Complexity) {Describing Complexity} }
        }
            child [solving problems] { node [yshift=1cm] (Solving Problems) {Solving Problems}
            child { node (Exact Algorithms) {Exact Algorithms} }
            child { node (Randomization) {Randomization} }
            child { node (Fixed-Parameter Algorithms) {Fixed-Parameter Algorithms} }
            child { node (Parallel Computation) {Parallel Computation} }
            child { node (Partial Solutions) {Partial Solutions} }
            child { node (Approximation) {Approximation} }
        };
    \end{scope}
    % Now comes the calendar code
    \tiny
    \calendar [day list downward,
        month text=\%mt\ \%y0,
        month yshift=3.5em,
        name=cal,
        at={(-.5\textwidth-5mm,.5\textheight-1cm)},
        dates=2009-04-01 to 2009-06-last
    ]
    if (weekend)[black!25]
    if (day of month=1) {
        \node at (0pt,1.5em) [anchor=base west] {\small\tikzmonthtext};
    };
    % The lecture annotations
    \lecture{1}{Computational Problems}{above,xshift=-5mm,yshift=5mm}{Computational Problems.north}{
        \item Knowledge of several key problems
        \item Knowledge of problem encodings
        \item Being able to formalize problems
    }{2009-04-08}
    \lecture{2}{Computational Models}{above left}
        {Computational Models.west}{
        \item Knowledge of Turing machines
        \item Being able to compare the computational power of different
        models
    }{2009-04-15}
    % Finally, the background
    \begin{pgfonlayer}{background}
        \clip[xshift=-1cm] (-.5\textwidth,-.5\textheight) rectangle ++(\textwidth,\textheight);
        \colorlet{upperleft}{green!50!black!25}
        \colorlet{upperright}{orange!25}
        \colorlet{lowerleft}{red!25}
        \colorlet{lowerright}{blue!25}
        % The large rectangles:
        \fill [upperleft] (Computational Complexity) rectangle ++(-20,20);
        \fill [upperright] (Computational Complexity) rectangle ++(20,20);
        \fill [lowerleft] (Computational Complexity) rectangle ++(-20,-20);
        \fill [lowerright] (Computational Complexity) rectangle ++(20,-20);
        % The shadings:
        \shade [left color=upperleft,right color=upperright]
            ([xshift=-1cm]Computational Complexity) rectangle ++(2,20);
        \shade [left color=lowerleft,right color=lowerright]
            ([xshift=-1cm]Computational Complexity) rectangle ++(2,-20);
        \shade [top color=upperleft,bottom color=lowerleft]
            ([yshift=-1cm]Computational Complexity) rectangle ++(-20,2);
        \shade [top color=upperright,bottom color=lowerright]
            ([yshift=-1cm]Computational Complexity) rectangle ++(20,2);
    \end{pgfonlayer}
\end{tikzpicture}

\end{document}