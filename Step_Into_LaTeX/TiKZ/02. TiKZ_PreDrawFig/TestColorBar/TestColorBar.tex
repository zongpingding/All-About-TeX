\documentclass[fontset=windows, 12pt]{article}
\usepackage{geometry}
\usepackage[dvipsnames,svgnames]{xcolor}
\usepackage[strict]{changepage} % 提供一个 adjustwidth 环境
\usepackage{framed} % 实现方框效果
\usepackage{ctex}
\usepackage{amsmath}
\geometry{a4paper,centering,scale=0.8}
% environment derived from framed.sty: see leftbar environment definition
% 文本框颜色
\definecolor{innercolor}{rgb}{0.95,0.95,1} 
% 边界框颜色
\definecolor{bordercolor}{RGB}{200, 0, 0} 
%% 常用的颜色
\definecolor{greenshade}{rgb}{0.90,0.99,0.91}   % 绿色文本框,竖线颜色设为 Green
\definecolor{redshade}{rgb}{1.00,0.90,0.90}     % 红色文本框,竖线颜色设为 LightCoral
\definecolor{brownshade}{rgb}{0.99,0.97,0.93}   % 莫兰迪棕色,竖线颜色设为 BurlyWood


% 定义一个formal样式环境;#1参数代表内部颜色,#2代表左边竖条颜色
% 注意:行末需要把空格注释掉,不然画出来的方框会有空白竖线
\newenvironment{formal}[2][innercolor]{%
    \def\FrameCommand{%
        \hspace{1pt}%
        % 左侧颜色条细线的宽度
        {\color{#2}\vrule width 2pt}%
        % 文本距离左侧颜色条的边距
        {\color{#1}\vrule width 4pt}%
        \colorbox{#1}%
    }%
    \MakeFramed{\advance\hsize-\width\FrameRestore}%
    \noindent\hspace{-4.55pt}% disable indenting first paragraph
    \begin{adjustwidth}{}{7pt}%
    \vspace{2pt}\vspace{2pt}%
}
{%
    \vspace{2pt}\end{adjustwidth}\endMakeFramed%
}



\begin{document}

\section{测试}

\subsection*{普通文本测试}
\begin{formal}{blue!70}
This is a Test!

这是一个测试
\end{formal}


\subsection*{公式测试}
\begin{formal}[green!30]{bordercolor}
下边开始证明:证明思路:1. 首先我们需要证明
5.证明解的唯一性

\textbf{证明1}:\par 
$
\begin{aligned}
    &1.~ \mbox{若} y = \varphi(x) \mbox{是方程1的解} ,\mbox{那么带入到方程1中即有}\\
    &\hspace*{10em} \frac{d \varphi(x)}{dx}=f(x, \varphi(x))\\
    &\mbox{现在我们两边同时取}[x_0, x] \mbox{上的定积分.为了避免混淆,变换积分变量}\\
    &\hspace*{10em}\int_{x_0}^{x}{\frac{d \varphi(t)}{dt} dt} = \int_{x_0}^{x}{f(t, \varphi(t)) dt}\\
    & \mbox{那么就可以得到}\\
    & \hspace*{10em} \varphi(x)-y_0(\mbox{原}\varphi(x_0)) = \int_{x_0}^{x}{f(t, \varphi(t)) dt}\\
    & \mbox{移项可以得到}\\
    & \hspace*{10em} \varphi(x) =  y_0 + \int_{x_0}^{x}{f(t, \varphi(t)) dt}\\
    &\mbox{由此说明}\varphi(x) \mbox{满足方程2}\\
    &2. ~ \mbox{若}y =  \varphi(x)\mbox{是2的解, 那么我们对方程2求导有:}\\
    & \hspace*{10em} \frac{d }{dx}[\varphi(x)] = \frac{d}{dx}\left[y_0 + \int_{x_0}^{x}{f(t, \varphi(t)) dt}\right]\\
    & \hspace*{10em} \mbox{即}~~\frac{d \varphi(x)}{dx}=f(x, \varphi(x))\\
    & \mbox{所以} y = \varphi(x) \mbox{方程1的解}
\end{aligned}
$
\end{formal}

\end{document}

