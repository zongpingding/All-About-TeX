\documentclass[fontset=windows, 12pt]{ctexart}
\usepackage{ctex}
\usepackage[all,pdf]{xy}
\usepackage{amsmath}
\usepackage{mathtools}
\usepackage{amssymb}
\usepackage{pstricks}
\usepackage{pst-plot, pst-node}
\usepackage{pstricks-add}


%% 只能够在Overleaf上编译通过。
%% sublime编译时不报错,但是无法生成pdf

\begin{document}
	\tableofcontents
	\section{绘制矩阵}
	$$
	\xymatrix{
		a & b & a+b\\
		1 & 2 & 3 \\
	}
	$$

	\section{绘制箭头}
	% u d l r-->上下左右
	$$
	\xymatrix{
		a & b\ar[rd] & a+b\\% b指向右下
		1 & 2 & 3\ar"1,1" % 3指向1.1
		\ar "1,1";"2,2" % 从1.1指向2.2
	}
	$$

	\section{箭头的上下标,间断插入符号(\^, \_, |)}
	$$
		\xymatrix{
			A\ar[r]^{\alpha} & B\ar[d]_{\beta}\\
			C\ar[ur]|{\Sigma} & D \\
		}
	$$

	\section{hole的运用,@1缩小矩阵间距,嵌入行内}
		$$
		\xymatrix@1{A\ar[r]|\hole & B}\\
		$$

		$$
		\xymatrix{
				A\ar[rd]|\hole & B \\
				C\ar[ur] & D \\
			}
		$$

	\section{绘制标签}
	\centering{
			位置\\
			(<:起点;>:中点)\\
			(<<,>>,因子$\in(0,1)$)
		}
	$$
	\xymatrix{
		A \ar[r]^>>{f} & B\\
		C \ar[r]_(0.2){\xi_i} & D
	}
	$$

	\begin{equation}
		\begin{gathered}
			\xymatrix{
				S \ar[r]^{f_s} \ar[d]_{\lambda} 
				& T \ar[d]^{\bar\lambda}\\
				S' \ar[r]_{f_{s'}} & T'\\
			}	
		\end{gathered}
	\end{equation}

	\subsection{@的作用}
	指定样式:$\backslash$ar@\{\}
	
	\section{中文框}
	$$
		\xymatrix{
			*++=[o][F]\txt{猫猫} \ar @{<->}[r] & 
			*+[F]\txt{狗\\狗}
			}
	$$
	\section{矩阵的旋转}
	\begin{equation}
		\begin{gathered}
			\xymatrix@ru{
				S \ar[r]^{f_s} \ar[d]_{\lambda} 
				& T \ar[d]^{\bar\lambda}\\
				S' \ar[r]_{f_{s'}} & T'\\
			}	
		\end{gathered}
	\end{equation}

	\section{基本的命令}
	\subsection{直线命令}
	直线1
	%\psline(0, 0)(1, 1em)(1.5, 0)
	直线2
	\begin{pspicture}(1.5, 1em)
		\psline(0, 0)(1, 1em)(1.5, 0)
	\end{pspicture}

	\section{绘制圆命令}
	\psset{linewidth=0.4pt}
	\begin{pspicture}(-1.2,-1.2)(1.2,1.2)
		\psaxes[labels=none,ticks=none]
		{->}(0,0)(-1.2,-1.2)(1.2,1.2) 	
		\pscircle[linewidth=0.8pt](0,0){1} 
	\end{pspicture}

	\section{绘制实战}
	\psset{linewidth=0.4pt}
	\begin{pspicture}(-1.2,-1.2)(1.2,1.2)
		\psaxes[labels=none,ticks=none]{->}(0,0)(-1.2,-1.2)(1.2,1.2) 
		\pscircle[linewidth=0.8pt](0,0){1}
		\pswedge[fillstyle=solid,fillcolor=gray,opacity=0.2]
			(0,0){1}{0}{120}
		\pswedge[fillstyle=solid,fillcolor=gray,opacity=0.5]
			(0,0){0.3}{0}{120}
		\uput[60](0.3;60){$120^\circ$}% 在扇形中间标注角度Apnode (1;120)
		\pnode(1;120){P}
		\pnode (P|0,0){PO} 
		\ncline{-}{P}{PO}				%正弦线
		\uput [120](P){$P$}
		\uput [d](PO){$P_0$} 
	\end{pspicture}

\end{document}


