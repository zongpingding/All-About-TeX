\PassOptionsToPackage{quiet}{fontspec}
\documentclass[zihao=-4, twoside]{ctexbook}
\usepackage{stys/Fancybook}



\begin{document}
\title{书籍标题}
\subtitle{副标题}
\edition{版本号}
\bookseries{书籍系列}
\author{作者}
\pressname{Springer} %出版社名称
\presslogo{figures/Springer-logo.png} %出版社徽标
\makecover
\tableofcontents
\newpage
\foottheme{数学历史漫谈}
\footimage{figures/newton.jpg}
\foottext{爱情既是友谊的代名词,又是我们为共同的事业而奋斗的可靠保证,爱情是人生的良伴,你和心爱的女子同床共眠是因为共同的理想把两颗心紧紧系在一起。\hfill ------ 法拉第}
\chapter{模板命令介绍}

\newpage
\foottheme{数学历史漫谈}  % 页脚主题
\footimage{figures/faraday.jpg} %页脚图片
\foottext{爱情既是友谊的代名词,又是我们为共同的事业而奋斗的可靠保证,爱情是人生的良伴.\hfill ------ 爱因斯坦} %页脚文字
\section{标题}
标题默认居中,subsection几乎不变。



\subsection{数学环境测试}
align环境
\begin{align}
  \sum_{i=1}^{+\infty}{\frac{1}{i^2}} = \frac{\pi^2}{6}
\end{align}

\subsection{练习}
\exercise[\bccrayon]{练习应用} % 练习标题

这个模板还定义了解答环境,相关的使用如下:

\begin{verbatim}
  \begin{solution}
    这里是解答,
    This is the Solution
  \end{solution}
\end{verbatim}

\begin{solution}
  这里是解答,This is the Solution:

  \[
    \forall x \in \mathcal{R}, \Rightarrow x \in \mathcal{C}  
  \]
\end{solution}


\newpage
\section{习题编写}
\foottheme{第二页}
\footimage{figures/newton.jpg}
\foottext{这里是第二页}

\examsection

examsection小节用于编写习题,具体的效果还是比较理想的

\end{document}
