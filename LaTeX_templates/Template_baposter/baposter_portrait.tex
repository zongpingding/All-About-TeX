%
% note: use pdflatex to compile
%
\documentclass[portrait,final,a0paper,fontscale=0.277]{baposter}
\usepackage{graphicx}
\usepackage{amsmath,amssymb}
%\usepackage{relsize}

\definecolor{lightblue}{rgb}{0.145,0.6666,1}

\begin{document}

\begin{poster} % Poster Options
{ grid=false, % Show grid to help with alignment
  colspacing=1em, % Column spacing
  bgColorOne=white, % Color style
  bgColorTwo=white,
  borderColor=lightblue,
  headerColorOne=black,
  headerColorTwo=lightblue,
  headerFontColor=white,
  boxColorOne=white,
  boxColorTwo=lightblue,
  textborder=roundedleft, % Format of textbox
  eyecatcher=true, % Format of text header
  headerborder=closed,
  headerheight=0.1\textheight,
  headershape=roundedright,
  headershade=shadelr,
  headerfont=\Large\bf\textsc, % Sans Serif
  textfont={\setlength{\parindent}{1.5em}}, % paragraph indentation
  boxshade=plain,
  background=plain,
  linewidth=1pt
}
%%%%%===== logo on the left
{\includegraphics[height=4em]{logo.png}}
%%%%%===== Title
{\bf\textsc{%
   Title Title Title Title Title}\vspace{0.5em}}
%%%%%===== Authors
{\textsc{author and institute}}
%%%%%===== logo on the right
{\includegraphics[height=4em]{logo.png}}


%%%%%%%%%%%%%%%%%%%%%%%%%%%%%%%%%%%%%%%%%%%%%%%%%%%%%%%%%%%%%%%%%%%%%%%%%%%%%%
%%%%%===== 第一列第一行(column=0,row=0)
\headerbox{Motivation}{name=motivation,column=0,row=0}{
  Here is the motivation.

  \vspace{8em}
}

%%%%%===== 第一列第二行
\headerbox{Contributions}{name=contribution,column=0,below=motivation}{
  Here is our contributions.

  \vspace{10em}
}

%%%%%===== 第一列第三行
\headerbox{Method}{name=method,column=0,below=contribution}{
  The cost is interpreted as a directed acyclic graph
  with weights on the nodes and edges. The nodes encode candidate positions,
  and the edges the transition costs between candidates. Additional edges
  (dashed) allow occlusion transitions which skip frames.

  The optimal track is found with a modification of Dijkstra's shortest path
  search. The search was speed up by lower bounding the cost, and lazily
  evaluating the accurate cost only where necessary to find the global optimum.

 \vspace{5em}
}

%%%%%===== 第一列第四行
\headerbox{Source Code}{name=source,column=0,below=method,above=bottom}{
  The source code and compiled executables with an interactive interface
  are available at http://math.ecnu.edu.cn
}

%%%%%===== 第二列第一行(column=1,row=0)
%%%%%%%%%%%%%%%%%%%%%%%%%%%%%%%%%%%%%%%%%%%%%%%%%%%%%%%%%%%%%%%%%%%%%%%%%%%%%%
\headerbox{Results}{name=results,column=1,span=1,row=0}{
    Between one and three user clicks were needed to achieve accurate tracking for
      the head sequence. Note the correct handling of the occluded ear, which
      required only a single click.

      The eye of the running giraffe required eight user interactions, of which three
      marked occlusions.
  \vspace{9em}
}


%%%%%===== 第二列第二行
\headerbox{Background Model}{name=background model,column=1,below=results}{
  We incorporate a background model, such that a click tells us not only `this
  is how the landmark looks like', but also `this is how the landmark
  does \emph{not} look like' for all other patches in that frame.

  \vspace{5em}
}

%%%%%===== 第二列第三行
\headerbox{A Future Direction}{name=questions,column=1,below=background model,above=bottom}{
  We incorporated a background model, where a click informs us not only that `this is how the
  patch looks like', but also for the rest of the frame, `this is how the patch
  does not look like'.

  Can we also \emph{efficiently} use a background tracks model, allowing us
  to reason, `this would be a good track, but part of it can be better
  explained by tracking another point'.

 \vspace{5em}
}

%%%%%%%%%%%%%%%%%%%%%%%%%%%%%%%%%%%%%%%%%%%%%%%%%%%%%%%%%%%%%%%%%%%%%%%%%%%%%%
%% 第三列第一行(column=2,row=0)
\headerbox{Comments}{name=comments,column=2,span=1,row=0}{
    Between one and three user clicks were needed to achieve accurate tracking for
      the head sequence. Note the correct handling of the occluded ear, which
      required only a single click.

      The eye of the running giraffe required eight user interactions, of which three
      marked occlusions.
  \vspace{15em}
}

%% 第三列第二行
\headerbox{Speed}{name=speed,column=2,below=comments}{
  Speed is achieved by preprocessing the video with an adaptive filter
  bank as in~\cite{Gol07,GV13}. Preprocessing was sped up significantly, but is
  still slower than realtime.

  This encodes the video into 16 byte per pixel feature vectors. We implemented
  an efficient search for similar patches using the SIMD hardware of modern
  processors, and only evaluate the cost on these candidate patches. (Typically
  200 patches per frame). The Graph-Structure focuses the evaluations on the most
  important areas, and makes candidate search and reasoning highly efficient,
  such that the system runs at interactive speed.

  Note that the preprocessing is not specific to the interestpoints tracked
  later. A single preprocessed video can therefore be used in many annotation
  sessions.
  \vspace{5em}
}

%% 第三列第二行
\headerbox{References}{name=references,column=2,below=speed,above=bottom}{
  \footnotesize \linespread{0.9}\selectfont

  \begin{thebibliography}{1}
    \bibitem{Gol07}
      G. H. Golub,
      \newblock History of numerical linear algebra: A personal view,
      \newblock Stanford, 2007.

    \bibitem{GV13}
      G. H. Golub and C. F. Van Loan,
      \newblock \emph{Matrix Computations},
      \newblock The 4th Editon, The Johns Hopkins University Press, Baltimore, MD, 2013.

  \end{thebibliography}
}
%%%%%%%%%%%%%%%%%%%%%%%%%%%%%%%%%%%%%%%%%%%%%%%%%%%%%%%%%%%%%%%%%%%%%%%%%%%%%%

\end{poster}

\end{document}
