% ======================================================================
% preface-de.tex
% Copyright (c) Markus Kohm, 2008-2022
%
% This file is part of the LaTeX2e KOMA-Script bundle.
%
% This work may be distributed and/or modified under the conditions of
% the LaTeX Project Public License, version 1.3c of the license.
% The latest version of this license is in
%   http://www.latex-project.org/lppl.txt
% and version 1.3c or later is part of all distributions of LaTeX
% version 2005/12/01 or later and of this work.
%
% This work has the LPPL maintenance status "author-maintained".
%
% The Current Maintainer and author of this work is Markus Kohm.
%
% This work consists of all files listed in MANIFEST.md.
% ======================================================================

\KOMAProvidesFile{preface-de.tex}
                 [$Date: 2022-06-05 12:40:11 +0200 (So, 05. Jun 2022) $
                  preface to a dedicated version]

\addchap{Vorwort zu \KOMAScript~3.36 und 3.37}

Mit \KOMAScript~3.36 wurde eine Phase größerer Umbauten an den Quellen von
\KOMAScript{} eingeläutet. Begonnen wurde mit den Quellen der Klassen und
Pakete. Dabei wurde nicht nur eine Umstellung auf Version~3 des Pakets
\Package{doc} durchgeführt. Die ursprüngliche Quellcode-Dokumentationsklasse
\Class{scrdoc} wurde auch obsolet und durch \Class{koma-script-source-doc}
ersetzt. Die komplette Dokumentation des Quellcodes wurde außerdem ins
Englische übertrage oder in Englisch neu erstellt. Im Zuge dieser Änderung
wurden die Quellen teilweise auch innerhalb der Dateien umorganisiert oder in
neue Dateien verschoben. Dabei entstand viele Dutzend neue Anmerkungen über
unerledigte Aufgaben. Ob ich die jemals alle selbst abarbeiten
kann, sei dahingestellt.

Durch die massiven Umbauten an den Quellen war von vornherein leider nicht
auszuschließen, dass sich neue Fehler einschleichen. Aufgrund des akuten
Mangels an Beta-Testern hat sich diese Befürchtung leider bewahrheitet. Ob all
diese Fehler inzwischen beseitigt sind, ist schwer zu sagen.

Mehr oder weniger nebenbei wurde eine neue, auf \Package{l3build} basierende
Teststruktur aufgebaut. Damit soll zukünftig sichergestellt werden, dass
einmal gemeldete Fehler künftig nicht wieder auftreten.

Mit \KOMAScript~3.37 wurde begonnen, die Quellen der Anleitung zu
restrukturieren. Für die Erzeugung der Deutschen und Englischen
Benutzeranleitung einschließlich der vollständigen Beispiele mit PDF wird nun
ebenfalls \Package{l3build} verwendet. Außerdem wird eine flache Hierarchie
für die Anleitungen in allen Sprachen verwendet. Es gibt dadurch in den
Quellen keine unterschiedlichen Dateien mit demselben Dateinamen
mehr. Dadurch sollen nicht nur Anforderungen von CTAN befriedigt werden. Damit
ist es auch erstmals seit langem wieder möglich, aus den CTAN-Quellen von
\KOMAScript{} selbst die Anleitungen zu erzeugen.

Aufgrund der bereits im Vorwort zu \KOMAScript~3.28 erklärten Probleme mit der
Endlichkeit der Zeit eines einzelnen Entwicklers, werde ich mich auch in
Zukunft weiterhin auf die Fehlerbehebung, die notwendige Reorganisation der
Quellen und die Kompatibilität mit neuen \LaTeX-Kernel-Versionen
konzentrieren. Vor allem bei letzterem steht mir inzwischen auch Marei
Peischl zur Seite, die eigentlich mit eigenen Projekten bereits sehr gut
ausgelastet ist. Von ihr stammt auch der ursprüngliche Code für die
Abbildungen zu Pseudolängen und Variablen, für den ich mich recht herzlich
bedanke. Damit war es mir endlich möglich, einem lange gehegten Wunsch vieler
Anwender nachzukommen. Mit wenigen Änderungen sind die Pseudolängen in der
Abbildung nun mit den zugehörigen Erklärungen im Text verlinkt.

Durch den weitgehenden Verzicht auf neue Funktionen schwindet natürlich auch
der Aufwand für die Dokumentation derselben. Leser dieser freien
Bildschirm-Version der Anleitung müssen aber auch weiterhin mit gewissen
Einschränkungen leben. So sind einige Informationen -- hauptsächliche solche
für fortgeschrittene Anwender oder die dazu geeignet sind, aus einem Anwender
einen fortgeschrittenen Anwender zu machen -- der Buchfassung vorbehalten. Das
führt auch dazu, dass weiterhin einige Links in dieser Anleitung lediglich zu
einer Seite führen, auf der genau diese Tatsache erwähnt ist. Darüber hinaus
ist die freie Version nur eingeschränkt zum Ausdruck geeignet. Der Fokus liegt
vielmehr auf der Verwendung am Bildschirm parallel zur Arbeit an einem
Dokument. Sie hat auch weiterhin keinen optimierten Umbruch, sondern ist quasi
ein erster Entwurf, bei dem Absatz- und Seitenumbruch in einigen Fällen
durchaus dürftig sind. Entsprechende Optimierungen bleiben den Buchausgaben
vorbehalten.

Mein Dank geht hauptsächlich an meine Familie und allen voran an meine
Frau. Sie federn all meine unschönen Erfahrungen im Internet ab. Ebenso
erdulden sie seit teilweise mehr als 25~Jahren, wenn ich wieder einmal nicht
ansprechbar bin, weil ich ganz und gar in \KOMAScript{} oder irgendwelche
\LaTeX-Probleme vertieft bin. Dass ich es mir leisten kann, überhaupt geradezu
wahnsinnig viel Zeit in ein derartiges Projekt zu investieren, ist allein
meiner Frau zu verdanken.

\bigskip\noindent
Markus Kohm, Neckarhausen im Mai 2022
\endinput

%%% Local Variables: 
%%% mode: latex
%%% TeX-master: "scrguide-de.tex"
%%% coding: utf-8
%%% ispell-local-dictionary: "de_DE"
%%% eval: (flyspell-mode 1)
%%% End: 

% LocalWords:  Dokumentationsklasse Teststruktur Benutzeranleitung
% LocalWords:  Pseudolängen Buchausgaben
