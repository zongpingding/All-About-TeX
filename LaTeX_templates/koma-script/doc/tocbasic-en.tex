% ======================================================================
% tocbasic-en.tex
% Copyright (c) Markus Kohm, 2002-2023
%
% This file is part of the LaTeX2e KOMA-Script bundle.
%
% This work may be distributed and/or modified under the conditions of
% the LaTeX Project Public License, version 1.3c of the license.
% The latest version of this license is in
%   http://www.latex-project.org/lppl.txt
% and version 1.3c or later is part of all distributions of LaTeX 
% version 2005/12/01 or later and of this work.
%
% This work has the LPPL maintenance status "author-maintained".
%
% The Current Maintainer and author of this work is Markus Kohm.
%
% This work consists of all files listed in MANIFEST.md.
% ======================================================================
%
% Package tocbasic for Package and Class Authors
% Maintained by Markus Kohm
%
% ======================================================================

\KOMAProvidesFile{tocbasic-en.tex}
                 [$Date: 2023-04-04 10:50:12 +0200 (Di, 04. Apr 2023) $
                  KOMA-Script guide (package tocbasic)]

\translator{Markus Kohm\and Arndt Schubert\and Karl Hagen}

\chapter{Managing Content Lists with \Package{tocbasic}}
\labelbase{tocbasic}
\BeginIndexGroup%
\BeginIndex{Package}{tocbasic}%
\BeginIndex{}{table of contents}%
\BeginIndex{}{content list}%
\BeginIndex{}{file>extension}%
The main purpose of the \Package{tocbasic} package is to give authors of
packages and classes the ability to create their own tables or lists of
content, content lists for short, similar to the list of figures and the list
of tables, allowing classes and other packages some control over these lists.
The \Package{tocbasic} package also delegates control of the
language-dependent parts of these content lists to the
\Package{babel}\IndexPackage{babel} package (see \cite{package:babel}). Using
\Package{tocbasic} relieves package and class authors from the burden of
implementing such features themselves.

As a minor side effect, the package can also be used to define new floating
environments, or floating environments like non-floating environments for
reference objects. For more details, after you read about the basic commands
in the next four sections, see the example in \autoref{sec:tocbasic.example},
which is compactly summarized in \autoref{sec:tocbasic.declarenewtoc}.

\KOMAScript{} itself uses \Package{tocbasic} not only for the table of
contents but also for the already mentioned lists of figures and tables.

\section{Basic Commands}
\label{sec:tocbasic.basics}

The basic commands are primarily used to handle a list of all known file
extensions\textnote{file extension, content lists} that represent a table or
list of contents. We call these auxiliary files\Index{auxiliary
  file}\footnote{The term \emph{auxiliary file} here refers not to the main
  \File{aux} file but to the other internal files used indirectly via the
  \File{aux} file, e.\,g. the \File{toc} file, the \File{lof} file, or the
  \File{lot} file.} TOC files\textnote{TOC file}\Index[indexmain]{TOC file}
regardless of the file extension that is actually used. Entries to such files
are typically written using
\Macro{addtocontents}\important{\Macro{addtocontents},
  \DescRef{\LabelBase.cmd.addxcontentsline}}, or
\DescRef{\LabelBase.cmd.addxcontentsline}. There are also commands to perform
actions on all of these file extensions. Additionally, there are commands to
set or unset features for the file associated with a given extension.
Typically an file extension also has an owner\textnote{file owner}. This owner
may be a class or package, or an identifier of a category that the author of
the class or package using \Package{tocbasic} has chosen independently. For
example, \KOMAScript{} uses the category \texttt{float} as owner for the
\File{lof} and \File{lot} file extensions that are associated with the list of
figures and list of tables, respectively. For the table of contents,
\KOMAScript{} uses the file name of the class.

\begin{Declaration}
  \Macro{Ifattoclist}\Parameter{extension}\Parameter{then code}%
                     \Parameter{else code}
\end{Declaration}
This\ChangedAt{v3.28}{\Package{tocbasic}} command tests whether or not the
\PName{extension} already exists in the list of known file extensions. If the
\PName{extension} is already known, the \PName{then code} will be
executed. Otherwise, the \PName{else code} will be executed.
\begin{Example}
  Suppose you want to know if the file name extension ``\File{foo}'' is
  already in use in order to report an error because it cannot be used:
\begin{lstcode}
  \Ifattoclist{foo}{%
    \PackageError{bar}{%
      extension `foo' already in use%
    }{%
      Each extension may be used only
      once.\MessageBreak
      The class or another package already
      uses extension `foo'.\MessageBreak
      This error is fatal!\MessageBreak
      You should not continue!}%
  }{%
    \PackageInfo{bar}{using extension `foo'}%
  }
\end{lstcode}
\end{Example}
\EndIndexGroup


\begin{Declaration}
  \Macro{addtotoclist}\OParameter{owner}\Parameter{extension}
\end{Declaration}
This command adds the \PName{extension} to the list of known extensions. But
if the \PName{extension} is known already, an error is reported to prevent
duplicate use of the same \PName{extension}.

If you specify the optional \OParameter{owner} argument, the given
\PName{owner} for this file extension is also saved. If you omit the optional
argument, \Package{tocbasic} tries to determine the file name of the class or
package currently being processed and saves it as the owner.
This\textnote{Attention!} procedure only works if you call
\Macro{addtotoclist} while loading a class or package. It will fail if a user
calls \Macro{addtotoclist} afterwards. In this case, the owner is empty.

Note\textnote{Attention!} that passing an empty \PName{owner} argument is not
always the same as completely omitting the optional argument, including the
square brackets. An empty argument would always result in an empty owner.
\begin{Example}
  Suppose you want to add the extension ``\File{foo}'' to the list of known
  file extensions while loading your package with the file name
  ``\File{bar.sty}'':
\begin{lstcode}
  \addtotoclist{foo}
\end{lstcode}%
  This will add the extension ``\PValue{foo}'' with the owner
  ``\PValue{bar.sty}'' to the list of known extensions if this extension was
  not already in the list. If the class or another package has already added
  the extension, you will get the error:
\begin{lstoutput}
  Package tocbasic Error: file extension `foo' cannot be used twice

  See the tocbasic package documentation for explanation.
  Type  H <return>  for immediate help.
\end{lstoutput}
  If you press the ``\texttt{h}'' key followed by return, you will get the
  following help:
\begin{lstoutput}
  File extension `foo' is already used by a toc-file, while bar.sty
  tried to use it again for a toc-file.
  This may be either an incompatibility of packages, an error at a package,
  or a mistake by the user.
\end{lstoutput}

  Perhaps your package also provides a command that dynamically generates a
  content list. In this case, you should use the optional argument of
  \Macro{addtotoclist} to specify the owner.
\begin{lstcode}
  \newcommand*{\createnewlistofsomething}[1]{%
    \addtotoclist[bar.sty]{#1}%
    % Do something more to make this content list available
  }
\end{lstcode}
  Now if the user calls this command, for example with
\begin{lstcode}
  \createnewlistofsomething{foo}
\end{lstcode}
  this will add the extension ``\PValue{foo}'' with the owner
  ``\PValue{bar.sty}'' to the list of known extension or report an error, if
  the extension is already in use.
\end{Example}
You can specify any \PName{owner} you want, but it must be unique. For
example, if you were the author of the \Package{float} package, you could
specify the category ``\PValue{float}'' instead of ``\PValue{float.sty}'' as
the \PName{owner}. In this case, the \KOMAScript{}
options\important{\DescRef{maincls.option.listof}}%
\IndexOption{listof~=\PName{setting}} for the list of figures and the list of
tables would also affect your content lists because \KOMAScript{} associates
the file extensions ``\PValue{lof}'' for the list of figures and
``\PValue{lot}'' for the list of tables with the owner ``\PValue{float}'' and
sets the options for this owner.

By the way, the \hyperref[cha:scrhack]{\Package{scrhack}}%
\IndexPackage{scrhack}\important{\hyperref[cha:scrhack]{\Package{scrhack}}}
package contains patches for several packages, such as
\Package{float}\IndexPackage{float} or
\Package{listings}\IndexPackage{listings}, which provide their own content
lists. If you use \hyperref[cha:scrhack]{\Package{scrhack}}, among other
things, the respective file extensions will be added to the list of known file
extensions. Their \PName{owner} is also ``\PValue{float}''. This is the basic
building block, so to speak, allowing you to use the features of
\Package{tocbasic} and the \KOMAScript{} classes with these content lists as
well.%
\EndIndexGroup


\begin{Declaration}
  \Macro{AtAddToTocList}\OParameter{owner}\Parameter{commands}
\end{Declaration}
This command adds the \PName{commands} to an internal list of commands that
will be processed whenever a file extension with the specified \PName{owner}
is added to the list of known extensions with
\DescRef{\LabelBase.cmd.addtotoclist}. The optional argument is handled in the
same way as in the \DescRef{\LabelBase.cmd.addtotoclist} command. If you leave
the optional argument blank, the commands will always be executed, regardless
of the owner, every time a new file extension is added to the list of known
file extensions. Furthermore, while processing the \PValue{commands},
\Macro{@currext}\IndexCmd{@currext}\important{\Macro{@currext}} is set to the
extension of the extension currently being added.
\begin{Example}
  \Package{tocbasic} itself uses
\begin{lstcode}
  \AtAddToTocList[]{%
    \expandafter\tocbasic@extend@babel
    \expandafter{\@currext}%
  }
\end{lstcode}
  to add every file extension to the existing \Package{tocbasic} integration 
  with the \Package{babel} package.
\end{Example}

The two \Macro{expandafter} commands in the example are needed because the
argument of \DescRef{\LabelBase.cmd.tocbasic@extend@babel} has to be expanded.
See the description of \DescRef{\LabelBase.cmd.tocbasic@extend@babel} in
\autoref{sec:tocbasic.internals},
\DescPageRef{\LabelBase.cmd.tocbasic@extend@babel} for more information.%
\EndIndexGroup


\begin{Declaration}
  \Macro{removefromtoclist}\OParameter{owner}\Parameter{extension}
\end{Declaration}
This command removes the \PName{extension} from the list of known extensions.
If the optional argument, \OParameter{owner}, is given, the \PName{extension}
will only be removed if it was added by this \PName{owner}.
% TODO: Fix new translation
This also applies to the empty \PName{owner}.  If, on the other hand, no
\OParameter{owner} is specified at all and the square brackets are also
omitted, the owner is not tested and the \PName{extension} is removed
regardless of the owner.%
% :ODOT
\EndIndexGroup


\begin{Declaration}
  \Macro{doforeachtocfile}\OParameter{owner}\Parameter{commands}
\end{Declaration}
To this point, we've introduced commands provide additional security for class
and package authors, but also more overhead. With \Macro{doforeachtocfile},
you can reap the first benefit for this. This command lets you execute the
specified \PName{commands} for each file extension associated with an
\PName{owner}. While processing the \PName{commands}, \Macro{@currext} is the
extension of the current file. If you omit the optional \OParameter{owner}
argument, all file extensions are processed regardless of the owner. If the
optional argument is empty, on the other hand, only extensions with an empty
owner will be processed.
\begin{Example}
  If you want to output a list of all known file extensions to the terminal
  and to the \File{log} file, you can easily accomplish this:
\begin{lstcode}
  \doforeachtocfile{\typeout{\@currext}}
\end{lstcode}
  If, on the other hand, you only want to output the extensions owned by
  ``\PValue{foo}'', this too is easy:
\begin{lstcode}
  \doforeachtocfile[foo]{\typeout{\@currext}}
\end{lstcode}
\end{Example}
The \KOMAScript{} classes \Class{scrbook} and \Class{scrreprt} use this
command to optionally put a vertical skip or the chapter heading in content
lists using the \PValue{chapteratlist} feature. You can learn how to set this
feature in \autoref{sec:tocbasic.toc} on
\DescPageRef{\LabelBase.cmd.setuptoc}.%
\EndIndexGroup


\begin{Declaration}
  \Macro{tocbasicautomode}
\end{Declaration}
This command redefines \LaTeX{} kernel macro \Macro{@starttoc} for class and
package authors so that every time \Macro{@starttoc} is called, the specified
file extension is added to the list of known extensions, if it has not already
been added. It also uses \DescRef{\LabelBase.cmd.tocbasic@starttoc} instead of
\Macro{@starttoc}. See \autoref{sec:tocbasic.internals},
\DescPageRef{\LabelBase.cmd.tocbasic@starttoc} for more information about
\DescRef{\LabelBase.cmd.tocbasic@starttoc} and \Macro{@starttoc}.

After you use \Macro{tocbasicautomode}, every content list created with
\Macro{@starttoc} is automatically put under the control of
\Package{tocbasic}. Whether or not this leads to the desired result, however,
depends very much on the individual content list. The \Package{babel} package
extensions, at least, will work for all those content lists. Nevertheless, it
is preferable for the class or package authors to use \Package{tocbasic}
explicitly. That way they can also take advantage of the other features of
\Package{tocbasic}, which are described in the following sections.%
\EndIndexGroup


\section{Creating a Content List}
\label{sec:tocbasic.toc}

In the previous section, you learned how to maintain a list of known file
extensions and how to automatically execute commands when adding new
extensions to this list. You also saw a command that can be used to execute
instructions for all known extensions or all extensions belonging to one
owner. In this section, you will learn commands that apply to the files
associated with these file extensions.

\begin{Declaration}
  \Macro{addtoeachtocfile}\OParameter{owner}\Parameter{content}
\end{Declaration}
The \Macro{addtoeachtocfile} command uses the \LaTeX{} kernel command
\Macro{addtocontents} to write the \PName{content} to every TOC file\Index{TOC
file} in the list of known file extensions that has the specified
\PName{owner}. If you omit the optional argument, the \PName{content} is
written to each file in the list of known file extensions. Incidentally, the
actual file name is constructed from \Macro{jobname} and the file extension.
While writing the \PName{content},
\Macro{@currext}\IndexCmd{@currext}\important{\Macro{@currext}} is the
extension of the file currently being processed.
\begin{Example}
  You want to add a vertical space of one line to all TOC files.
\begin{lstcode}
    \addtoeachtocfile{%
      \protect\addvspace{\protect\baselineskip}%
    }
\end{lstcode}
  If, on the other hand, you want to do this only for the files that have the
  owner ``\PValue{foo}'', use:
\begin{lstcode}
    \addtoeachtocfile[foo]{%
      \protect\addvspace{\protect\baselineskip}%
    }
\end{lstcode}
\end{Example}
You should protect commands that should not be expanded when they are written
by prefixing them with \Macro{protect}, in the same way as you would in
\Macro{addtocontents}.
\EndIndexGroup


\begin{Declaration}
  \Macro{addxcontentsline}%
  \Parameter{extension}\Parameter{level}\OParameter{section number}%
  \Parameter{text}
\end{Declaration}
This command is very similar to
\Macro{addcontentsline}\IndexCmd{addcontentsline} from the \LaTeX{} kernel.
However, it has an additional optional argument for the \PName{section number}
of the entry, whereas for \Macro{addcontentsline}, it is specified in the
\PName{text} argument. It is used to include numbered or unnumbered entries in
the content list specified by the file \PName{extension}, where \PName{level}
is the sectioning level and \PName{text} is the content of the corresponding
entry. The page number is determined automatically.

In contrast to \Macro{addcontentsline}, \Macro{addxcontentsline} first tests
whether the \Macro{add\PName{level}\PName{extension}entry} command is defined.
If so, it will be used for the entry, passing the \PName{section number} as an
optional argument and \PName{text} as a mandatory argument. You can find an
example of such a command provided by the \KOMAScript{} classes in
\DescRef{maincls-experts.cmd.addparttocentry} (see
\autoref{sec:maincls-experts.toc},
\DescPageRef{maincls-experts.cmd.addparttocentry}). If the corresponding
command is undefined, the internal command
\DescRef{\LabelBase.cmd.tocbasic@addxcontentsline} is used instead. This takes
all four arguments as mandatory arguments and then uses
\Macro{addcontentsline} to create the desired entry. You can find more about
\DescRef{\LabelBase.cmd.tocbasic@addxcontentsline} in
\autoref{sec:tocbasic.internals},
\DescPageRef{\LabelBase.cmd.tocbasic@addxcontentsline}.

One advantage of using \Macro{addxcontentsline} rather than
\Macro{addcontentsline} is that the \PValue{numberline} feature is respected
(see \DescPageRef{\LabelBase.cmd.setuptoc}). Furthermore, you can configure
the form of the entries by defining the appropriate commands specific to the
\PName{level} and file \PName{extension}.%
%
\EndIndexGroup


\begin{Declaration}
  \Macro{addxcontentslinetoeachtocfile}\OParameter{owner}
                                       \Parameter{level}
                                       \OParameter{section number}%
                                       \Parameter{text}
  \Macro{addcontentslinetoeachtocfile}\OParameter{owner}
                                      \Parameter{level}\Parameter{text}%
\end{Declaration}
These two commands are directly related to
\DescRef{\LabelBase.cmd.addxcontentsline}\ChangedAt{v3.12}{\Package{tocbasic}}
explained above, or to \Macro{addcontentsline} from the \LaTeX kernel. The
difference is that these statements write \PName{text} not just to a single
file but to all the files of a given \PName{owner} and, if the first optional
argument is omitted, to all files in the list of known file extensions.
\begin{Example}
  Suppose you are a class author and you want to write the chapter entry not
  just in the table of contents but in all content-list files. Suppose further
  that \PValue{\#1} currently contains the text to be written.
\begin{lstcode}
  \addxcontentslinetoeachtocfile{chapter}%
                                [\thechapter]{#1}
\end{lstcode}
  In this case, of course, the current chapter number should be expanded
  directly when writing to the TOC file, which is why it was not protected
  from expansion with \Macro{protect}.
\end{Example}
While writing the \PName{content},
\Macro{@currext}\IndexCmd{@currext}\important{\Macro{@currext}} here is also
the extension of the file being written to, as it is with
\DescRef{\LabelBase.cmd.addtoeachtocfile}.%

Whenever\ChangedAt{v3.12}{\Package{tocbasic}} possible, the
\Macro{addxcontentslinetoeachtocfile} command is preferable to
\Macro{addcontentslinetoeachtocfile} because only the former offers the
enhancements of \DescRef{\LabelBase.cmd.addxcontentsline}. You can find more
about these enhancements and benefits in the explanation of
\DescRef{\LabelBase.cmd.addxcontentsline} above.%
%
\EndIndexGroup


\begin{Declaration}
  \Macro{listoftoc}\OParameter{list-of title}\Parameter{extension}%
  \Macro{listoftoc*}\Parameter{extension}%
  \Macro{listofeachtoc}\OParameter{owner}%
  \Macro{listof\PName{extension}name}
\end{Declaration}
You can use these commands to print the content lists.
The\important{\Macro{listoftoc*}} stared version \Macro{listoftoc*} needs only
one argument, the \PName{extension} of the file. The command first initializes
the vertical and horizontal spacing of paragraphs, calls the hook to execute
commands before reading the file, then reads the file, and finally executes
the hook to execute commands after reading the file. Thus you can think of
\Macro{listoftoc*} as a direct replacement for the \LaTeX{} kernel macro
\Macro{@starttoc}\IndexCmd{@starttoc}\important{\Macro{@starttoc}}.

The\important{\Macro{listoftoc}} version of \Macro{listoftoc} without the star
prints the complete content list and also creates an optional entry in the
table of contents and the running heads. If you provide the optional
\OParameter{list-of title} argument, it is used both as the title for the list
and as an optional entry in the table of contents and the running head.
If\textnote{Attention!} the \OParameter{list-of title} argument is empty, the
title will be empty too. If, on the other hand, you completely omit the
optional argument, including the square brackets, the
\Macro{listof\PName{extension}name} command will be used if it is defined. If
it is undefined, a default replacement name is used and a warning is issued.

The\important{\Macro{listofeachtoc}} \Macro{listofeachtoc} command outputs all
content lists associated with the given \PName{owner}, or of all known file
extensions if the optional argument is omitted. To\textnote{Attention!} output
the correct titles, \Macro{listof\PName{extension}name} should be defined.

You\textnote{Hint!} should define \Macro{listof\PName{extension}name}
appropriately for all file extensions because users may use \Macro{listoftoc}
without the optional argument, or \Macro{listofeachtoc}, themselves.
\begin{Example}
 Suppose you have a new ``list of algorithms'' with the file extension
 \PValue{loa} and want to output it. The command
\begin{lstcode}
  \listoftoc[List of Algorithms]{loa}
\end{lstcode}
  will do it for you. If, however, you want to output this list without a
  title, you could use
\begin{lstcode}
  \listof*{loa}
\end{lstcode}
  In the second case, of course, there will be no entry in the table of
  contents. For more information about creating entries in the table of
  contents, see the \DescRef{\LabelBase.cmd.setuptoc} command on
  \DescPageRef{\LabelBase.cmd.setuptoc}.

  If you have defined
\begin{lstcode}
  \newcommand*{\listofloaname}{%
    List of Algorithms%
  }
\end{lstcode}
  already, then
\begin{lstcode}
  \listoftoc{loa}
\end{lstcode}
  will suffice to print the content list with the desired heading. It may be
  easier for users to remember if you also define a simple list-of command:
\begin{lstcode}
  \newcommand*{\listofalgorithms}{\listoftoc{loa}}
\end{lstcode}
\end{Example}

Because\textnote{Attention!} \LaTeX{} normally opens a new file for each of
these content lists, calling each of these commands may
result in an error like:
\begin{lstoutput}
  ! No room for a new \write .
  \ch@ck ...\else \errmessage {No room for a new #3}
                                                    \fi
\end{lstoutput}
if there are no more write handles left. Loading the
\Package{scrwfile}\important{\Package{scrwfile}}\IndexPackage{scrwfile}
package \cite{package:scrwfile} can solve this problem.

Also, the \hyperref[cha:scrhack]{\Package{scrhack}}\IndexPackage{scrhack}%
\important{\hyperref[cha:scrhack]{\Package{scrhack}}} package contains patches
for several packages, such as \Package{float}\IndexPackage{float} or
\Package{listings}\IndexPackage{listings}, so that their content-list commands
can use \Macro{listoftoc}. As a result, many features of \Package{tocbasic}
and the \KOMAScript{} classes are also available for their content lists.%
\EndIndexGroup


\begin{Declaration}
  \Macro{BeforeStartingTOC}\OParameter{extension}\Parameter{commands}%
  \Macro{AfterStartingTOC}\OParameter{extension}\Parameter{commands}
\end{Declaration}
Sometimes it's useful to be able to execute \PName{commands} immediately
before reading the auxiliary TOC file. With \Macro{BeforeStartingTOC} you can
do so either for a single file \PName{extension} or for all files that are
read using \DescRef{\LabelBase.cmd.listoftoc*},
\DescRef{\LabelBase.cmd.listoftoc}, or \DescRef{\LabelBase.cmd.listofeachtoc}.
You can also execute \PName{commands} after reading the file if you define
them with \Macro{AfterStartingTOC}. If you omit the optional argument (or set
an empty one) a general hook will be set and the commands will be applied to
all content lists. The general before-hook is called before the individual
one, and the general after-hook is called after the individual one. While
executing the commands in these hooks,
\Macro{@currext}\IndexCmd{@currext}\important{\Macro{@currext}} is the
extension of the TOC file which is about to be or has just been read.%
\EndIndexGroup


\begin{Declaration}
  \Macro{BeforeTOCHead}\OParameter{extension}\Parameter{commands}%
  \Macro{AfterTOCHead}\OParameter{extension}\Parameter{commands}
\end{Declaration}
You can also define \PName{commands} that will be executed immediately before
or after setting the title of a content list using
\DescRef{\LabelBase.cmd.listoftoc*} or \DescRef{\LabelBase.cmd.listoftoc}. The
treatment of the optional parameter and the meaning of
\Macro{@currext}\IndexCmd{@currext}\important{\Macro{@currext}} is the same as
described for \DescRef{\LabelBase.cmd.BeforeStartingTOC} and
\DescRef{\LabelBase.cmd.AfterStartingTOC}.%
\EndIndexGroup


\begin{Declaration}
  \Macro{MakeMarkcase}\Parameter{text}
\end{Declaration}
Whenever \Package{tocbasic} sets a mark for a running head, The text of the
mark will be an argument of \Macro{MakeMarkcase}. You can use this command to
change the case of the running head if necessary. The \KOMAScript{} classes
use \Macro{@firstofone}\IndexCmd{@firstofone}\important{\Macro{@firstofone}}
by default. This means the text of the running head will be set without
changing the capitalisation. If you use a different class,
\Macro{MakeMarkcase} will be set to
\Macro{MakeUppercase}\IndexCmd{MakeUppercase}%
\important{\Macro{MakeUppercase}}. However, \Package{tocbasic} only defines
this command if it is not already defined. It can therefore be predefined by
another class or package and \Package{tocbasic} will use that definition
rather than overwriting it.
\begin{Example}
  For some strange reason, you want to set the running heads entirely in
  lower-case letters. To apply this change automatically for all running heads
  set by \Package{tocbasic}, you define:
\begin{lstcode}
  \let\MakeMarkcase\MakeLowercase
\end{lstcode}
\end{Example}
Let\textnote{Hint!} me give you some advice about
\Macro{MakeUppercase}\IndexCmd{MakeUppercase}. First of all, this command is
not fully expandable. This means that it can cause problems interacting with
other commands. Beyond that, typographers agree that whenever you set whole
words or phrases in capital letters, additional spacing is absolutely
necessary. However, adding a fixed spacing between all letters is not an
adequate solution. Different pairs of letters require different spaces between
them. Additionally, some letters already create gaps in the text that must be
taken into account. Packages like \Package{ulem} or \Package{soul} can
scarcely achieve this, nor can \Macro{MakeUppercase}. The automatic letter
spacing using the \Package{microtype} package is in this respect only an
approximate solution, because it does not take into account the concrete,
font-dependent glyphs. Because\textnote{Attention!} typesetting all-capital
text is expert work and almost always requires manual adjustment, ordinary
users are recommended avoid using it, or to use it only sparingly and not in
such an exposed place as the running head.%
\EndIndexGroup


\begin{Declaration}
  \Macro{deftocheading}\Parameter{extension}\Parameter{definition}
\end{Declaration}
The \Package{tocbasic} package contains a default definition for typesetting
the titles of content lists. You can configure this default definition through
various features discussed in the \DescRef{\LabelBase.cmd.setuptoc} comand
below. If those features are not enough, you can use \Macro{deftocheading}
to define your own title for the content list with the given file
\PName{extension}. The \PName{definition} of the title can use a single
parameter, \PValue{\#1}. When the command is called inside
\DescRef{\LabelBase.cmd.listoftoc} or \DescRef{\LabelBase.cmd.listofeachtoc},
that \PValue{\#1} will be replaced by the title of this content list.

Obviously, the \PName{definition} is also responsible for the interpretation
of additional features related to the title. This is especially true
for the features \PValue{leveldown}, \PValue{numbered} and \PValue{totoc},
that are be explained next.%
\EndIndexGroup


\begin{Declaration}
  \Macro{setuptoc}\Parameter{extension}\Parameter{feature list}%
  \Macro{unsettoc}\Parameter{extension}\Parameter{feature list}
\end{Declaration}
You can use these two commands to set and clear features for a TOC file
\PName{extension} or the content list associated with it. The \PName{feature
list} is a comma-separated list of features. The \Package{tocbasic} package
recognizes following features:
\begin{description}
\item[\PValue{leveldown}] means that the content list's heading is created not
  with the highest sectioning level below
  \DescRef{maincls.cmd.part}\,---\,\DescRef{maincls.cmd.chapter} if available,
  \DescRef{maincls.cmd.section} otherwise\,---\,but with a heading of the next
  level below that. This feature is evaluated by the internal heading command.
  On the other hand, if a user-defined heading command has been created with
  \DescRef{\LabelBase.cmd.deftocheading}, the person defining that command is
  responsible for evaluating the feature. The \KOMAScript{} classes set this
  feature using the \OptionValueRef{maincls}{listof}{leveldown}%
  important{\OptionValueRef{maincls}{listof}{leveldown}}%
  \IndexOption{listof~=\textKValue{leveldown}} option for all file extensions
  with the owner \PValue{float}.
\item[\PValue{nobabel}] prevents using the language-switching features of
  \Package{babel}\IndexPackage{babel} for the TOC file associated with the
  this file \PName{extension}. This feature should be used only for content
  lists that are created in a fixed language, which means that changes of the
  language inside of the document will no longer affect the content list. The
  \Package{scrwfile}\important{\Package{scrwfile}}\IndexPackage{scrwfile}
  \cite{package:scrwfile} package also uses this feature for cloned
  destinations, as because those files already inherit any language changes
  from the cloning source.

  Note\textnote{Attention!} that you must set this feature before you add a
  file extension to the list of known extensions. Changing the feature
  afterwards will have no effect.
\item[\PValue{noindent}]\ChangedAt{v3.27}{\Package{tocbasic}}%
  causes all content-list entry styles provided by \KOMAScript{} to ignore
  the \PValue{indent} attribute (see
  \autoref{tab:tocbasic.tocstyle.attributes},
  \autopageref{tab:tocbasic.tocstyle.attributes.indent}) and instead to
  deactivate the indentation.%
\item[\PValue{noparskipfake}] prevents\ChangedAt{v3.17}{\Package{tocbasic}}
  the insertion of a final \Length{parskip} before switched off paragraph
  spacing for content lists. In general, this will cause documents that use
  spacing between paragraphs to have less vertical space between the list
  heading and first entry than between normal headings and normal text.
  Normally, therefore, you will obtain a more uniform formatting without this
  feature.
\item[\PValue{noprotrusion}] prevents\ChangedAt{v3.10}{\Package{tocbasic}}
  character protrusion, which allows optical margin alignment, from being
  disabled in the content lists. By default, character protrusion is disabled
  when the \Package{microtype}\IndexPackage{microtype} package, or another
  package that supports \Macro{microtypesetup}\IndexCmd{microtypesetup}, is
  loaded. So if you want optical margin alignment in the content lists, you
  must set this feature. Note\textnote{Attention!}, however, that character
  protrusion in content lists often results in incorrect results. This is a
  known issue with character protrusion.
\item[\PValue{numbered}] means that the heading for the content list should
  be numbered and included in the table of contents. This feature is evaluated
  by the internal heading command. However, if a user-defined heading command
  has been created with \DescRef{\LabelBase.cmd.deftocheading}, the person
  creating that definition is responsible evaluating the feature. The
  \KOMAScript{} classes set this feature using the
  \OptionValueRef{maincls}{listof}{numbered}%
  \important{\OptionValueRef{maincls}{listof}{numbered}}%
  \IndexOption{listof~=\textKValue{numbered}} option for all file extensions
  with the owner \PValue{float}.
\item[\PValue{numberline}] \leavevmode\ChangedAt{v3.12}{\Package{tocbasic}}%
  means that any entries made using \DescRef{\LabelBase.cmd.addxcontentsline}
  or \DescRef{\LabelBase.cmd.addxcontentslinetoeachtocfile}, where the
  optional argument for the number is missing or empty, will be provided with
  an empty \DescRef{\LabelBase.cmd.numberline} command. This usually results
  in these entries being left-aligned not with the number but with the text of
  the numbered entries of the same level.
  Using\ChangedAt{v3.20}{\Package{tocbasic}} the \PValue{tocline} entry style
  can have additional side effects. See the style attribute
  \Option{breakafternumber} and \Option{entrynumberformat} in
  \autoref{tab:tocbasic.tocstyle.attributes} starting on
  \autopageref{tab:tocbasic.tocstyle.attributes}.

  \KOMAScript{} classes set this feature for the file extensions with the
  owner \PValue{float} if you use the
  \OptionValueRef{maincls}{listof}{numberline}%
  \important{\OptionValueRef{maincls}{listof}{numberline}}%
  \IndexOption{listof~=\textKValue{numberline}} option and for the file
  extension \PValue{toc} if you use the
  \OptionValueRef{maincls}{toc}{numberline}%
  \important{\OptionValueRef{maincls}{toc}{numberline}}%
  \IndexOption{toc~=\textKValue{numberline}} option. Similarly, this feature
  is reset if you use \OptionValueRef{maincls}{listof}{nonumberline}%
  \important{\OptionValueRef{maincls}{listof}{nonumberline}}%
  \IndexOption{listof~=\textKValue{nonumberline}} or
  \OptionValueRef{maincls}{toc}{nonumberline}%
  \important{\OptionValueRef{maincls}{toc}{nonumberline}}%
  \IndexOption{toc~=\textKValue{nonumberline}}.
\item[\PValue{onecolumn}] \leavevmode\ChangedAt{v3.01}{\Package{tocbasic}}%
  means that this content list will automatically use \LaTeX's internal
  one-column mode with \Macro{onecolumn}\IndexCmd{onecolumn}.
  However\textnote{Attention!}, this applies only if this content list does
  not use the \PValue{leveldown}\important{\PValue{leveldown}} feature. The
  \KOMAScript{} classes \Class{scrbook} and \Class{scrreprt} activate this
  feature with \DescRef{\LabelBase.cmd.AtAddToTocList} (see
  \autoref{sec:tocbasic.basics}, \DescPageRef{\LabelBase.cmd.AtAddToTocList})
  for all content lists with the owner \PValue{float} or with themselves as
  owner. Thus, for example, the table of contents, the list of figures, and
  the list of tables are automatically printed in a single column for both
  these classes. The multicolumn mode of the
  \Package{multicol}\IndexPackage{multicol} package is expressly unaffected by
  this option.
\item[\PValue{totoc}] means that the title of content list should be included
  in the table of contents. This feature will be evaluated by the internal
  heading command. However, if an user-defined heading command has been
  created with \DescRef{\LabelBase.cmd.deftocheading}, the person defining
  that command is responsible for evaluating this feature. The \KOMAScript{}
  classes set this feature using the \OptionValueRef{maincls}{listof}{totoc}%
  \important{\OptionValueRef{maincls}{listof}{totoc}}%
  \IndexOption{listof~=\textKValue{totoc}} option for all file extensions with
  the owner \PValue{float}.
\end{description}
The \KOMAScript{} classes recognize an additional feature:
\begin{description}
\item[\PValue{chapteratlist}] ensures that an optional subdivision is added
  to the content list for each new chapter. By default, this subdivision is a
  vertical space. See \DescRef{maincls.option.listof}%
  \IndexOption{listof}\important{\DescRef{maincls.option.listof}} in
  \autoref{sec:maincls.floats}, \DescPageRef{maincls.option.listof} for more
  information about this option.
\end{description}
\begin{Example}
  Because \KOMAScript{} classes use \Package{tocbasic} for the list of figures
  and list of tables, there is another way to prevent chapter subdivisions in
  these lists:
\begin{lstcode}
  \unsettoc{lof}{chapteratlist}
  \unsettoc{lot}{chapteratlist}
\end{lstcode}

  If you want the chapter subdivisions for your own list that you have defined
  with the file \PName{extension} ``\PValue{loa}'' to use the same subdivision
  format used by the \KOMAScript{} classes, you can use
\begin{lstcode}
  \setuptoc{loa}{chapteratlist}
\end{lstcode}
  And if you also want classes that use \DescRef{maincls.cmd.chapter} as the
  top-level structure to use the one-column mode automatically, you can
  use
\begin{lstcode}
  \Ifundefinedorrelax{chapter}{}{%
    \setuptoc{loa}{onecolumn}%
  }
\end{lstcode}
  Using \DescRef{scrbase.cmd.Ifundefinedorrelax} requires the
  \Package{scrbase} package (see \autoref{sec:scrbase.if},
  \DescPageRef{scrbase.cmd.Ifundefinedorrelax}).

  Even\textnote{Hint!} if your package will be used with another class, it
  does not hurt to set these features. To the contrary, if another class also
  evaluates these features, then your package would automatically use the
  features of that class.
\end{Example}
As you can see, packages that use \Package{tocbasic} already support a wide
range of options for content lists that would otherwise require a great deal
of effort to implement and which are therefore missing in many packages.%
\EndIndexGroup


\begin{Declaration}
  \Macro{Iftocfeature}\Parameter{extension}\Parameter{feature}%
  \Parameter{then code}\Parameter{else code}
\end{Declaration}
You\ChangedAt{v3.28}{\Package{tocbasic}} can use this command to test if a
\PName{feature} was set for the given file \PName{extension}. If so the
\PName{then code} will be executed, otherwise the \PName{else code} will
be. This can be useful, for example, if you define your own heading command
using \DescRef{\LabelBase.cmd.deftocheading} but want to support the features
\PValue{totoc}, \PValue{numbered} or \PValue{leveldown}.%
\EndIndexGroup


\section{Configuring Content-List Entries}
\seclabel{tocstyle}%
\BeginIndexGroup
\BeginIndex{}{table of contents>entry}%
\Index{list of contents|\see{table of contents}}

Since\ChangedAt{v3.20}{\Package{tocbasic}} version~3.20, the
\Package{tocbasic}  package has been able not only to configure the tables or
lists of contents and their auxiliary files but also to influence their
entries. To do so, you can define new styles or you can use and configure one
of the predefined styles. In the medium term, \Package{tocbasic} will replace
the experimental \Package{tocstyle} package that never became an official part
of the \KOMAScript{} bundle. The \KOMAScript{} classes themselves have relied
completely on the \Package{tocbasic} entry styles since version~3.20.

\begin{Declaration}
  \Counter{tocdepth}
\end{Declaration}
Entries to content lists are usually hierarchical. For this purpose, each
entry level has a numerical value. The higher this value, the lower in the
hierarchy is this level. In the standard classes, for example, parts have the
level -1 and chapters have the value 0. The \LaTeX{} counter
\Counter{tocdepth} determines the deepest level that should be shown in the
table of contents and other content lists.

For example, the \Class{book} class sets \Counter{tocdepth} to 2, so entries
of the levels \PValue{part}, \PValue{chapter}, \PValue{section}, and
\PValue{subsection} are printed. Deeper levels like \PValue{subsubsection},
which has the numerical value 3, are not printed. Nevertheless the entries are
part of the auxiliary file for the table of contents.

Note that most \Package{tocbasic} entry styles, with the exception of
\PValue{gobble} (see \DescRef{\LabelBase.cmd.DeclareTOCStyleEntry}\iffree{}{,
later in this section}) observe \Counter{tocdepth}.%
\EndIndexGroup


\begin{Declaration}
  \Macro{numberline}\Parameter{entry number}%
  \Macro{usetocbasicnumberline}\OParameter{code}
\end{Declaration}
Although\ChangedAt{v3.20}{\Package{tocbasic}} the \LaTeX{} kernel
already defines a \Macro{numberline} command, the definition is not sufficient
for \Package{tocbasic}. Therefore \Package{tocbasic} defines its own commands
and sets \Macro{numberline} as needed using \Macro{usetocbasicnumberline} for
each content-list entry. Redefining \Macro{numberline}, therefore, is often
ineffective and may result in warnings if you use \Package{tocbasic}.

You can use the definition of \Package{tocbasic} by putting
\Macro{usetocbasicnumberline} into your document's preamble. The command first
checks if the current definition of \Macro{numberline}
uses certain important, internal commands of \Package{tocbasic}. If this is not
the case, \Macro{usetocbasicnumberline} redefines \Macro{numberline} and
executes \PName{code}. If you omit the optional argument, it issues
a message about the redefinition with \Macro{PackageInfo}. If
you just do not want such a message, use an empty optional argument.

Note\textnote{Attention!} that \Macro{usetocbasicnumberline} can change the
internal switch \Macro{@tempswa} globally!%
\EndIndexGroup


\begin{Declaration}
  \Macro{DeclareTOCStyleEntry}\OParameter{option list}\Parameter{style}%
                              \Parameter{entry level}
  \Macro{DeclareTOCStyleEntries}\OParameter{option list}\Parameter{style}%
                                \Parameter{entry level list}
\end{Declaration}
These\ChangedAt{v3.20}{\Package{tocbasic}} commands define or configure the
content-list entries of a given \PName{entry level}. The \PName{entry level}
argument is a symbolic name, e.\,g. \PValue{section}, for the entry to the
table of contents of the section level with the same name or \PValue{figure}
for an entry of a figure to the list of figures. A \PName{style} is assigned
to each \PName{entry level}. The \PName{style} has to be defined before using
it as an argument of \Macro{DeclareTOCStyleEntry} or
\Macro{DeclareTOCStyleEntries}. You can use the \PName{option list} to
configure the various, usually \PName{style}-dependent, attributes of the
entries.

Currently, \Package{tocbasic} defines the following entry styles:
\begin{description}
\item[\PValue{default}] defaults to a clone of the \PValue{dottedtocline}
  style. Class authors who use \Package{tocbasic} are encouraged to change
  this style to the default content-list style of the class using
  \Macro{CloneTOCStyle}. For example the \KOMAScript{} classes change
  \PValue{default} into a clone of \PValue{tocline}.
\item[\PValue{dottedtocline}] is similar to the style used by the standard
  classes \Class{book} and \Class{report} for the \PValue{section} down to
  \PValue{subparagraph} entry levels of the table of contents and for the
  entries at the list of figures or list of tables. It supports the attributes
  \PValue{level}, \PValue{indent}, and \PValue{numwidth}. The entries are
  indented by the value of \PValue{indent} from the left. The width of the
  entry number is given by the value of \PValue{numwidth}. For multi-line
  entries, the indent will be increased by the value of \PValue{numwidth} for
  the second and following lines. The page number is printed using
  \Macro{normalfont}\IndexCmd{normalfont}. The entry text and the page number
  are connected by a dotted line. \hyperref[fig:tocbasic.dottedtocline]%
  {Figure~\ref*{fig:tocbasic.dottedtocline}} illustrates the attributes of
  this style.
  \begin{figure}
    \centering
    \resizebox{.8\linewidth}{!}{%
      \begin{tikzpicture}
        \draw[color=lightgray] (0,2\baselineskip) -- (0,-2.5\baselineskip);
        \draw[color=lightgray] (\linewidth,2\baselineskip) --
                               (\linewidth,-2.5\baselineskip);
        \node (dottedtocline) at (0,0) [anchor=west,inner sep=0,outer sep=0]
        {%
          \hspace*{7em}%
          \parbox[t]{\dimexpr\linewidth-9.55em}{%
            \setlength{\parindent}{-3.2em}%
            \addtolength{\parfillskip}{-2.55em}%
            \makebox[3.2em][l]{1.1.10}%
            Text of an entry in the table of contents with the
            \PValue{dottedtocline} style and more than one line%
            \leaders\hbox{$\csname m@th\endcsname
              \mkern 4.5mu\hbox{.}\mkern 4.5mu$}\hfill\nobreak
            \makebox[1.55em][r]{12}%
          }%
        };
        \draw[|-|,color=gray,overlay] (0,0) --
                              node [anchor=north,font=\small] {
                                \PValue{indent}
                              }
                              (3.8em,0);
        \draw[|-|,color=gray,overlay] (3.8em,\baselineskip) -- 
                              node [anchor=south,font=\small] {
                                \PValue{numwidth}
                              }
                              (7em,\baselineskip);
        \draw[|-|,color=gray,overlay] (\linewidth,\ht\strutbox) -- 
                              node [anchor=south,font=\small] { 
                                \Macro{@tocrmarg} 
                              }
                              (\linewidth-2.55em,\ht\strutbox);
        \draw[|-|,color=gray,overlay] (\linewidth,-\baselineskip) -- 
                              node [anchor=north,font=\small] { 
                                \Macro{@pnumwidth} 
                              } 
                              (\linewidth-1.55em,-\baselineskip);
      \end{tikzpicture}%
    }
    \caption{Illustrations of some attributes of a TOC entry with the
      \PValue{dottedtocline} style}
    \label{fig:tocbasic.dottedtocline}
  \end{figure}
\item[\PValue{gobble}] is the simplest possible style. Entries in this style,
  regardless of the setting of \DescRef{\LabelBase.counter.tocdepth}%
  \IndexCounter{tocdepth}\important{\DescRef{\LabelBase.counter.tocdepth}},
  will never be printed. The style simply gobbles the
  entries, so to speak. It has the default \PValue{level} attribute, but
  it is never evaluated.
\item[\PValue{largetocline}] is similar to the style used by the standard
  classes for the \PValue{part} level. It supports the \PValue{level} and
  \PValue{indent} attributes only. The latter deviates from the standard
  classes, which do not support an indent of the \PValue{part} entries.

  A penalty is set to permit page breaks before an entry of an appropriate
  level. The entries will be indented by the value of \PValue{indent} from the
  left and printed with the font style \Macro{large}\Macro{bfseries}. If
  \DescRef{\LabelBase.cmd.numberline} is used, the number width is 3\Unit{em}.
  \DescRef{\LabelBase.cmd.numberline} is not redefined. The standard classes
  do not use \DescRef{\LabelBase.cmd.numberline} for \PName{part} entries. The
  value of \PName{indent} also has no effect on the indentation from the
  second line and after in a multi-line entry.

  \hyperref[fig:tocbasic.largetocline]%
  {Figure~\ref*{fig:tocbasic.largetocline}} illustrates the characteristics of
  this style. You will also notice that the style has adopted some
  inconsistencies present in the standard classes, e.\,g. the missing indent
  of the second and following lines of an entry or the different values of
  \Macro{@pnumwidth} that results from the font-size dependency. This can
  result, in extreme cases, in the entry text coming too close. Note that the
  width of the entry number shown in the figure is only valid if
  \DescRef{\LabelBase.cmd.numberline} has been used. The standard classes,
  however, use a distance of 1\Unit{em} after the number.
  \begin{figure}
    \centering
    \resizebox{.8\linewidth}{!}{%
      \begin{tikzpicture}
        \draw[color=lightgray] (0,2\baselineskip) -- (0,-2.5\baselineskip);
        \draw[color=lightgray] (\linewidth,2\baselineskip) --
                               (\linewidth,-2.5\baselineskip);
        \node (largetocline) at (0,0) [anchor=west,inner sep=0,outer sep=0] {%
          \parbox[t]{\dimexpr \linewidth-1.55em\relax}{%
            \makebox[3em][l]{\large\bfseries I}%
            \large\bfseries
            Text of an entry to the table of contents with the
            \PValue{largetocline} style and more than one line%
            \hfill
            \makebox[0pt][l]{\normalsize\normalfont
              \makebox[1.55em][r]{\large\bfseries 1}}%
          }%
        };
        \draw[|-|,color=gray] (0,\baselineskip) -- 
                              node [anchor=south] { 3\,em } 
                              (3em,\baselineskip);
        \draw[|-|,color=gray,overlay] (\linewidth,\ht\strutbox) -- 
                              node [anchor=south] { \Macro{@pnumwidth} }
                              (\linewidth-1.55em,\ht\strutbox);
        \large\bfseries
        \draw[|-|,color=gray,overlay] (\linewidth,-\baselineskip) -- 
                              node [anchor=north,font=\normalfont\normalsize] { 
                                \Macro{large}\Macro{@pnumwidth} 
                              }
                              (\linewidth-1.55em,-\baselineskip);
      \end{tikzpicture}%
    }
    \caption{Illustrations of some attributes of a TOC entry with style 
      \PValue{largetocline}}
    \label{fig:tocbasic.largetocline}
  \end{figure}
\item[\PValue{tocline}] is a flexible style. The \KOMAScript{} classes use
  this style by default for all kinds of entries. Likewise, these classes
  define the clones \PValue{part}, \PValue{chapter}, and \PValue{section}, or
  \PValue{section} and \PValue{subsection} using this style, but add extra
  \PName{initial code} to the clones to change their defaults.

  The style supports 20\important{\PValue{level}, \PValue{beforeskip},
    \PValue{breakafternumber}, \PValue{dynindent}, \PValue{dynnumwidth},
    \PValue{entryformat}, \PValue{entrynumberformat}, \PValue{indent},
    \PValue{indentfollows}, \PValue{linefill}, \PValue{numsep},
    \PValue{numwidth}, \PValue{onstarthigherlevel},
    \PValue{onstartlowerlevel}, \PValue{onstartsamelevel},
    \PValue{pagenumberbox}, \PValue{pagenumberformat},
    \PValue{pagenumberwidth}, \PValue{raggedentrytext},
    \PValue{raggedpagenumber}, \PValue{rightindent}} additional attributes in
  addition to the default \PValue{level} attribute. The defaults of all these
  attributes depend on the name of the \PName{entry level} and correspond to
  the results of the standard classes. So after loading \Package{tocbasic},
  you can change the style of the entries in the table of contents of the
  standard classes into \PValue{tocline} using
  \DescRef{\LabelBase.cmd.DeclareTOCEntryStyle} without this leading directly
  to major changes in their appearance. Thus you can precisely change only
  those attributes that are necessary for the desired changes. The same
  applies to the list of figures and the list of tables for the standard
  classes.

  Because its great flexibility, this style can in principle replace the
  \PValue{dottedtocline}, \PValue{undottedtocline}, and \PValue{largetocline}
  styles, but this requires more effort to configure.

  \hyperref[fig:tocbasic.tocline]%
  {Figure~\ref*{fig:tocbasic.tocline}} illustrates some of the length
  attributes of this style. The others are explained in
  \autoref{tab:tocbasic.tocstyle.attributes} starting on
  \autopageref{tab:tocbasic.tocstyle.attributes}.
  \begin{figure}
    \centering
    \resizebox{.8\linewidth}{!}{%
      \begin{tikzpicture}
        \coordinate (subsection) at (0,0);
        \coordinate (section) at ($(subsection)+(0,2\baselineskip)$);
        \coordinate (chapter) at ($(section)+(0,2\baselineskip)$);
        \coordinate (part)    at ($(chapter)+(0,2.4\baselineskip+1em)$);

        \draw[color=lightgray] 
          ($(part)+(0,2\baselineskip)$) -- 
          (0,-2.5\baselineskip);
        \draw[color=lightgray] 
          ($(part)+(\linewidth,2\baselineskip)$) --
          (\linewidth,-2.5\baselineskip);

        \coordinate (subsection) at (0,0);

        \node at (part) [anchor=west,inner sep=0,outer sep=0]
        {%
          \hspace*{3em}%
          \parbox[t]{\dimexpr\linewidth-5.55em}{%
            \setlength{\parindent}{-3em}%
            \addtolength{\parfillskip}{-2.55em}%
            \makebox[3em][l]{\large\bfseries I.}%
            \textbf{\large Text of a part entry with the
            \PValue{tocline} style and at least two lines of text}%
            \hfill
            \makebox[1.55em][r]{\bfseries 12}\large
          }%
        };
        \draw[|-|,color=gray,overlay] 
          (part) --
          ($(part)+(3em,0)$)
          node [anchor=north east,font=\small] {
            \PValue{numwidth}
          };
        \draw[|-|,color=gray,overlay] 
          ($(part)+(\linewidth,\ht\strutbox)$)
          node [anchor=north,font=\small] { 
            \Macro{@tocrmarg} 
          } --
          ($(part)+(\linewidth-2.55em,\ht\strutbox)$);
        \draw[|-|,color=gray,overlay] 
          ($(part)+(\linewidth,-\baselineskip)$) -- 
          node [anchor=north,font=\small] { 
            \Macro{@pnumwidth} 
          } 
          ($(part)+(\linewidth-1.55em,-\baselineskip)$);
        \node at (chapter) [anchor=west,inner sep=0,outer sep=0]
        {%
          \hspace*{1.5em}%
          \parbox[t]{\dimexpr\linewidth-4.05em}{%
            \setlength{\parindent}{-1.5em}%
            \addtolength{\parfillskip}{-2.55em}%
            \makebox[1.5em][l]{\bfseries 1.}%
            \textbf{Text of a chapter entry with the
            \PValue{tocline} style and more than one line of text
            for demonstration purposes}%
            \hfill
            \makebox[1.55em][r]{\bfseries 12}%
          }%
        };
        \draw[|-|,color=gray,overlay]
          ($(chapter)+(3em,\baselineskip)$) --
          node [anchor=west,font=\small] {
            \PValue{beforeskip}
          }
          ($(part)+(3em,-\baselineskip)$);
        \draw[|-|,color=gray,overlay] 
          (chapter) --
          ($(chapter)+(1.5em,0)$)
          node [anchor=north east,font=\small] {
            \PValue{numwidth}
          };
        \draw[|-|,color=gray,overlay] 
          ($(chapter)+(\linewidth,\ht\strutbox)$)
          node [anchor=north,font=\small] { 
            \Macro{@tocrmarg} 
          } --
          ($(chapter)+(\linewidth-2.55em,\ht\strutbox)$);
        \draw[|-|,color=gray,overlay] 
          ($(chapter)+(\linewidth,-\baselineskip)$)
          node [anchor=north,font=\small] { 
            \Macro{@pnumwidth} 
          } --
          ($(chapter)+(\linewidth-1.55em,-\baselineskip)$);
        \node at (section) [anchor=west,inner sep=0,outer sep=0]
        {
          \hspace*{3.8em}%
          \parbox[t]{\dimexpr\linewidth-6.35em}{%
            \setlength{\parindent}{-2.3em}%
            \addtolength{\parfillskip}{-2.55em}%
            \makebox[2.3em][l]{1.1.}%
            Text of a section entry with the \PValue{tocline}
            style and more than one line of text for
            demonstration purposes%
            \leaders\hbox{$\csname m@th\endcsname
              \mkern 4.5mu\hbox{.}\mkern 4.5mu$}\hfill\nobreak
            \makebox[1.55em][r]{3}%
          }%
        };
        \node at (subsection) [anchor=west,inner sep=0,outer sep=0]
        {%
          \hspace*{7em}%
          \parbox[t]{\dimexpr\linewidth-9.55em}{%
            \setlength{\parindent}{-3.2em}%
            \addtolength{\parfillskip}{-2.55em}%
            \makebox[3.2em][l]{1.1.10.}%
            Text of a subsection entry with the \PValue{tocline}
            and more than one line of text for demonstration
            purposes%
            \leaders\hbox{$\csname m@th\endcsname
              \mkern 4.5mu\hbox{.}\mkern 4.5mu$}\hfill\nobreak
            \makebox[1.55em][r]{12}%
          }%
        };
        \draw[|-|,color=gray,overlay] 
          ($(subsection)+(0,\ht\strutbox)$) -- 
          node [anchor=north,font=\small] {
            \PValue{indent}
          }
          ($(subsection)+(3.8em,\ht\strutbox)$);
        \draw[|-|,color=gray,overlay] 
          ($(subsection)+(3.8em,0)$) --
          ($(subsection)+(7em,0)$)
          node [anchor=north east,font=\small] {
            \PValue{numwidth}
          };
        \draw[|-|,color=gray,overlay] 
          ($(subsection)+(\linewidth,\ht\strutbox)$)
          node [anchor=north,font=\small] { 
            \Macro{@tocrmarg} 
          } --
          ($(subsection)+(\linewidth-2.55em,\ht\strutbox)$);
        \draw[|-|,color=gray,overlay] 
          ($(subsection)+(\linewidth,-\baselineskip)$) -- 
          node [anchor=north,font=\small] { 
            \Macro{@pnumwidth} 
          } 
          ($(subsection)+(\linewidth-1.55em,-\baselineskip)$);
      \end{tikzpicture}%
    }
    \caption{Illustrations of some attributes of a TOC entry with the
      \PValue{tocline} style}
    \label{fig:tocbasic.tocline}
  \end{figure}
\item[\PValue{toctext}]\ChangedAt{v3.27}{\Package{tocbasic}}%
  is a special feature. While all other styles produce one paragraph per
  entry, this one produces one paragraph for all successive entries of this
  style. With 14\important{\PValue{afterpar}, \PValue{beforeskip},
    \PValue{entryformat},
    \PValue{entrynumberformat}, \PValue{indent}, \PValue{numsep},
    \PValue{onendentry}, \PValue{onendlastentry}, \PValue{onstartentry},
    \PValue{onstartfirstentry}, \PValue{pagenumberformat},
    \PValue{prepagenumber}, \PValue{raggedright}, \PValue{rightindent}}
  attributes in addition to the default \PValue{level} attribute, there are
  almost as many options as for \PValue{tocline}. However, this style
  depends on the fact that an unfinished paragraph will be concluded at the
  beginning of the entry for all other styles, as well as at the end of the
  current content list. So it should never be combined with entries or content
  lists that are generated outside of \Package{tocbasic}.
\item[\PValue{undottedtocline}] is similar to the style used by the standard
  \Class{book} and \Class{report} classes for the \PValue{chapter} entry
  level, or by \Class{article} for the \PValue{section} entry level in the
  table of contents. It supports\important{\PValue{level}, \PValue{indent},
  \PValue{numwidth}} only three attributes. A penalty is inserted permitting
  an appropriate page break before the entry, as is a vertical skip. The
  entries are printed with an indentation of \PValue{indent} from the left and
  in \Macro{bfseries}. This is a departure from the standard classes, which do
  not support the indentation of these entry levels.
  \DescRef{\LabelBase.cmd.numberline} is used unchanged. The width of the
  entry number is determined by \PValue{numwidth}. For multi-line entries the
  indent will be increased by the value of \PValue{numwidth} for the second
  and following lines. \hyperref[fig:tocbasic.undottedtocline]%
  {Figure~\ref*{fig:tocbasic.undottedtocline}} illustrates the attributes of
  this style.
  \begin{figure}
    \centering
    \resizebox{.8\linewidth}{!}{%
      \begin{tikzpicture}
        \draw[color=lightgray] (0,2\baselineskip) -- (0,-2.5\baselineskip);
        \draw[color=lightgray] (\linewidth,2\baselineskip) --
                               (\linewidth,-2.5\baselineskip);
        \node (undottedtocline) at (0,0) [anchor=west,inner sep=0,outer sep=0]
        {%
          \makebox[1.5em][l]{\bfseries 1}%
          \parbox[t]{\dimexpr \linewidth-4.05em\relax}{%
            \bfseries
            Text of an entry to the table of contents with the
            \PValue{undottedtocline} style and more than one line%
          }%
          \raisebox{-\baselineskip}{\makebox[2.55em][r]{\bfseries 3}}%
        };
        \draw[|-|,color=gray,overlay] (0,\baselineskip) -- 
                              node [anchor=south,font=\small] {
                                \PValue{numwidth}
                              }
                              (1.5em,\baselineskip);
        \draw[|-|,color=gray,overlay] (\linewidth,\ht\strutbox) -- 
                              node [anchor=south,font=\small] { 
                                \Macro{@tocrmarg} 
                              }
                              (\linewidth-2.55em,\ht\strutbox);
        \draw[|-|,color=gray,overlay] (\linewidth,-\baselineskip) -- 
                              node [anchor=north,font=\small] { 
                                \Macro{@pnumwidth} 
                              } 
                              (\linewidth-1.55em,-\baselineskip);
      \end{tikzpicture}%
    }
    \caption{Illustration of some attributes of the \PValue{undottedtocline}
      style with the example of a chapter title}%
    \label{fig:tocbasic.undottedtocline}
  \end{figure}
\end{description}
You can find an explanation of the attributes of all styles that
\Package{tocbasic} defines in \autoref{tab:tocbasic.tocstyle.attributes}.
In\ChangedAt{v3.27}{\Package{tocbasic}} addition to the usual assignment with
\Option{\PName{key}=\PName{value}}, both commands understand an assignment in 
the form \Option{\PName{key}:=\PName{entry level}} for all options of the
\KOMAScript{} styles. In this case, the current setting of \PName{key} for the
\PName{entry level} will be copied. For example, you can copy the current
indent of the \PValue{figure} level using \OptionValue{indent:}{figure}. For
options that expect a length or integer value, you can also use
\Option{\PName{key}+=\PName{value}} to add \PName{value} to the current setting
of the \PName{key}. To subtract simply, you can use a negative \PName{value}.
For example, with \OptionValue{indent+}{1cm} you can increase the indent by
1\Unit{cm}. For options that expect a list value, you can use
\Option{\PName{key}+=\PName{value}} to append \PName{value} to the current
setting of the \PName{key}.

If\ChangedAt{v3.21}{\Package{tocbasic}} you use these attributes as
options to \DescRef{\LabelBase.cmd.DeclareNewTOC} (see
\DescPageRef{\LabelBase.cmd.DeclareNewTOC}), you must prefix the names of the
attributes with \PValue{tocentry}, e\,g., \PValue{level} becomes
\Option{tocentrylevel}. The copy operation with \Option{:=} described above is
also available here. However, addition with \Option{+=} is not currently
supported.

If\ChangedAt{v3.20}{\Package{tocbasic}} you use
these attributes as options for
\DescRef{maincls-experts.cmd.DeclareSectionCommand}%
\IndexCmd{DeclareSectionCommand}\IndexCmd{DeclareNewSectionCommand}%
\IndexCmd{RedeclareSectionCommand}\IndexCmd{ProvideSectionCommand} (see
\DescPageRef{maincls-experts.cmd.DeclareSectionCommand}) and similar commands,
you must prefix the names of the attributes with \PValue{toc}, e\,g.
\PValue{level} becomes \Option{toclevel}.
At this time, neither the copy operation with \Option{:=} nor the
addition operation with \Option{+=} is supported.

Finally, using \Macro{DeclareTOCStyleEntry} will define the internal command
\Macro{l@\PName{entry level}}.

\begin{desclist}
  \desccaption{%
    Attributes of the predefined TOC-entry styles of \Package{tocbasic}%
    \label{tab:tocbasic.tocstyle.attributes}%
  }{%
    Attributes of the TOC-entry styles (\emph{continued})%
  }%
  \entry{\OptionVName{afterpar}{code}}{%
    \ChangedAt{v3.27}{\Package{tocbasic}}%
    The specified \PName{code} will be executed after the end of the paragraph
    in which an entry with the \PValue{toctext} style is printed. If several
    entries have such settings, their code will be executed in the order of
    the entries.%
  }%
  \entry{\OptionVName{beforeskip}{length}}{%
    The  vertical distance inserted before an entry of this level in the
    \PValue{tocline} style (see \autoref{fig:tocbasic.tocline}). The distance
    is made using either \Macro{vskip} or \Macro{addvspace} depending on the
    \PName{entry level}, to maintain compatibility as far as possible with the
    standard classes and earlier versions of \KOMAScript.

    For the \PValue{part} \PName{entry level}, the attribute will be
    initialised with \texttt{2.25em plus 1pt}; for \PValue{chapter}, with
    \texttt{1em plus 1pt}. If the \PName{chapter} \PName{entry level} is
    undefined, \PValue{section} is initialised with \texttt{1em plus 1pt}
    instead. The initial value for all other levels is \texttt{0pt plus
      .2pt}.%

    In\ChangedAt{v3.31}{\Package{tocbasic}} the style \PValue{toctext} the
    vertical space is inserted before the paragraph if it is the first entry
    in the paragraph. It is ignored for all other entries in the paragraph. If
    the initialization takes place via this style, \texttt{0pt} is used as
    default.%
  }%
  \entry{\OptionVName{breakafternumber}{switch}}{%
    \PName{switch} is one of the values for simple switches from
    \autoref{tab:truefalseswitch}, \autopageref{tab:truefalseswitch}. If the
    switch is active in the \PValue{tocline} style, there will be a line break
    after the number set with
    \DescRef{\LabelBase.cmd.numberline}\IndexCmd{numberline}. The line after
    the entry number again starts flush left with the number. The default is
    false for the \PValue{tocline} style.

    If\textnote{Attention!} the \Option{numberline} feature has been activated
    for a content list (see \DescRef{\LabelBase.cmd.setuptoc},
    \autoref{sec:tocbasic.toc}, \DescPageRef{\LabelBase.cmd.setuptoc}), as is
    the case with the \KOMAScript{} classes when the
    \OptionValueRef{maincls}{toc}{numberline}%
    \IndexOption{toc~=\textKValue{numberline}} option is used, then the
    unnumbered entries will nevertheless have a (by default empty) number line
    using the formatting of \Option{entrynumberformat}.%
  }%
  \entry{\OptionVName{dynindent}{switch}}{%
    \ChangedAt{v3.31}{\Package{tocbasic}}%
    \PName{switch} is one of the values for simple switches from
    \autoref{tab:truefalseswitch}, \autopageref{tab:truefalseswitch}. If the
    switch is active with the \PValue{tocline} style, the \PValue{indent}
    attribute only specifies a minimum value. The maximum value is determined
    by the number width and the indentation of the levels specified via
    \PValue{indentfollows}.%
  }%
  \entry{\OptionVName{dynnumwidth}{switch}}{%
    \PName{switch} is one of the values for simple switches from
    \autoref{tab:truefalseswitch}, \autopageref{tab:truefalseswitch}. If the
    switch is active with the \PValue{tocline} style, the \PValue{numwidth}
    attribute specifies a minimum value. If a previous \LaTeX{} run has
    determined that the maximum width of the entry numbers of the same level
    plus the value of \PValue{numsep} is greater than this minimum, the
    calculated value will be used instead.%
  }%
  \entry{\OptionVName{entryformat}{command}}{%
    You can use this attributes to change the format of the entry. The value
    should be a \PName{command} with exactly one argument. This argument is
    not necessarily fully expandable. You should not use commands like
    \Macro{MakeUppercase}, which expects a fully expandable argument. Font
    changes are relative to \Macro{normalfont}\Macro{normalsize}. Note that
    the output of \Option{linefill} and the page number are independent of
    \Option{entryformat}. See also the \Option{pagenumberformat} attribute .

    The initial value of the attribute for the \PValue{part} \PName{entry
    level} is \Macro{large}\Macro{bfseries}, and for \PValue{chapter}, it is
    \Macro{bfseries}. If the \PValue{chapter} level is not defined,
    \PValue{section} uses \Macro{bfseries}. All other levels print the
    argument unchanged.%
  }%
  \entry{\OptionVName{entrynumberformat}{command}}{%
    You can use this attribute to format the entry number within
    \DescRef{\LabelBase.cmd.numberline}. The value should be a \PName{command}
    with exactly one argument. Font changes are relative to the one of
    attribute \Option{entryformat}.

    The initial \PName{command} prints the argument unchanged. This means the
    entry number will be printed as it is.

    If\textnote{Attention!} the \Option{numberline} feature for a content list
    has been activated (see \DescRef{\LabelBase.cmd.setuptoc},
    \autoref{sec:tocbasic.toc}, \DescPageRef{\LabelBase.cmd.setuptoc}), as is
    the case with the \KOMAScript{} classes using the
    \OptionValueRef{maincls}{toc}{numberline}%
    \IndexOption{toc~=\textKValue{numberline}} option, the unnumbered entries
    will execute the \PName{command} as well.%
  }%
  \entry{%
    \OptionVName{indent}{length}%
    {\phantomsection\label{tab:tocbasic.tocstyle.attributes.indent}}%
  }{%
    For\ChangedAt{v3.27}{\Package{tocbasic}} the \PValue{toctext} style, the
    \PName{length} is the horizontal distance of the paragraph from the left
    margin. If different entries within the paragraph have different settings,
    the last one is used. For the remaining styles, the \PName{length} is the
    horizontal distance of the entry from the left margin (see
    \autoref{fig:tocbasic.dottedtocline} and \autoref{fig:tocbasic.tocline}).
    
    For the styles \PValue{tocline} and \PValue{toctext}, all entry levels
    whose names start with ``\texttt{sub}'' are initialised with the
    \PValue{indent}+\PValue{numwidth} of the entry level of the same name
    without this prefix. For the \PValue{dottedtocline},
    \PValue{undottedtocline}, and \PValue{tocline} styles, the initial values
    of levels \PValue{part} down to \PValue{subparagraph} and the levels
    \PValue{figure}, \PValue{table} and
    \ChangedAt{v3.39}{\Package{tocbasic}}\PValue{lstlisting} are compatible
    with the standard classes resp. package
    \Package{listings}\IndexPackage{listings}. All other levels do not have an
    initial value. Therefore you have to set an explicit value for such levels
    when they are defined first time.

    If the \PValue{noindent} attribute is set for a content list via
    \DescRef{\LabelBase.cmd.setuptoc}, the entries of all styles provided by
    \KOMAScript{} enforce the value 0\Unit{pt} to deactivate the indent.%
  }%
  \entry{\OptionVName{level}{integer}}{%
    The numerical value of the \PName{entry level}. Only those entries whose
    numerical value is less than or equal to the
    \DescRef{\LabelBase.counter.tocdepth}%
    \important{\DescRef{\LabelBase.counter.tocdepth}}\IndexCounter{tocdepth}
    counter are printed.

    This attribute is mandatory for all styles and will be defined
    automatically when the style is declared.

    For the \PValue{tocline} and \PValue{toctext} styles, all entry levels
    whose name starts with ``\texttt{sub}'' are initialised with the value of
    the entry level of the same name without this prefix plus one. For the
    \PValue{dottedtocline}, \PValue{largetocline}, \PValue{tocline},
    \PValue{toctext}, and \PValue{undottedtocline} styles, the entry levels
    from \PValue{part} to \PValue{subparagraph}, as well as \PValue{figure},
    \PValue{table} and
    \ChangedAt{v3.39}{\Package{tocbasic}}\PValue{lstlisting}, are initialised
    to be compatible with the standard classes resp. package
    \Package{listings}\IndexPackage{listings}. For all other levels, the
    initialisation is done with the value of \Macro{\PName{entry
        level}numdepth}, if this is defined.%
  }%
  \entry{\OptionVName{indentfollows}{list of levels}}{%
    \ChangedAt{v3.31}{\Package{tocbasic}}%
    If \Option{dynindent} is set with style \PValue{tocline}, the
    comma-separated list of level names specified here is used to determine
    the actual indentation. Levels whose names begin with ``\texttt{sub}''
    will be initialized with the name without this prefix. The \KOMAScript{}
    classes also automatically set appropriate values for the levels
    \PValue{section} and \PValue{paragraph}.%
  }%
  \entry{\OptionVName{linefill}{code}}{%
    With the \PValue{tocline} style, you can change what is used to fill the
    space between the end of the entry text and the page number. The
    \PName{linefill} attribute contains the \PName{code} that prints this
    filler. For the \PValue{part} and \PValue{chapter} \PName{entry level}s,
    the attribute is initialised with \Macro{hfill}. If no \PValue{chapter}
    \PName{entry level} has been defined, \PValue{section} also uses
    \Macro{hfill}. All other entry levels are initialised with
    \DescRef{\LabelBase.cmd.TOCLineLeaderFill} (see
    \DescPageRef{\LabelBase.cmd.TOCLineLeaderFill}).

    Incidentally, if the \PName{code} specified does not automatically fill
    the gap, you should also activate the \PValue{raggedpagenumber} attribute 
    to avoid ``\texttt{underfull \Macro{hbox}}'' messages.%
  }%
  \entry{\OptionVName{numsep}{length}}{%
    The \PValue{tocline} style tries to ensure a minimum distance of
    \PName{length} between the entry number and the entry text. If
    \PValue{dynnumwidth} is active, it will correct the number width to
    achieve this. Otherwise it simply throws a warning if the condition is not
    met.

    The\ChangedAt{v3.27}{\Package{tocbasic}} \PValue{toctext} style, on the
    other hand, always adds a horizontal space of width \PName{length} after
    the number of the entry.
    
    The initial \PName{length} is 0.4\Unit{em}.%
  }%
  \entry{\OptionVName{numwidth}{length}}{%
    The width reserved for the entry number (see
    \autoref{fig:tocbasic.dottedtocline} to
    \autoref{fig:tocbasic.undottedtocline}). With the \PValue{dottedtocline},
    \PValue{tocline}, and \PValue{undottedtocline} styles, this \PName{length}
    is added to the \PName{length} of attribute \PValue{indent} for the second
    and following lines of the entry text.

    With the \PValue{tocline} style, the initial \PName{length} of all entries
    whose name starts with ``\texttt{sub}'' is the value of the level without
    this prefix plus 0.9\Unit{em}, if such a level with corresponding
    attributes exists. With the \PValue{dottedtocline},
    \PValue{undottedtocline}, and \PValue{tocline} styles, the initial
    \PName{length}s of levels from \PValue{part} to \PValue{subparagraph}, as
    well as \PName{figure}, \PName{table} and
    \ChangedAt{v3.39}{\Package{tocbasic}}\PName{lstlisting}, are compatible
    with those of the standard classes resp. package
    \Package{listings}\IndexPackage{listings}. All other levels do not have an
    initial value. Therefore you must set \PValue{numwidth} explicitly when
    the entry level is first used.%
  }%
  \entry{\OptionVName{onendentry}{code}}{%
    \ChangedAt{v3.27}{\Package{tocbasic}}%
    The \PName{code} is executed immediately after an entry with the
    \PValue{toctext} style, if this entry is not the last one of the
    paragraph. The user must ensure that the \PName{code} does not result in
    the paragraph ending.

    Note: In reality the \PName{code} is not executed at the end of the entry
    but before the next entry with style \PValue{toctext}.%
  }%
  \entry{\OptionVName{onendlastentry}{code}}{%
    \ChangedAt{v3.27}{\Package{tocbasic}}%
    The \PName{code} is executed immediately before the end of the paragraph
    with an entry in the \PValue{toctext} style, as long as this entry is the
    last one in the paragraph. The user must ensure that the \PName{code} does
    not result in the paragraph ending.%
  }%
  \entry{\OptionVName{onstartentry}{code}}{%
    \ChangedAt{v3.27}{\Package{tocbasic}}%
    The \PName{code} is executed immediately before an entry with the
    \PValue{toctext} style, unless it is the first one in the paragraph. The
    user must ensure that the \PName{code} does not result in the paragraph
    ending.%
  }%
  \entry{\OptionVName{onstartfirstentry}{code}}{%
    \ChangedAt{v3.27}{\Package{tocbasic}}%
    The \PName{code} is executed immediately before an entry with the
    \PValue{toctext} style if this entry is the first one of the paragraph. The
    user must ensure that the \PName{code} does not result in the paragraph
    ending.%
  }%
  \entry{\OptionVName{onstarthigherlevel}{code}}{%
    The \PValue{tocline} style can execute \PName{code} at the start of an
    entry, depending on whether the previous entry had numerical level greater
    than, the same as, or less than the current entry. The \PName{code}
    specified by this attribute will be executed if the current entry has a
    greater numerical value, i.\,e. it is lower in the entry hierarchy, than
    the previous one.

    Note that detecting the level of the previous entry only works so long as
    \Macro{lastpenalty} has not changed since the previous entry.

    The initial \PName{code} is \DescRef{\LabelBase.cmd.LastTOCLevelWasLower}
    (see \DescPageRef{\LabelBase.cmd.LastTOCLevelWasLower}).%
  }%
  \entry{\OptionVName{onstartlowerlevel}{code}}{%
    The \PValue{tocline} style can execute \PName{code} at the start of an
    entry, depending on whether the previous entry had numerical level greater
    than, the same as, or less than the current entry. The \PName{code}
    specified by this attribute will be executed if the current entry has a
    lower numerical value, i.\,e. it is higher in the entry hierarchy, than
    the previous one.

    Note that detecting the level of the previous entry only works so long as
    \Macro{lastpenalty} has not changed since the previous entry.

    The initial \PName{code} is \DescRef{\LabelBase.cmd.LastTOCLevelWasHigher}
    (see \DescPageRef{\LabelBase.cmd.LastTOCLevelWasHigher}), which usually
    favours a page break before the entry.%
  }%
  \entry{\OptionVName{onstartsamelevel}{code}}{%
    The \PValue{tocline} style can execute \PName{code} at the start of an
    entry, depending on whether the previous entry had numerical level greater
    than, the same as, or less than the current entry. The \PName{code}
    specified by this attribute will be executed if the current entry has the
    same numerical value, i.\,e. it is on the same level in the entry
    hierarchy, as the previous one.

    Note that detecting the level of the previous entry only works so long as
    \Macro{lastpenalty} has not changed since the previous entry.

    The initial \PName{code} is \DescRef{\LabelBase.cmd.LastTOCLevelWasSame}
    (see \DescPageRef{\LabelBase.cmd.LastTOCLevelWasSame}), which usually
    favours a page break before the entry.%
  }%
  \entry{\OptionVName{pagenumberbox}{command}}{%
    By default the page number of an entry is printed flush right in a box
    of width \Macro{@pnumwidth}. In the \PValue{tocline} style, you can
    change the \PName{command} to print the number using this attribute. The
    \PName{command} should expect exactly one argument, the page number.
    
    This attribute is initialised with the box already mentioned.%
  }%
  \entry{\OptionVName{pagenumberformat}{command}}{%
    You can use this attribute to change the format of the page number of an
    entry. The \PName{command} should expect exactly one argument, the page
    number. Font changes are relative to the font of \Option{entryformat}
    followed by \Macro{normalfont}\Macro{normalsize}.

    The initial \PName{command} of entry level \PValue{part} prints the
    argument in \Macro{large}\Macro{bfseries} and of entry level
    \PValue{chapter} in \Macro{bfseries}. For classes without
    \Macro{l@chapter} \PValue{section} also uses \Macro{bfseries}. The initial
    \PName{command} of all other levels prints the argument in
    \Macro{normalfont}\Macro{normalcolor}.%
  }%
  \entry{\OptionVName{pagenumberwidth}{length}}{%
    \ChangedAt{v3.27}{\Package{tocbasic}}%
    You can use this attribute to locally change the width of the default box
    for the page number of an entry with the style \PValue{tocline} from
    \Macro{@pnumwidth} to the specified \PName{length}. Note that if you
    change the default page number box with the \Option{pagenumberbox}
    attribute, the specified \PName{length} will no longer be used
    automatically.%
  }%
  \entry{\OptionVName{prepagenumber}{code}}{%
    \ChangedAt{v3.27}{\Package{tocbasic}}%
    The \PValue{toctext} style executes the \PName{code} between the text and
    the page number of the entry. Usually this is used to add a horizontal
    space or separator between text and page number.%

    The default is a non-breaking space using \Macro{nonbreakspace}.%
  }%
  \entry{\OptionVName{raggedentrytext}{switch}}{%
    The\ChangedAt{v3.21}{\Package{tocbasic}} \PName{switch} is one of the
    values for simple switches from \autoref{tab:truefalseswitch},
    \autopageref{tab:truefalseswitch}. If the switch is active, the
    \PValue{tocline} style prints the text of an entry ragged right instead of
    fully justified, and only words that are longer than a text line are
    automatically hyphenated.

    This \PName{switch} is false by default.%
  }%
 \entry{\OptionVName{raggedpagenumber}{switch}}{%
    The \PName{switch} is one of the values for simple switches from
    \autoref{tab:truefalseswitch}, \autopageref{tab:truefalseswitch}. If the
    switch is active, the \PValue{tocline} style does not force the page
    number to be right justified.

    Depending on the value of \PValue{linefill}, setting this attribute could
    affect only whether a warning message appears, or the formatting of the
    page number as well. So it is important to set both attributes so that
    they correspond.

    By default the \PName{switch} is not activated and therefore corresponds
    with an initial value of \Macro{hfill} or
    \DescRef{\LabelBase.cmd.TOCLineLeaderFill} for the \PValue{linefill}
    attribute.%
  }%
  \entry{\OptionVName{raggedright}{switch}}{%
    \ChangedAt{v3.27}{\Package{tocbasic}}%
    The \PName{switch} is one of the values for simple switches from
    \autoref{tab:truefalseswitch}, \autopageref{tab:truefalseswitch}. If the
    switch is active for any entry with the \PValue{toctext} style inside the
    same paragraph, the whole paragraph is printed ragged right.
  }%
  \entry{\OptionVName{rightindent}{length}}{%
    \ChangedAt{v3.27}{\Package{tocbasic}}%
    You can use this attribute to locally change the right indent for the text
    of an entry with the \PValue{tocline} style from \Macro{@tocrmarg} to the
    specified \PName{length}.%
  }%
\end{desclist}

While \Macro{DeclareTOCStyleEntry} defines only one \PName{entry level},
\Macro{DeclareTOCStyleEntries}\ChangedAt{v3.26}{\Package{tocbasic}} can define
an entire list of \PName{entry level}s in one command. Each entry level in the
comma-separated \PName{entry-level list} is defined with the same \PName{style}
and settings of the given \PName{option list}.%
\EndIndexGroup


\begin{Declaration}
  \Macro{DeclareTOCEntryStyle}\Parameter{style}%
                              \OParameter{initial code}%
                              \Parameter{command code}%
  \Macro{DefineTOCEntryOption}\Parameter{option}\OParameter{default value}%
                              \Parameter{code}%
  \Macro{DefineTOCEntryBooleanOption}\Parameter{option}%
                                     \OParameter{default value}%
                                     \Parameter{prefix}%
                                     \Parameter{postfix}%
                                     \Parameter{description}%%
                                     %\OParameter{initial code}%
  \Macro{DefineTOCEntryCommandOption}\Parameter{option}%
                                     \OParameter{default value}%
                                     \Parameter{prefix}%
                                     \Parameter{postfix}%
                                     \Parameter{description}%%
                                     %\OParameter{initial code}%
  \Macro{DefineTOCEntryIfOption}\Parameter{option}%
                                     \OParameter{default value}%
                                     \Parameter{prefix}%
                                     \Parameter{postfix}%
                                     \Parameter{description}%%
                                     %\OParameter{initial code}%
  \Macro{DefineTOCEntryLengthOption}\Parameter{option}%
                                     \OParameter{default value}%
                                     \Parameter{prefix}%
                                     \Parameter{postfix}%
                                     \Parameter{description}%%
                                     %\OParameter{initial code}%
  \Macro{DefineTOCEntryNumberOption}\Parameter{option}%
                                     \OParameter{default value}%
                                     \Parameter{prefix}%
                                     \Parameter{postfix}%
                                     \Parameter{description}%
                                     %\OParameter{initial code}%
\end{Declaration}
\Macro{DeclareTOCEntryStyle}\ChangedAt{v3.20}{\Package{tocbasic}} is
one of the most complex commands in \KOMAScript. It is therefore explicitly
intended for \LaTeX{} developers and not for ordinary \LaTeX{} users. It lets
you define new a \PName{style} for content-list entries. Usually, entries to
content lists are made using
\Macro{addcontentsline}\IndexCmd{addcontentsline}, or preferably, if you use
\Package{tocbasic}, with
\DescRef{\LabelBase.cmd.addxcontentsline}\IndexCmd{addxcontentsline} (see
\autoref{sec:tocbasic.basics}, \DescPageRef{\LabelBase.cmd.addxcontentsline}).
In both cases \LaTeX{} writes a corresponding
\Macro{contentsline}\IndexCmd{contentsline} to the appropriate auxiliary file.
When reading this auxiliary file, \LaTeX{} then executes a
\Macro{l@\PName{entry level}} command for each \Macro{contentsline}.

If you later assign a \PName{style} to an entry level using
\DescRef{\LabelBase.cmd.DeclareTOCStyleEntry}, the \PName{initial code} is
executed first, if provided, and then the \PName{command code} for the
definition of \Macro{l@\PName{entry level}}. The \PName{command code} is the
code that will be expanded and executed by \Macro{l@\PName{entry level}}.
Inside \PName{command code} \texttt{\#1} is the name of the TOC entry level
and \texttt{\#\#1} and \texttt{\#\#2} are the arguments of
\Macro{l@\PName{entry level}}.

The \PName{initial code} serves first to initialise all attributes of the
\PName{style}. Developers should make sure that all attributes are provided
with values here. Only then does \DescRef{\LabelBase.cmd.DeclareTOCStyleEntry}
work without errors if an \PName{option list} is not specified. The second
task of the \PName{initial code} is to define all the options that this
\PName{style} recognises. The \Option{level} option is always defined
automatically. The value of the \Option{level} can be queried within the
\PName{command code} with \Macro{@nameuse}\PParameter{\#1tocdepth}%
\important{\Macro{\PName{entry level}tocdepth}}, for example, to compare it
with the \DescRef{\LabelBase.counter.tocdepth}\IndexCounter{tocdepth} counter.

To define options for the attributes of the \PName{style} inside the
\PName{initial code}, you can use the commands
\Macro{DefineTOCEntryBooleanOption}, \Macro{DefineTOCEntryCommandOption},
\Macro{DefineTOCEntryIfOption}, \Macro{DefineTOCEntryLengthOption}, and
\Macro{DefineTOCEntryNumberOption}.  These commands each define an
\PName{option} that, when called, defines a macro named
\Macro{\PName{prefix}\PName{entry level}\PName{postfix}} set to the given
value or to the \PName{default value} of the option. The
\Macro{DefineTOCEntryIfOption} command is a somewhat special case. It defines
\Macro{\PName{prefix}\PName{entry level}\PName{postfix}} as a command with two
arguments. If the value passed to the option is one of the activation (true)
values from \autoref{tab:truefalseswitch}, \autopageref{tab:truefalseswitch},
the command expands to the first argument. If the value to the option is a
deactivation (false) value, the command expands to the second argument.

In\ChangedAt{v3.27}{\Package{tocbasic}} addition to the usual options of the
form \Option{\PName{key}=\PName{value}}, the five
\Macro{DefineTOCEntry\dots Option} commands automatically define options of
the form \Option{\PName{key}:=\PName{entry level}}. These copy the value of
another \PName{entry level} if the value is stored in a macro with the same
\PName{prefix} and \PName{postfix}. For the styles predefined by
\Package{tocbasic}, this is the case for all options of the same name
independent of the name of the style. The commands
\Macro{DefineTOCEntryLengthOption} and \Macro{DefineTOCEntryNumberOption} also
define options of the form \Option{\PName{key}:=\PName{value}}, which are used
to add the new \PName{value} to the value already stored in
\Macro{\PName{prefix}\PName{entry level}\PName{postfix}}.

The \PName{description} should be a brief message that describes the sense
of the option with some keywords. The \Package{tocbasic} package uses this text
in error messages, warnings, and information output on the terminal and to the
\File{log} file.

The simplest style of \Package{tocbasic}, \PValue{gobble}, is defined
using:
\begin{lstcode}
  \DeclareTOCEntryStyle{gobble}{}%
\end{lstcode}
If you now define an entry level \PValue{dummy} in this style using:
\begin{lstcode}
  \DeclareTOCStyleEntry[level=1]{gobble}{dummy}
\end{lstcode}
this would correspond, among other things, to:
\begin{lstcode}
  \def\dummytocdepth{1}
  \def\l@dummy#1#2{}
\end{lstcode}

For example, within the \PValue{tocline} style,
\begin{lstcode}
  \DefineTOCEntryCommandOption{linefill}[\TOCLineLeaderFill]%
  {scr@tso@}{@linefill}{filling between text and page number}%
\end{lstcode}
is used to define the \Option{linefill} option. By specifying 
\DescRef{\LabelBase.cmd.TOCLineLeaderFill} as the \PName{default value},
a call such as
\begin{lstcode}
  \DeclareTOCStyleEntry[linefill]{tocline}{part}
\end{lstcode}
would, among other things, create the definition
\begin{lstcode}
  \def\scr@tso@part@linefill{\TOCLineLeaderFill}
\end{lstcode}

If you want to define your own styles, you should first study the definition
of the \PValue{dottedtocline} style. After you understand this definition, you
can find many hints as to how to use the commands effectively in the much more
complex definition of the \PValue{tocline} style.

However, in many cases it will be sufficient to clone an existing style using
\DescRef{\LabelBase.cmd.CloneTOCEntryStyle} and to change the initial code of
the new style using \DescRef{\LabelBase.cmd.TOCEntryStyleInitCode} or
\DescRef{\LabelBase.cmd.TOCEntryStyleStartInitCode}.

\Macro{DefineTOCEntryOption} is merely used to define the other commands and
you should not use it directly. Normally, there is no need for it. It is
mentioned here only for the sake of completeness.%
\EndIndexGroup


\begin{Declaration}
  \Macro{CloneTOCEntryStyle}\Parameter{style}\Parameter{new style}%
\end{Declaration}
With\ChangedAt{v3.20}{\Package{tocbasic}} this command you can clone
an existing \PName{style}. It defines a \PName{new style} with the same
attributes and settings as the existing \PName{style}. The package itself uses
\Macro{CloneTOCEntryStyle} to declare the \PValue{default} style as a clone of
\PValue{dottedtocline}. The \KOMAScript{} classes use the command to declare
the styles \PValue{part}, \PValue{section}, and \PValue{chapter} or
\PValue{subsection} as clones of \PValue{tocline} and then modify them with
\DescRef{\LabelBase.cmd.TOCEntryStyleInitCode} and
\DescRef{\LabelBase.cmd.TOCEntryStyleStartInitCode}. The \Class{scrbook} and
\Class{scrreprt} classes newly declare the \PValue{default} style as a clone
of \PValue{section}, and \Class{scrartcl} declares it as a clone of
\PValue{subsection}.%
\EndIndexGroup


\begin{Declaration}
  \Macro{TOCEntryStyleInitCode}\Parameter{style}%
                               \Parameter{initial code}%
  \Macro{TOCEntryStyleStartInitCode}\Parameter{style}%
                                    \Parameter{initial code}
\end{Declaration}
Every\ChangedAt{v3.20}{\Package{tocbasic}} TOC-entry style has an
initialisation code. This is used whenever a \PName{style} is assigned to an
TOC entry using \DescRef{\LabelBase.cmd.DeclareTOCEntryStyle}. This
\PName{initial code} should not have global side effects, because it is also
used for local initialisation inside other commands like
\DescRef{\LabelBase.cmd.DeclareNewTOC}\IndexCmd{DeclareNewTOC}. The
\PName{initial code} not only defines all attributes of a \PName{style}, but
it also sets the defaults for those attributes.

You can use \Macro{TOCEntryStyleStartInitCode} and
\Macro{TOCEntryStyleInitCode} to extend previously existing initialisation
code with further \PName{initial code}. \Macro{TOCEntryStyleStartInitCode}
adds \PName{initial code} in front of the existing code.
\Macro{TOCEntryStyleInitCode} adds the \PName{initial code} at the end of the
existing initialisation code. The \KOMAScript{} classes, for example, use
\Macro{TOCEntryStyleStartInitCode} to properly initialise the fill, fonts, and
vertical spacing of the \PValue{part} style cloned from \PValue{tocline}. For
example, the \Class{scrbook} and \Class{scrreprt} classes use
\begin{lstcode}
  \CloneTOCEntryStyle{tocline}{section}
  \TOCEntryStyleStartInitCode{section}{%
    \expandafter\providecommand%
    \csname scr@tso@#1@linefill\endcsname
    {\TOCLineLeaderFill\relax}%
  }
\end{lstcode}
to define \PValue{section} as a modified clone of \PValue{tocline}.%
\EndIndexGroup


\begin{Declaration}
  \Macro{LastTOCLevelWasHigher}%
  \Macro{LastTOCLevelWasSame}%
  \Macro{LastTOCLevelWasLower}
\end{Declaration}
At\ChangedAt{v3.20}{\Package{tocbasic}} the beginning of entries
using the \PValue{tocline} style, \Package{tocbasic} executes one of these
three commands depending on \Macro{lastpenalty}. \Macro{LastTOCLevelWasHigher}
and \Macro{LastTOCLevelWasSame} used in vertical mode add
\Macro{addpenalty}\PParameter{\Macro{@lowpenalty}} and therefore permit a page
break before an entry with the same or higher hierarchical position.
\Macro{LastTOCLevelWasLower} is empty, so a page break between an entry and
its first sub-entry is not permitted.

Users should not redefine these commands. Instead, you should change the
behaviour of single entry levels using the \PValue{onstartlowerlevel},
\PValue{onstartsamelevel}, and \PValue{onstarthigherlevel} attributes.%
\EndIndexGroup


\begin{Declaration}
  \Macro{TOCLineLeaderFill}\OParameter{leader}
\end{Declaration}
This\ChangedAt{v3.20}{\Package{tocbasic}} command is intended to be
used as a value for the \Option{linefill} option of the \PName{tocline}
TOC-entry style. It creates a connection between the end of the entry text and
the entry's page number. You can specify the \PName{leader}, which is repeated
at regular intervals, as an optional argument. The default is a dot.

As the name suggests, the command uses \Macro{leaders} to output the
\PName{leader}. The spacing used is defined analogously to the \LaTeX{} kernel
command \Macro{@dottedtocline} by
\Macro{mkern}\Macro{@dotsep}\Unit{\texttt{mu}}.%
\EndIndexGroup
\EndIndexGroup


\section{Internal Commands for Class and Package Authors}
\seclabel{internals}

The \Package{tocbasic} package provides some internal commands for the use of
class and package authors. These commands all begin with the prefix
\Macro{tocbasic@}. But\textnote{Attention!} even class or package authors
should not redefine them! Their inner functioning may be changed or extended
at any time, so redefining these commands could significantly damage the
\Package{tocbasic}'s operation.

\begin{Declaration}
  \Macro{tocbasic@extend@babel}\Parameter{extension}
\end{Declaration}
At every change of the current language, either at the beginning of the
document or inside the document, the \Package{babel}\IndexPackage{babel}
package (see \cite{package:babel}), or rather a \LaTeX{} kernel enhanced by
\Package{babel}'s language management, writes language-switching commands to
the files with the \File{toc}, \File{lof}, and \File{lot} extensions. The
\Package{tocbasic} package extends this mechanism with
\Macro{tocbasic@extend@babel} so that it also works for other file extensions.
The \PName{extension} argument must be completely expanded! Otherwise the
there is a risk that, for example, the meaning of the argument has already
change at the time it is actually evaluated.

This command is typically invoked by default for every file \PName{extension}
added to the list of known extensions with
\DescRef{\LabelBase.cmd.addtotoclist}. You can suppress this with the
\PValue{nobabel}\important{\PValue{nobabel}} feature (see
\DescRef{\LabelBase.cmd.setuptoc}, \autoref{sec:tocbasic.toc},
\DescPageRef{\LabelBase.cmd.setuptoc}). \Package{tocbasic} does this
automatically for the extensions \File{toc}, \File{lof}, and \File{lot} to
avoid switching languages twice in the corresponding files.

There is usually no reason to call this command yourself. However, there could
conceivably be content lists that are not under the control of
\Package{tocbasic} and so are not in \Package{tocbasic}'s list of known file
extensions, but which nevertheless should use \Package{babel}'s language
switching mechanism. You can call the command explicitly for those files.
But\textnote{Attention!} please note that this should be done only once per
file extension!%
\EndIndexGroup


\begin{Declaration}
  \Macro{tocbasic@starttoc}\Parameter{extension}
\end{Declaration}
This command is the actual replacement for the
\Macro{@starttoc}\IndexCmd{@starttoc}\important{\Macro{@starttoc}} command
from the \LaTeX{} kernel. It is the command behind
\DescRef{\LabelBase.cmd.listoftoc*} (see \autoref{sec:tocbasic.toc},
\DescPageRef{\LabelBase.cmd.listoftoc*}). Class or package authors who want to
take advantage of \Package{tocbasic} should at least use this command, or even
better, \DescRef{\LabelBase.cmd.listoftoc}. The command uses
\Macro{\@starttoc} internally, but sets
\Length{parskip}\IndexLength{parskip}\important{\Length{parskip}\\
\Length{parindent}\\ \Length{parfillskip}},
\Length{parindent}\IndexLength{parindent} to 0, and \Length{parfillskip} to 0
to infinity. Moreover,
\Macro{@currext}\important{\Macro{@currext}}\IndexCmd{@currext} is set to the
file extension of the current TOC file, so it can be evaluated during the
subsequent execution of the hooks. You can find an explanation of these hooks
below.

Because\textnote{Attention!} \LaTeX{} opens a new content-list file for
writing after reading that file, calling this command may result in an error
message of the type
\begin{lstoutput}
  ! No room for a new \write .
  \ch@ck ...\else \errmessage {No room for a new #3}
                                                    \fi
\end{lstoutput}
if no more write handles are available. You can solve this problem by loading
the \Package{scrwfile}\important{\Package{scrwfile}}\IndexPackage{scrwfile}
package \cite{package:scrwfile} or by using \LuaLaTeX{}.%
\EndIndexGroup


\begin{Declaration}
  \Macro{tocbasic@@before@hook}%
  \Macro{tocbasic@@after@hook}
\end{Declaration}
The \Macro{tocbasic@@before@hook} hook is executed immediately before reading
an auxiliary file for a content list, before executing the commands defined
with \DescRef{\LabelBase.cmd.BeforeStartingTOC} command. You can extend this
hook using \Macro{g@addto@macro}\IndexCmd{g@addto@macro}.

Similarly, \Macro{tocbasic@@after@hook} is executed immediately after reading
such an auxiliary file and before executing the commands defined with
\DescRef{\LabelBase.cmd.AfterStartingTOC}. You can extend this hook using
\Macro{g@addto@macro}\IndexCmd{g@addto@macro}.

\KOMAScript{} uses these hooks to dynamically adjust content lists to the
width of the heading numbers. Only classes and packages should use these
hooks. Users\textnote{Attention!} should really use
\DescRef{\LabelBase.cmd.BeforeStartingTOC} and
\DescRef{\LabelBase.cmd.AfterStartingTOC} instead. Authors of packages should
also prefer these commands. These hooks should not be used to generate any
output!

If neither\textnote{Attention!} \DescRef{\LabelBase.cmd.listofeachtoc} nor
\DescRef{\LabelBase.cmd.listoftoc} nor \DescRef{\LabelBase.cmd.listoftoc*} are
used to output the content list, the hooks should be executed explicitly.%
\EndIndexGroup


\begin{Declaration}
  \Macro{tocbasic@\PName{extension}@before@hook}%
  \Macro{tocbasic@\PName{extension}@after@hook}
\end{Declaration}
These hooks are executed directly after
\DescRef{\LabelBase.cmd.tocbasic@@before@hook} or before
\DescRef{\LabelBase.cmd.tocbasic@@after@hook} for the TOC file with the
corresponding file \PName{extension}. Class\textnote{Attention!} and package
authors should never change them under any circumstances! If
neither\textnote{Attention!} \DescRef{\LabelBase.cmd.listofeachtoc} nor
\DescRef{\LabelBase.cmd.listoftoc} nor \DescRef{\LabelBase.cmd.listoftoc*} are
used to output a content list, the hooks should nevertheless be called, if
they are defined. These commands can be undefined.%
\iffalse % With current LaTeX you can simply use \@ifundefined
For an appropriate test, see \DescRef{scrbase.cmd.Ifundefinedorrelax}%
\IndexCmd{Ifundefinedorrelax} in \autoref{sec:scrbase.if},
\DescPageRef{scrbase.cmd.Ifundefinedorrelax}.%
\fi%
\EndIndexGroup


\begin{Declaration}
  \Macro{tocbasic@listhead}\Parameter{title}
\end{Declaration}
This command is used by \DescRef{\LabelBase.cmd.listoftoc} and
\DescRef{\LabelBase.cmd.listofeachtoc} to set the heading of the content list.
This can be either the default heading of the \Package{tocbasic} package or a
custom definition. If you define your own command for the heading, you can
also use \Macro{tocbasic@listhead}. In this case, you should define
\Macro{@currext}\important{\Macro{@currext}}\IndexCmd{@currext} to be the file
extension of the corresponding TOC file before using
\Macro{tocbasic@listhead}.%
\EndIndexGroup


\begin{Declaration}
  \Macro{tocbasic@listhead@\PName{extension}}\Parameter{title}
\end{Declaration}
This command is used in \DescRef{\LabelBase.cmd.tocbasic@listhead} to set the
individual headings, optional table of contents entry, and running head, if it
is defined. Otherwise, \DescRef{\LabelBase.cmd.tocbasic@listhead} defines them
before their use.%
\EndIndexGroup


\begin{Declaration}
  \Macro{tocbasic@addxcontentsline}%
  \Parameter{extension}\Parameter{level}\Parameter{number}\Parameter{text}%
  \Macro{nonumberline}
\end{Declaration}
The\ChangedAt{v3.12}{\Package{tocbasic}} \Macro{tocbasic@addxcontentsline}
command creates entry of the specified level in the TOC file with the given
\PName{extension}. Whether the entry is numbered or not depends on whether or
not the \PName{number} argument is empty. In this case the \PName{text} will
be prefixed by \Macro{nonumberline} without any argument. Otherwise,
\DescRef{\LabelBase.cmd.numberline} with the \PName{number} argument will used
as usual.

The \Macro{nonumberline} command is redefined inside
\DescRef{\LabelBase.cmd.listoftoc} (see \autoref{sec:tocbasic.toc},
\DescPageRef{\LabelBase.cmd.listoftoc}) depending on the \PValue{numberline}
feature (see \autoref{sec:tocbasic.toc},
\DescPageRef{\LabelBase.cmd.setuptoc}). As a result, changing this feature
results in changes of the corresponding TOC immediately at the next \LaTeX{}
run.%
\EndIndexGroup


\begin{Declaration}
  \Macro{tocbasic@DependOnPenaltyAndTOCLevel}\Parameter{entry level}%
  \Macro{tocbasic@SetPenaltyByTOCLevel}\Parameter{entry level}
\end{Declaration}
The\ChangedAt{v3.20}{\Package{tocbasic}} \PValue{tocline}
content-list style (see \autoref{sec:tocbasic.tocstyle}) sets a
\Macro{penalty} at the end of each entry via
\Macro{tocbasic@SetPenaltyByTOCLevel} so that no page break can occur after an
entry. The exact value chosen depends on the \PName{entry level}.

At the beginning of an entry, \Macro{tocbasic@DependOnPenaltyAndTOCLevel} is
used to execute the value of the \Option{onstartlowerlevel}, the
\Option{onstartsamelevel}, or the \Option{onstarthigherlevel} style option,
depending on \Macro{lastpenalty} and the current \PName{entry level}. By
default, the first two permit a page break when executed in vertical mode.

Developers of \PValue{tocline}-compatible styles should copy this behaviour.
To do so, they can fall back on these internal macros.%
\EndIndexGroup


\section{A Complete Example}
\seclabel{example}

This section provides a complete example of how to define your own floating
environment including an associated content list and \KOMAScript{} integration
using \Package{tocbasic}. This example uses internal commands, that is, they
have a ``\texttt{@}'' in their name. This means\textnote{Attention}, that you
must either put the code into a package or class or placed it between
\Macro{makeatletter}\important[i]{\Macro{makeatletter}\\\Macro{makeatother}}%
\IndexCmd{makeatletter} and \Macro{makeatother}\IndexCmd{makeatother}.

First\textnote{environment}, we need a new floating environment.
That's easy with the following:
\begin{lstcode}
  \newenvironment{remarkbox}{%
    \@float{remarkbox}%
  }{%
    \end@float
  }
\end{lstcode}
The new environment is named \Environment{remarkbox}.

Each\textnote{placement} floating environment has a default placement. It
consists of one or more of the well-known placement options: \PValue{b},
 \PValue{h}, \PValue{p} and \PValue{t}.
\begin{lstcode}
  \newcommand*{\fps@remarkbox}{tbp}
\end{lstcode}
The new floating environment should be placed by default only either at
the top of a page, at the bottom of a page, or on a separate page.

Floating\textnote{type} environments also have a numerical floating
type between 1 and 31. Environments with the same active bit at the floating type cannot change
their order. Figures and tables normally use type~1 and 2. So a figure that
comes later in the source code than a table may be output earlier than the
table and vice versa.
\begin{lstcode}
  \newcommand*{\ftype@remarkbox}{4}
\end{lstcode}
The new environment has floating type~4, so it may pass figures and floats and
may be passed by those.

The\textnote{number} captions of floating environment also have numbers.
\begin{lstcode}
  \newcounter{remarkbox}
  \newcommand*{\remarkboxformat}{%
    Remark~\theremarkbox\csname autodot\endcsname}
  \newcommand*{\fnum@remarkbox}{\remarkboxformat}
\end{lstcode}
Here, a new counter is defined first, which is independent of the chapters
or the counters of other structural levels. \LaTeX{} itself also defines
\Macro{theremarkbox} with the default output as an Arabic number.
This is then used to define the formatted output of the
counter. The formatted output is again defined as a floating-point
number for use in the \DescRef{maincls.cmd.caption} command.

Floating\textnote{file name extension} environments have their own content lists
and those need an auxiliary file named \Macro{jobname} and a file 
extension:
\begin{lstcode}
  \newcommand*{\ext@remarkbox}{lor}
\end{lstcode}
As the file extension, we use ``\File{lor}''.

With this, the floating environment works. But the content list of 
is still missing. So that we do not have to implement it ourselves, we
use the \Package{tocbasic} package. This is loaded with
\begin{lstcode}
  \usepackage{tocbasic}
\end{lstcode}
inside of document preambles. Class or package authors would use
\begin{lstcode}
  \RequirePackage{tocbasic}
\end{lstcode}
instead.

Now\textnote{extension} we register the file name extension with the
\Package{tocbasic} package:
\begin{lstcode}
  \addtotoclist[float]{lor}
\end{lstcode}
We use \PValue{float} as the owner so that all options of \KOMAScript{}
classes that relate to lists of floating environments also apply to the new
content list.

Next\textnote{title} we define a title or heading for this content list:
\begin{lstcode}
  \newcommand*{\listoflorname}{List of Remarks}
\end{lstcode}
When working with multiple languages, the normal practice is to define an
English title first and then, for example with the help of the
\Package{scrbase} package, to add titles for all the other languages you want
to support. See \autoref{sec:scrbase.languageSupport}, starting on
\autopageref{sec:scrbase.languageSupport}.

Now\textnote{entry} all we have to do is define what a single entry in the
content list should look like:
\begin{lstcode}
  \newcommand*{\l@remarkbox}{\l@figure}
\end{lstcode}
This specifies that entries in the list of remarks should look exactly like the
entries in the list of figures. This would be the easiest solution. A more
explicit definition would be something like:
\begin{lstcode}
  \DeclareTOCStyleEntry[level=1,indent=1em,numwidth=1.5em]%
                       {tocline}{remarkbox}
\end{lstcode}

You\textnote{chapter entry} also want chapter entries to affect the content
list.
\begin{lstcode}
  \setuptoc{lor}{chapteratlist}
\end{lstcode}
Setting this property allows this when you use a \KOMAScript{} class, and other class
that supports this property. Unfortunately, the standard classes do not.

This\textnote{list of remarks} should be enough. Users can now 
select different kinds of headings using the corresponding options of
the \KOMAScript{} classes or \DescRef{\LabelBase.cmd.setuptoc}, (e.\,g. with
or without an entry in the table of contents, with or without numbering). But
with a simple
\begin{lstcode}
  \newcommand*{\listofremarkboxes}{\listoftoc{lor}}
\end{lstcode}
you can simplify the usage even more.

As you've seen, just five one-line commands, of which only three or four are
really necessary, refer to the content list. Nevertheless, the new list of
remarks already provides the ability to place both numbered and unnumbered
entries into the table of contents.You can use a lower sectioning level for
the headings. Running heads are set for the \KOMAScript{} classes, the
standard classes, and all classes that explicitly support \Package{tocbasic}.
Supporting classes even pay attention to this new list of remarks at each new
\DescRef{maincls.cmd.chapter}. Even changes to the current language made with
\Package{babel} are included in the list of remarks.

Of course\textnote{additional features}, package authors can add more
features. For example, they could explicitly offer options to hide
\DescRef{\LabelBase.cmd.setuptoc} from users. Or they can refer to the
\Package{tocbasic} manual when explaining the appropriate features. The
advantage of this is that users automatically benefit from any future
extensions to \Package{tocbasic}. However, if you do not want to burden the
user with the fact that the file extension \File{lor} is used for the key
terms, then
\begin{lstcode}
  \newcommand*{\setupremarkboxes}{\setuptoc{lor}}
\end{lstcode}
is sufficient to set a list of features passed as an argument to
\Macro{setupremarkboxes} as a list of features for the file extension
\File{lor}.

\section{Everything with Only One Command}
\label{sec:tocbasic.declarenewtoc}

The example in the previous section has shows that \Package{tocbasic} makes it
easy to define your own floating environments with their own content lists.
This section shows how it can be even easier.

\begin{Declaration}
  \Macro{DeclareNewTOC}\OParameter{options}\Parameter{extension}
\end{Declaration}
This command declares\ChangedAt{v3.06}{\Package{tocbasic}} a new content list,
its heading, and the description of the entries controlled by
\Package{tocbasic} all in a single step. Optionally, you can also define
floating and non-floating environments at the same time. Inside of both such
environments, \DescRef{maincls.cmd.caption}%
\important{\DescRef{maincls.cmd.caption}}\IndexCmd{caption} creates entries
for this new content list. You can also use the \KOMAScript{} extensions
\DescRef{maincls.cmd.captionabove}\important[i]{%
  \DescRef{maincls.cmd.captionabove}\\
  \DescRef{maincls.cmd.captionbelow}}, \DescRef{maincls.cmd.captionbelow}, and
\DescRef{maincls.env.captionbeside} (see \autoref{sec:maincls.floats}).

The \PName{extension} argument is the file extension of the TOC file that
represents the content list, as explained in  \autoref{sec:tocbasic.basics}.
This argument is mandatory and must not be empty!

The \PName{options} argument is a comma-separated list, of the same type as,
for example, \DescRef{maincls.cmd.KOMAoptions} (see
\autoref{sec:typearea.options}). However\textnote{Attention!}, those options
cannot be set using \DescRef{maincls.cmd.KOMAoptions}\IndexCmd{KOMAoptions}!
You can find an overview of all available options in
\autoref{tab:tocbasic.DeclareNewTOC-options}.

If\ChangedAt{v3.20}{\Package{tocbasic}} the \Option{tocentrystyle}
option is not used, the \PValue{default} style will be used if required. For
information about this style, see \autoref{sec:tocbasic.tocstyle}. If you do
not want to define a command for entries to the content list, you can use an
empty argument, i.\,e. \OptionValue{tocentrystyle}{} or
\OptionValue{tocentrystyle}{\PParameter{}}.

Depending\ChangedAt{v3.20}{\Package{tocbasic}}%
\ChangedAt{v3.21}{\Package{tocbasic}} on the style of the entries to
the content list, you can set all valid attributes of the selected style as
part of the \PName{options}. To do so, you must add the prefix
\PValue{tocentry} to the names of the attributes given in
\autoref{tab:tocbasic.tocstyle.attributes}, starting on
\autopageref{tab:tocbasic.tocstyle.attributes}. You can make later changes to
the style of the entries at any time using
\DescRef{\LabelBase.cmd.DeclareTOCStyleEntry}%
\IndexCmd{DeclareTOCStyleEntry}%
\important{\DescRef{\LabelBase.cmd.DeclareTOCStyleEntry}}. See
\autoref{sec:tocbasic.tocstyle},
\DescPageRef{\LabelBase.cmd.DeclareTOCStyleEntry} for more information about
the styles.%
%
\begin{desclist}
  \renewcommand*{\abovecaptionskipcorrection}{-\normalbaselineskip}%
  \desccaption[{Options for command \Macro{DeclareNewTOC}}]{%
    Options for the
    \Macro{DeclareNewTOC}\ChangedAt{v3.06}{\Package{tocbasic}} command%
    \label{tab:tocbasic.DeclareNewTOC-options}%
  }{%
    Options for the \Macro{DeclareNewTOC} command (\emph{continued})%
  }%
  \entry{\OptionVName{atbegin}{commands}%
    \ChangedAt{v3.09}{\Package{tocbasic}}}{%
    The \PName{commands} will be executed at the begin of the floating or
    non-floating environment.%
  }%
  \entry{\OptionVName{atend}{commands}%
    \ChangedAt{v3.09}{\Package{tocbasic}}}{%
    The \PName{commands} will be executed at the end of the floating or
    non-floating environment.%
  }%
  \entry{\OptionVName{category}{string}}{%
    \ChangedAt{v3.27}{\Package{tocbasic}}%
    This option can be used as a synonym for \OptionVName{owner}{string}.%
  }%
  \entry{\OptionVName{counterwithin}{\LaTeX{} counter}}{%
    If you define a new floating or non-floating environment, a new counter
    \Counter{\PName{type}} will be created as well (see option
    \Option{type}). You can make this counter dependent on another
    \PName{\LaTeX{} counter} in the same way, for example, that the
    \Counter{figure} counter in the \Class{book} classes is dependent on the
    \Counter{chapter} counter.  A\ChangedAt{v3.35}{\Package{tocbasic}} setting
    \OptionValue{counterwithin}{chapter} is used for classes with
    \DescRef{maincls.cmd.chapter}\IndexCmd{chapter} only in the main matter
    (see \DescRef{maincls.cmd.frontmatter}\IndexCmd{frontmatter},
    \DescRef{maincls.cmd.mainmatter} and
    \DescRef{maincls.cmd.backmatter}\IndexCmd{backmatter} in
    \autoref{sec:maincls.separation}, \DescPageRef{maincls.cmd.frontmatter})
    and only if the counter \Counter{chapter}\IndexCounter{chapter} is greater
    than zero at output. For classes without \DescRef{maincls.cmd.chapter}
    this applies accordingly to the \OptionValue{counterwithin}{section}
    setting and counter
    \Counter{section}\IndexCounter{section}\IndexCmd{section}.%
  }%
  \entry{\Option{float}}{%
    If set, defines a new content list and a floating environment, both named
    \PName{type}, and an environment for double-column floats named
    \PName{type*}.%
  }%
  \entry{\OptionVName{floatpos}{float positions}}{%
    Each floating environment has default \PName{float positions} that can be
    changed through the optional argument of the floating environment. The
    syntax and semantics are identical to those of the standard floating
    environments. If the option is not used, the default \PName{float
      positions} are ``\texttt{tbp}'', that is \emph{top}, \emph{bottom},
    \emph{page}.%
  }%
  \entry{\OptionVName{floattype}{number}}{%
    Each floating environment has a \PName{number}. Floating environments
    where only different bits are set can be moved past each other. The
    floating environments \Environment{figure} and \Environment{table} usually
    have the types 1 and 2, so they can move past each other. The numerical
    float type can be between 1 and 31. If common bits are set, the float
    types cannot be reordred. If no float type is given, the greatest possible
    one-bit type, 16, will be used.%
  }%
  \entry{\Option{forcenames}}{%
    If set, the names will be defined even if they were already defined
    before.%
  }%
  \entry{\OptionVName{hang}{length}}{%
    \ChangedAt{v3.20}{\Package{tocbasic}}%
    \ChangedAt{v3.21}{\Package{tocbasic}}%
    This option has been deprecated since \KOMAScript~3.20. Instead, the
    amount of the hanging indent of entries to the content list\index{content
      list>entry} depends on attributes of the TOC-entry style given by the
    \Option{tocentrystyle} option. The \KOMAScript{} styles provide the
    \PValue{numwidth} attribute. If the style used has such an attribute,
    \Macro{DeclareNewTOC} will initialise it with a default of 1.5\Unit{em}.
    You can easily change the \PName{value} using
    \OptionVName{tocentrynumwidth}{value}. The \KOMAScript{} classes, for
    example, use \OptionValue{tocentrynumwidth}{2.3em}.%
  }%
  \entry{\OptionVName{indent}{length}}{%
    \ChangedAt{v3.20}{\Package{tocbasic}}%
    \ChangedAt{v3.21}{\Package{tocbasic}}%
    This option has been deprecated since \KOMAScript~3.20. Instead, the
    amount that entries to the content list\index{content list>entry} are
    indented depends on attributes of the TOC-entry style given by the
    \Option{tocentrystyle} option. The \KOMAScript{} styles provide the
    \PValue{indent} attribute. If the style used has such an attribute,
    \Macro{DeclareNewTOC} will initialise it with a default of 1\Unit{em}. You
    can easily change the \PName{value} using
    \OptionVName{tocentryindent}{value}. The \KOMAScript{} classes for example
    use \OptionValue{tocentrynumwidth}{1.5em}.%
  }%
  \entry{\OptionVName{level}{number}}{%
    \ChangedAt{v3.20}{\Package{tocbasic}}%
    \ChangedAt{v3.21}{\Package{tocbasic}}%
    This option has been deprecated since \KOMAScript~3.20. Instead, the level
    of the entries to the content list\index{content list>entry} depends on
    attributes of the TOC-entry style given by the \Option{tocentrystyle}
    option. Nevertheless, all styles have the \PValue{level} attrobite, and
    \Macro{DeclareNewTOC} initialises it with a default value of 1. You can
    easily change the \PName{value} using \OptionVName{tocentrylevel}{value}.%
  }%
  \entry{\OptionVName{listname}{title}}{%
    Each content list has a heading, or title, that you can specify with this
    option. If the option is not specified, the title will be ``List of
    \PName{entry type}'' (see the \Option{types} option), with the first
    character of the \PName{entry type} changed to upper case. It also defines
    the \Macro{list\PName{entry type}name} macro with this value, which you
    can change at any time. This macro, however, is only defined if it is not
    already defined or if the \Option{forcenames} option is also set.%
  }%
  \entry{\OptionVName{name}{entry name}}{%
    Both the optional prefix for entries in the content list and the labels in
    floating or non-floating environments (see the \Option{float} and
    \Option{nonfloat} options) require an \PName{entry name} for an entry to
    the content list. If no \PName{entry name} is given, the value of the
    \PValue{type} (see the \Option{type} option) with the first character
    changed to upper case will be used. It also defines a \Macro{\PName{entry
        type}name} macro with this value, which you can change at any
    time. This macro, however, is only defined if it is not already defined or
    if the \Option{forcenames} option is also set.%
  }%
  \entry{\Option{nonfloat}}{%
    If set, defines not only a content list but also a non-floating
    environment, \Environment{\PName{entry type}-} (see the \Option{type}
    option), which can be used similarly to a floating environment, but which
    does not move from the place where it is used.%
  }%
  \entry{\OptionVName{owner}{string}}{%
    Every new content list has an owner in \Package{tocbasic} (see
    \autoref{sec:tocbasic.basics}). You can specify this here. If no owner is
    specified, the owner ``\PValue{float}'' is used. The \KOMAScript{} classes
    use this owner for the list of figures and the list of tables.%
  }%
  \entry{\OptionVName{setup}{list of attributes}}{%
    \ChangedAt{v3.25}{\Package{tocbasic}}%
    The \PName{list of attributes} is set with
    \DescRef{\LabelBase.cmd.setuptoc}. Note that to specify multiple
    attributes in a comma-separated list, you must put this list between
    braces.%
  }%
  \entry{\OptionVName{tocentrystyle}{TOC-entry style}}{%
    \ChangedAt{v3.20}{\Package{tocbasic}}%
    \PName{TOC-entry style} specifies the style that should be used for all
    entries to the content list corresponding to the \PName{extension}. The
    name of the entry level is given by the \Option{type} option. In addition
    to the options in this table, all attributes of the \PName{TOC-entry
      style} can be used as options. To do so, you have to prefix the name of
    such an attribute with \PValue{tocentry}. For example, you can change the
    numerical level of the entries using the \Option{tocentrylevel} option.
    For more information about the styles and their attributes see
    \autoref{sec:tocbasic.tocstyle}, starting on
    \autopageref{sec:tocbasic.tocstyle}.%
  }%
  \entry{\OptionVName{tocentry\PName{style-option}}{value}}{%
    \ChangedAt{v3.20}{\Package{tocbasic}}%
    Additional options depending on the \PName{TOC-entry style} given by
    \Option{tocentrystyle}. See \autoref{sec:tocbasic.tocstyle},
    \autopageref{sec:tocbasic.tocstyle} for additional information about
    TOC-entry styles. See \autoref{tab:tocbasic.tocstyle.attributes},
    \autopageref{tab:tocbasic.tocstyle.attributes} for information about the
    attributes of the predefined TOC-entry styles of package
    \Package{tocbasic} that can be used as \PName{style-option}.%
  }%
  \entry{\OptionVName{type}{entry type}}{%
    Sets the type of the newly declared content list. The \PName{entry type}
    is also used as a base name for various macros and possibly environments
    and counters. It should therefore consist only of letters. If this option
    is not used, the file \PName{extension} from the mandatory argument will
    be used as the \PName{entry type}.%
  }%
  \entry{\OptionVName{types}{string}}{%
    In several places, the plural form of the \PName{entry type} is required.
    If no plural is given, the value of the \PValue{entry type} with an ``s''
    appended will be used.%
  }%
  \entry{\OptionVName{unset}{list of attributes}}{%
    \ChangedAt{v3.25}{\Package{tocbasic}}%
    The \PName{list of attributes} is unset with
    \DescRef{\LabelBase.cmd.unsettoc}. Note that to specify a comma-separated
    list of attributes, you must put this list between braces.%
  }%
\end{desclist}

\begin{Example}
  Using \Macro{DeclareNewTOC} significantly shortens the example from
  \autoref{sec:tocbasic.example}:
\begin{lstcode}
  \DeclareNewTOC[%
    type=remarkbox,%
    types=remarkboxes,%
    float,% define a floating environment
    floattype=4,%
    name=Remark,%
    listname={List of Remarks}%
  ]{lor}
  \setuptoc{lor}{chapteratlist}
\end{lstcode}
  In addition to the \Environment{remarkbox} and \Environment{remarkbox*} environments,
  this also defines the \Counter{remarkbox} counter; the commands \Macro{theremarkbox},
  \Macro{remarkboxname}, and \Macro{remarkboxformat} that are used for
  captions; the commands \Macro{listremarkboxnames} and
  \Macro{listofremarkboxes} that are used in the list of remarks; and some
  internal commands that depend on the file name extension \File{lor}.
  If the package should use a default for the floating type, the
  Option{floattype} option can be omitted. If the \Option{nonfloat} option is specified,
  a non-floating environment, \Environment{remarkbox-}, will
  also be defined, inside which you can use \DescRef{maincls.cmd.caption}\IndexCmd{caption}.
  \hyperref[tab:tocbasic.comparison]{Figure~\ref*{tab:tocbasic.comparison}}
  compares the commands, counters, and environments of the
  example \Environment{remarkbox} environment to the commands, counters,
  and environments of figures.%
  \begin{table}
    \centering
    \caption{Comparing the example \Environment{remarkbox} environment
      with the \Environment{figure} environment}
    \label{tab:tocbasic.comparison}
    \begin{tabularx}{\textwidth}{ll>{\raggedright}p{6em}X}
      \toprule
      \Environment{remarkbox} & \Environment{figure}
      & options of \Macro{DeclareNewTOC} & short description \\[1ex]
      \midrule
      \Environment{remarkbox} & \Environment{figure}
      & \Option{type}, \Option{float}
      & floating environments of the respective types\\[1ex]
      \Environment{remarkbox*} & \Environment{figure*}
      & \Option{type}, \Option{float}
      & columns spanning floating environments of the respective types\\[1ex]
      \Counter{remarkbox} & \Counter{figure}
      & \Option{type}, \Option{float}
      & counter used by \DescRef{maincls.cmd.caption}\\[1ex]
      \Macro{theremarkbox} & \Macro{thefigure}
      & \Option{type}, \Option{float}
      & output command to the respective counters\\[1ex]
      \Macro{remarkboxformat} & \DescRef{maincls.cmd.figureformat}
      & \Option{type}, \Option{float}
      & formatting command to the respective counters used by
        \DescRef{maincls.cmd.caption}\\[1ex]
      \Macro{remarkboxname} & \Macro{figurename}
      & \Option{type}, \Option{float}, \Option{name}
      & names used in the label of \DescRef{maincls.cmd.caption}\\[1ex]
      \Macro{listofremarkboxes} & \DescRef{maincls.cmd.listoffigures}
      & \Option{types}, \Option{float}
      & command to show the list of the respective environments\\[1ex]
      \Macro{listremarboxname} & \Macro{listfigurename}
      & \Option{type}, \Option{float}, \Option{listname}
      & heading text of the respective list \\[1ex]
      \Macro{fps@remarkbox} & \Macro{fps@figure}
      & \Option{type}, \Option{float}, \Option{floattype}
      & numeric float type for order perpetuation\\[1ex]
      \File{lor} & \File{lof}
      &
      & file name extension of the TOC file of the respective list \\
      \bottomrule
    \end{tabularx}
  \end{table}

  And here is a possible use of the example environment:
\begin{lstcode}
  \begin{remarkbox}
    \centering
    The same thing should always be typeset in the same way
    and with the same appearance.
    \caption{First Law of Typography}
    \label{rem:typo1}
  \end{remarkbox}
\end{lstcode}
  A snippet of a sample page with this environment might look like this:
  \begin{center}\footnotesize
    \begin{tabular}
      {|!{\hspace{.1\linewidth}}p{.55\linewidth}!{\hspace{.1\linewidth}}|}
      \\
      \centering
      The same thing should always be typeset in the same way
      and with the same appearance.\\[\abovecaptionskip]
      {%
        \usekomafont{caption}\footnotesize{\usekomafont{captionlabel}%
          Remark 1: }First Law of Typography
      }\\
    \end{tabular}%
  \end{center}%
\end{Example}

Users of old versions of package \Package{hyperref} should always use the
\Option{listname} option.  Otherwise they may get an error message because
\Package{hyperref} usually has a problem with the
\Macro{MakeUppercase}\IndexCmd{MakeUppercase} command that is needed to
convert the first letter of \Option{types} to upper case. Better is of course
to use an up-to-date \Package{hyperref} with an up-to-date \LaTeX.%
\EndIndexGroup

 
\section{Obsolete Befehle}
\seclabel{obsolete}

% TODO: new translation
Prior releases of \Package{tocbasic} provide some commands that has been
renamed, because of a statement of The \LaTeX{} Project Team. Those deprecated
commands should not be used any longer.
% :ODOT

\LoadNonFree{tocbasic}{0}%
% 
\EndIndexGroup
%
\endinput

%%% Local Variables: 
%%% mode: latex
%%% TeX-master: "scrguide-en.tex"
%%% coding: utf-8
%%% ispell-local-dictionary: "en_GB"
%%% eval: (flyspell-mode 1)
%%% End:

%  LocalWords:  Multiline multiline
