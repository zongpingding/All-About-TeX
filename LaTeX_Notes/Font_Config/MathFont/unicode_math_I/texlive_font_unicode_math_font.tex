\documentclass{article}
\usepackage[margin=1.5in]{geometry}
\usepackage{amsmath}
\setlength{\parindent}{0pt}
\usepackage{fontspec}
% ==> 'Source Code Pro' in 'texmf-dist/fonts/opentype/abobe/sourcecodepro/*'
% \setmainfont[Ligatures=TeX]{FiraMono-Regular.otf}
\newfontfamily{\sourceCodePro}{Source Code Pro}


% ==> To Set math font like fontspec, loading 'unicodemath' package
%%% XeTeX find font using 'file name' not 'font name'
\usepackage{unicode-math}
% \setmathfont{FiraMath-Regular.otf}
% \setmathfont{Garamond-Math.otf}
\setmathfont{texgyrepagella-math.otf}
% \setmathfont{STIXTwoMath-Regular.otf}


\begin{document}
\section{Fontsepc}
\subsection{normal font}
Hello world

\subsection{switch font}
Use \verb|\newfontfamily{cmd}{font}| to decalre a new font family, and use \verb|{\cmd text}| to switch font.
{\sourceCodePro Hello world}




\section{Unicodemath}
\subsection{Math font selection}
\begin{align}
  & \int x\; \mathrm{d}x = \frac{x^2}{2} + \mathrm{C}\\
  & \sum i = \frac{i(i+1)}{2}\\
  & \mathbb{R} = \mathcal{R} + \mathbf{R}
\end{align}


\subsection{unicode input}
Input can be unicode instead of \LaTeX{} command:
\[
    𝐉 = ∇×𝐇 \qquad 𝐁 = μ₀(𝐌 + 𝐇)
\]

\[
  ∫₀³ xⁿφ₁₂(x)\,ⅆx
\]

\subsection{unimath-erewhon}
There are some predefined symbols in manual of \textbf{unimath-erewhon}, ecah 
command behaves slightly different in different math font:

\begin{itemize}
  \item[L:] Latin\ Modern\ Math (latinmodern-math.otf)
  \item[X:] XITS\ Math (XITSMath-Regular.otf)
  \item[S:] STIX\ Math\ Two (STIXTwoMath-Regular.otf)
  \item[P:] TeX\ Gyre\ Pagella\ Math (texgyrepagella-math.otf)
  \item[D:] DejaVu\ Math\ TeX\ Gyre (texgyredejavu-math.otf)
  \item[F:] Fira\ Math (FiraMath-Regular.otf)
  \item[N:] NCM\ Math (NewCMMath-Book.otf)
  \item[H:] GFS Neohellenic Math (GFSNeohellenicMath.otf)
  \item[E:] Erewhon Math (Erewhon-Math.otf)
  \item[C:] XCharter Math (XCharter-Math.otf)
  \item[R:] Concrete Math (Concrete-Math.otf)
\end{itemize}

Loading form, take ``\TeX{} Gyre Pagella'' for an example, see below:
\begin{verbatim}
\documentclass{article}
\usepackage{unicode-math}
\setmathfont{texgyrepagella-math.otf}

\begin{document}
\[
  \mscrG
\]
\end{document}
\end{verbatim}


The output like:
\begin{align}
  % \Big\lbrack
  % \left\{
  % \lblkbrbrak
  \begin{aligned}
    & \mscrG\; \mscrf  \\
    & \mscrK\; \mscrj
  \end{aligned}
  % \rblkbrbrak
  % \Big\rbrack
  % \right\}
\end{align}

\end{document}