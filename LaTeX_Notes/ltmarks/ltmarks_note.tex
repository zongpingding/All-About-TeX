\documentclass{article}
\usepackage[T1]{fontenc}
\usepackage{fancyhdr}
\pagestyle{fancy}
\def\footrulewidth{0.4pt}
\fancyfoot[C]{\thepage}
\fancyfoot[L]{\TopMark{test}}
\fancyhead[L]{\FirstMark{test}}
\fancyhead[R]{\FirstMark{2e-right-nonempty}}


\ExplSyntaxOn
% create a new mark class
% NOTE: only in the preamble
\mark_new_class:n {test}
\ExplSyntaxOff



\begin{document}
\section{WARN}
The commands are only meaningful inside the \textbf{output routine}(such as in a running header declaration), in other places their
result is (while not random) unpredictable due to the way \LaTeX{} cuts text material into
pages.

Using any of the these commands is useful only when your code is \textbf{executed} while 
\LaTeX{} is building a page. At other times they still return values from the \textbf{last page-build},
so their result is not random, but it is essentially meaningless because you do not
know on which page the current point in the text galley ends up.


\vskip\fill
\noindent\textbf{REFERENCE}\\{}
[1]. The LaTeX Companion, 3nd Edition, Page 391.\\{}
[2]. ltmarks-doc -- The new LaTeX marks mechanism.
\newpage



\section{I}
Basic marks:
\begin{itemize}
  \item \verb|2e-left|: \textbf{Main-Mark} -- ``section'' in article (or ``chapter''  in book class)
  \item \verb|2e-right|: \textbf{Sub-Mark} -- ``subsection'' in article (or ``section'' in book class)
  \item \verb|2e-right-nonempty|: if you have a section at the start of a page.
\end{itemize}

\subsection{I-1}
\subsection{I-2}
PAGE I: top mark -- last mark on the previous page. These marks only avaliable 
after shipout(a page has been created).


\ExplSyntaxOn
% insert content to a mark class
% NOTE: It has no effect in places in which you can’t place floats
\mark_insert:nn {test} {Mark-A}
\mark_insert:nn {test} {Mark-B}
\mark_insert:nn {test} {Mark-C}

% use content of a mark class
% 1. optional regions are:
% - page
% - previous-page
% - column
% - previous-column

% 2. when does these marks update ?
% If a page has just been finished then the region page refers to the current page and
% previous-page, as the name indicates, to the page that has been finished previously.
\begin{itemize}
  \item \mark_use_top:nn {page}{test} 
  \item \mark_use_first:nn {page}{test}
  \item \mark_use_last:nn {page}{test}
\end{itemize}


% The~ text~ in~ shipout~ routine~ is:
% \AddToHook{shipout}{
%   \begin{itemize}
%     \item \mark_use_top:nn {page}{test} 
%     \item \mark_use_first:nn {page}{test}
%     \item \mark_use_last:nn {page}{test}
%   \end{itemize}
% }
% \cs_meaning:N \FirstMark
\ExplSyntaxOff


\newpage
\subsection{II-1}
\subsection{II-2}
PAGE II: Insert a new mark ``Mark-Z1'', ``Mark-Z2''
\ExplSyntaxOn
\begin{itemize}
  \item \mark_use_top:nn {page}{test} 
  \item \mark_use_first:nn {page}{test}
  \item \mark_use_last:nn {page}{test}
\end{itemize}

\mark_debug_on:
\mark_insert:nn {test} {Mark-Z1}
\mark_insert:nn {test} {Mark-Z2}
\mark_debug_off:
\ExplSyntaxOff

\newpage
PAGE III
\ExplSyntaxOn
\begin{itemize}
  \item \mark_use_top:nn {page}{test} 
  \item \mark_use_first:nn {page}{test}
  \item \mark_use_last:nn {page}{test}
\end{itemize}
\ExplSyntaxOff

\newpage
PAGE IV
\ExplSyntaxOn
\mark_insert:nn {test} {Mark-A2}
\mark_insert:nn {test} {Mark-B2}
\mark_insert:nn {test} {Mark-C2}

\begin{itemize}
  \item \mark_use_top:nn {page}{test} 
  \item \mark_use_first:nn {page}{test}
  \item \mark_use_last:nn {page}{test}
\end{itemize}
\ExplSyntaxOff


\newpage
PAGE V: The ``first'' and ``last'' marks are those seen first and last in the current region/page, respectively.
\ExplSyntaxOn
\begin{itemize}
  \item \mark_use_top:nn {page}{test} 
  \item \mark_use_first:nn {page}{test}
  \item \mark_use_last:nn {page}{test}
\end{itemize}
\ExplSyntaxOff


\section{An Example}
\begin{verbatim}
\newenvironment{story}[1]
{\InsertMark{story}{#1}%
  \InsertMark{story}%
  {\ldots continued}%
  \InsertMark{storyend}%
  {turn page to continue}}
{\InsertMark{storyend}{}}


\begin{story}{A story}
\section{Lorem}
\lipsum[1][1]
\subsection{Ipsum} 
\lipsum[1][2-7]\newpage
\lipsum[1][2-7]
\end{story}
\end{verbatim}


\end{document}