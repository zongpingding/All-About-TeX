\documentclass[lang=en, layout=oneside]{zlatex}
\usepackage{ztikz}
\usepackage{ragged2e}
\newcommand\marginComment[1]{\marginpar{\RaggedRight{#1}}}
% \usepackage[hshift=0mm,vshift=0mm]{fgruler}



\begin{document}
\mainmatter
\chapter{ztikz}
\section{2D Plot Example}
There are some examples about using \texttt{ztikz} to draw. The operations 
\texttt{`plus', `minus', `multiply', `divide'} are supported,
using like \texttt{'x+y', 'x-y', 'x*y', 'x/y'} is also OK. But for 
\texttt{'power'} please use the form \texttt{'x**y'}. 

Also, the ztikz support most functions you need, like $\sin, \cos, \cdots$,

\marginComment{Indeed, the support operators and functions are 100\% 
compatible with gnuplot,to say, ztikz is base on gnuplot.}



\begin{figure}[!htb]
    \centering
    \begin{tikzpicture}[font=\small]
        % draw grid and axis
        \ShowGrid[step=1, thin, gray]{(-5, -5); (5, 5)}
        \ShowAxis[>=Latex]{(0, -5); (0, 5)}
        \ShowAxis{(-5, 0); (5, 0)}

        % draw functions
        \Plot[-3:3][red]{x+1}
        \ContourPlot[-3:3; -3:3][dashed, orange, thick]{x+1}
        \ContourPlot[-3:3; -3:3][dashed, blue, thick]{y+1}
        \Plot[-1.5*pi:1.5*pi][green]{sin(x)}
        \Plot[-2:2][blue, name path=power]{x**2-0.75}
        \PlotPrecise{1000}
        \Plot[-2:5][purple]{3*sin(1/(x-3))-2}
        
        % param plot
        \ParamPlot[0:pi][purple, name path=ellipse]{sin(t), 2*cos(t)}

        % contour plot
        \ContourPlot[-2:0][red, thick]{x**2/4+y**2-1}
        \ContourPlot[-4:4; -4:4][teal]{sin(x**2+y**2)-exp(-x*y)}

        % show points
        \ShowPoint[radius=2pt, color=orange]{(0.92, 0.79)}[$p_1=(0.92, 0.97)$][right=.5cm]
        % find intersections will cost a lot time, especially the path generate by contour plot
        \ShowIntersecions{power; ellipse}{1}
    \end{tikzpicture}
\end{figure}

you can also plot the data generated by other program, like Mathematica,
\begin{figure}[!htb]
    \centering
    \begin{tikzpicture}
        \draw[red] plot[smooth] file {./data/mma.data};
    \end{tikzpicture}
    \caption{Plot MMA data}
\end{figure}


\begin{remark}
    \upshape
    If you only draw half of the ellipse, then you can only draw only 1 intersections,
    if you change your parameter from $[0, \pi]$ to $[0, 2\pi]$, then 2 intersections
    are support. 
\end{remark}

Then, there are some examples about using \texttt{ztikz} to draw with external python,
in the next section\footnote[1]{Thanks to the \texttt{\textbackslash write18} function of \TeX{}}.


\section{Python matplotlib Example}
There is a matplotlib figure as explaination for you. All the following Examples are from 
matplotlib official site.

\begin{figure}[!htb]
    \centering
    \input{./data/matplotlibExample_1.mpl}
    \caption{Matplotlib Example}
\end{figure}

Then i will show you a complex graph using matplotlib, which from matplotlib official
gallery.

\begin{figure}[!htb]
    \centering
    \input{./data/matplotlibExample_2.mpl}
    \caption{Mandelbrot Set}
\end{figure}

\begin{remark}\upshape
    If your Python interpreter is not configured correctly, thus making 
    your compiling takes too long, consider to use \texttt{matplotlib.use('Agg')}.

    Additionally, please remove the code about \textbf{show} or \textbf{save} figure,
    ztikz will automatically add a save-figure command in the end. But if your source code 
    have some snippets like \texttt{if \_\_name\_\_ == "\_\_main\_\_"} and so on, which makes your 
    last code about saving figure need a indent, then consider format the source code to 
    prevent the indent.   
\end{remark}

Why i make such command ? can we make the picture first in python first, then introduce this 
picture in the \LaTeX{} document ? yes, of course you can. but, i want ``\textbf{all in one}''.

In my opnion, even this function is hard to make, but i'd like to give a try 
\footnote[2]{the truth is that i have taken too much time on writing this package}, 
then i just need to input this package and use it. 

\section{sympy Example}
Finally, i will show your a addtionally function provided by \texttt{ztikz}, which 
is called \texttt{\textbackslash sympy}, it will calculate using the python package
\texttt{sympy}. By the way, similarly to the matplotlib function before, the \texttt{\textbackslash sympy}
allows to have the cache mechanism, which means that you won't recompile the same code the 
second time you compile your document. 

There is a example.
\[
    \int x^8 + \cos(7x) + 6t\,\mathrm{d}x  
    = \sympy{integrate( x**8 + cos(7*x) + 6*t, x )}    
\]

% \ExplSyntaxOn
% \sys_shell_now:n {sed~ -i~ "2s|set~ isosamples~ .*|set~ isosamples~ 200,200|"~ ./ztikz_output/scripts/data_gen.gp}

\end{document}