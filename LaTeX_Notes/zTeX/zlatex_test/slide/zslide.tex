\documentclass[
  % hyper,
  % fancy,
  class=book,
  % classOption={10pt, oneside},
  mathSpec={envStyle=paris},
  % font={config},
  layout={slide, aspect=16|9},
  % toc={redef}
]{zlatex}
% \def\chapter*{}
% \def\chapter{}
\usepackage[bottom]{footmisc}
\usepackage{lipsum, layout}
% \usepackage{csquotes}
% \zslideMetadataSetup{
%   UL = \textcolor{zslideII}{Section \thesection},
%   UR = \textcolor{zslideMain}{Subsection \thesubsection},
% }





\title{z\TeX{} Fancy Slide}
\author{Eureka}
\date{\today}
\begin{document}
% \frontmatter
\maketitle
\thispagestyle{fancy}
\tableofcontents
\newpage

% \mainmatter
\chapter{z\TeX{} Fisrt}
\section{Test}
\subsection{Math Env}
\begin{theorem}[Pythagorean theorem]\label{pythagorean}
  This is a right triangle, where $a$ is the length of one of the legs, $b$ is the length of the other leg, 
  and $c$ is the length of the hypotenuse. The Pythagorean theorem is then:
  \begin{equation}
    a^2 + b^2 = c^2
  \end{equation}
\end{theorem}


\subsection{Comment}
Consectetuer id, vulputate a, magna. Donec vehicula augue eu neque. Pellentesque habitant
morbi tristique senectus et netus et malesuada fames ac turpis egestas. Mauris ut leo. Cras
viverra metus rhoncus sem.


\section{Graphics}
\subsection{include}
\begin{figure}[!htb]
  \centering
  \includegraphics[height=.5\paperheight]{test.png}
\end{figure}
% \newpage

\lipsum[1]
\subsection{export}
Hello world


\section*{Other}
\contentsname

Hello \footnote[1]{the fisrt note}\footnote[2]{the second note}\footnote[3]{the third note}

\chapter*{Other Item}
\section{layout}
\lipsum[2]
Print layout using \verb|\layout| command provided by \texttt{layout} package.

% \layout

\section{Hello}
Hello world \pageref{zslide-last-page}

New Box:\hbox to 6em{Hello\hfill} world

\section{new toc}

\end{document}