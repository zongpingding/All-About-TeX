\documentclass{article}

\begin{document}
\ExplSyntaxOn
% ==> meta key: keys which themselves set other keys.
% ==> meta key declare 
\keys_define:nn { mymodule }
{
  % normal key
  keyname   .code:n  = Some~code~using:~#1\par,
  % meta key II
  mkey-a    .code:n    = #1\par,
  mkey-b    .tl_set:N  = \l_tmpb_tl,  
  MetanameS .meta:n    = { mkey-a = #1, mkey-b = #1 },
  % meta key
  Metaname  .meta:nn = { mymodule / metakey }{#1},
  metakey / mkey-inner .code:n = Some~inner~key~using:~#1\par,
}
% meta key define outside
\keys_define:nn { mymodule / metakey }
{
  mkey-c   .code:n = MetaKey~Some~code~using:~#1\par,
  unknown  .code:n = MetaKey~Unknown~key:\l_keys_key_str\par
}


% ==> \keys_set:nn will get or run or get the "def-value" part to stream,
% 1. like 'Some~code~using:~#1\par' for key 'keyname', 
% 2. or 'MetaKey~Some~code~using:~#1\par' for key 'mkey-c' in matekey 'Metaname'
% 3. if no value given, #1 = <empty>
\keys_set:nn { mymodule }{
  % ===> normal key
  keyname = KEY-CONTENT,
  keyname = \textbf{COMMAND VALUE},

  % ===> .meta:nn
  Metaname = { mkey-c = META-CONTENT,  mkey-inner = INNER-VALUE },
  % Metaname = { mkey-c },
  % Metaname = { mkey-ca },
  % Metaname = { mkey-c = \textbf{META-CONTENT} },

  % ===> .meta:n 
  MetanameS = { CONTENTS-FOR-ALL },
  mkey-a = AAA, 
  mkey-b = BBB
}
\tl_use:N \l_tmpb_tl

% ==> difference between '.multichoice' and '.multichoices'
% 1. def: {key .multichoices:nn {c-a, c-b}{CODE-RUN}}; 
%    use: either {key=c-a} or {key=c-b} will get 'CODE-RUN'
% 2. def: {key .multichoice: {c-a}{CODE-RUN-A}, {c-b}{CODE-RUN-B}}; 
%    use: {key=c-a} will get 'CODE-RUN-A', {key=c-b} will get 'CODE-RUN-B'
\ExplSyntaxOff
\end{document}
