\documentclass{article}
\usepackage[T1]{fontenc}
\usepackage{xtemplate}


\begin{document}
\ExplSyntaxOn
% 1. create a new object type
\DeclareObjectType{ObjA}{2}

% 2. create a new template for the object type
\DeclareTemplateInterface{ObjA}{TemplateA}{2}
  {
    keyA : function 2 = #1|#2,
  }
\DeclareTemplateCode{ObjA}{TemplateA}{2}
  {
    keyA = \func:nn,
  }{
    \AssignTemplateKeys
    \func:nn {#1}{#2}AB
    % \cs_meaning:N \func:nn
  }

% 3. create an instance of this template
\cs_set:Npn \__mwe_swap:nn #1#2 {#2<-->#1}

% \SetTemplateKeys{ObjA}{TemplateA}{
%   keyA = \__mwe_swap:nn,
% }

\DeclareInstance{ObjA}{InstA}{TemplateA}
  {
    % keyA = WORLD,            % bug
  }

% 4. use this instance
\UseInstance{ObjA}{InstA}
  {ARG-1}{ARG-2}
\ExplSyntaxOff
\end{document}





\documentclass{article}
\usepackage[T1]{fontenc}
\usepackage{xtemplate}


\begin{document}
Ha-Ha

\ExplSyntaxOn
% 1. create a new object type
\DeclareObjectType{ObjA}{0}

% 2. create a new template for the object type
\DeclareTemplateInterface{ObjA}{TemplateA}{0}
  {
    keyA : tokenlist = HELLO,
  }

BEGIN:\cs_meaning:N \l__xtemplate_assignments_tl\par
\DeclareTemplateCode{ObjA}{TemplateA}{0}
  {
    keyA = \l_tmpa_tl,
  }{
    \AssignTemplateKeys
    % \tl_use:N \l_tmpa_tl
    INNER:\cs_meaning:N \l__xtemplate_assignments_tl\par
  }

AFTER:\cs_meaning:N \l__xtemplate_assignments_tl\par

% 3. create an instance of this template
\DeclareInstance{ObjA}{InstA}{TemplateA}
  {
    keyA = WORLD
  }

% 4. use this instance
% \cs_meaning:N \l_tmpb_tl\par
\UseInstance{ObjA}{InstA}\par
\cs_meaning:N \l_tmpa_tl\par
\ExplSyntaxOff
\end{document}





\documentclass{article}
\usepackage[T1]{fontenc}
\usepackage{amsmath}
\usepackage[hmargin=1in, vmargin=1in]{geometry}
\setlength{\parindent}{0pt}
\usepackage{xtemplate}
\usepackage{minted}
\usepackage{tcolorbox}
\tcbuselibrary{listings, minted, breakable, skins}
\tcbset{listing engine=minted}
\definecolor{bg}{RGB}{242, 242, 242}
\DeclareTCBListing{code}{!O{tex}}{
  enhanced,
  breakable,
  colback=bg, 
  frame hidden, 
  left=1mm, right=1mm,
  top=0mm, bottom=0mm, 
  % listing side text,
  minted language=#1, 
  minted style=manni,
  minted options = {  
    autogobble,
    tabsize=2,
    breaklines=true, 
    breakanywhere=true,
    breakanywheresymbolpre=,
    breaksymbolleft=,
    fontsize=\footnotesize
  }
}
% \usepackage{comment}
% \usepackage{hyperref}
% \hypersetup{
%   bookmarksnumbered,
%   colorlinks = true,
%   linkcolor = teal,
%   urlcolor = red,
%   citecolor = blue,
%   pdfauthor = {Eureka},
%   pdftitle = {LaTeX Hook Management Tutorial}
% }


\title{Xtemplate Note}
\author{Eureka}
\date{\today}
\begin{document}
\maketitle
\begin{code}
\ExplSyntaxOn
% 1. create a new object type
\DeclareObjectType{ObjA}{2}

% 2. create a new template for the object type
\DeclareTemplateInterface{ObjA}{TemplateA}{2}
  {
    keyA : tokenlist,
    keyB : integer = 1,
    keyC : function 2 = #1|#2,
    keyD : commalist,
  }
\DeclareTemplateCode{ObjA}{TemplateA}{2}
  {
    keyA = \l_tmpa_tl,
    keyB = \l_tmpa_int,
    keyC = \func:nn,
    keyD = \l_tmpa_clist,
  }{
    \AssignTemplateKeys
    % Args:
    \#1 = #1;~ \#2 = #2\par
    % predefined keys(template):
    % keyA~ Value=\tl_use:N \l_tmpa_tl \par
    % keyB~ Value=\int_use:N \l_tmpa_int\par
    % keyC~ Value=\func:nn {#1}{#2} \par
    keyC~ func=\cs_meaning:N \func:nn\par
    \func:nn AB{#1}{#2}
    % keyD~ Value=\clist_use:Nn \l_tmpa_clist {:}
  }

% 3. create an instance of this template
\cs_set:Npn \__temp_swap:nn #1#2 {#2<-->#1}
\DeclareInstance{ObjA}{InstA}{TemplateA}
  {
    keyA = {Hello, \LaTeX{}!},
    keyB = 100,
    % keyC = \__temp_swap:nn {#1}{#2}, % right
    keyC = \__temp_swap:nn,            % bug
    keyD = {A, B, C},
  }

% 4. use this instance
\UseInstance{ObjA}{InstA}
  {ARG-1}{ARG-2}
\ExplSyntaxOff
\end{code}

\end{document}