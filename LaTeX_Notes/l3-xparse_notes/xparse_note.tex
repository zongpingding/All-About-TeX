\documentclass{article}
\usepackage[T1]{fontenc}
\usepackage{amsmath}
\usepackage[hmargin=1.5in, vmargin=1in]{geometry}
\setlength{\parindent}{0pt}
\usepackage{xparse}
\usepackage{minted}
\usepackage{tcolorbox}
\tcbuselibrary{listings, minted, breakable, skins}
\tcbset{listing engine=minted}
\definecolor{bg}{RGB}{242, 242, 242}
\DeclareTCBListing{code}{!O{tex}}{
  enhanced,
  breakable,
  colback=bg, 
  frame hidden, 
  left=1mm, right=1mm,
  top=0mm, bottom=0mm, 
  listing side text,
  minted language=#1, 
  minted style=manni,
  minted options = {  
    autogobble,
    tabsize=2,
    breaklines=true, 
    breakanywhere=true,
    breakanywheresymbolpre=,
    breaksymbolleft=,
    fontsize=\footnotesize
  }
}
\usepackage{hyperref}
\hypersetup{
  bookmarksnumbered,
  colorlinks = true,
  linkcolor = teal,
  urlcolor = red,
  citecolor = blue,
  pdfauthor = {Eureka},
  pdftitle = {LaTeX Hook Management Tutorial}
}

\begin{document}
{\setlength{\parskip}{1.5em}
\tableofcontents
}
\newpage

\section{argument sepcification}
Three more characters have a special meaning when creating an argument specifier:

\begin{itemize}
\item First, \texttt{+} is used to make an argument \textbf{long} (to accept paragraph tokens-\verb|\par|).
\item Secondly, \texttt{>} is used to declare so-called ``argument processors'', see below.
\item Thirdly, \texttt{!} is used to stress the \textbf{space(s)}, such that 
  \verb|\NewDocumentCommand\foobar{ m!o } {...}| and \verb|\foobar{arg1} [arg2]| will raise error.
\end{itemize}

Recommendation: A very common syntax is to have one optional argument \texttt{o} treated as a key–value list
(using for instance \texttt{l3keys}) followed by some mandatory arguments \texttt{m} (or \texttt{+m}).

\vskip2em
Other argument type: \texttt{l, u} (mandatory), and the corresponding \texttt{g, G}(optional) is not reccomended to use.

\subsection{basic intro}
The mandatory types are: \texttt{m, r, R, v, b}; The types which define optional arguments are: 
\texttt{o, d, D, s, t, e, E}, The first three types are \textbf{corresponding}.

\begin{code}
\NewDocumentCommand{\foobar}{R(){DEFAULT-ARG}}
{
  \#1~ is:~ #1
}
\foobar()\par
\foobar(a)\par
% \foobar
% > raise error and leave: "DEFAULT-ARG" into the stream.
\end{code}

To avoid the ``argument missing error", you should use ``The types which define optional arguments", 
the corresponding types for \texttt{R} is \texttt{D}.

\begin{code}
\NewDocumentCommand{\foobar}{D(){DEFAULT-ARG}}
{
  \#1~ is:~ #1
}
\foobar()\par
\foobar(a)\par
\foobar
\end{code}


\def\TypeT#1{\textbullet\; \texttt{#1}-type:}
\subsection{special case Example}
\TypeT{v}
\begin{code}
\NewDocumentCommand\foo{v}
{
  BEFORE-(#1)-AFTER
}

\foo|\hello #1|\par
\foo{\hello #1}
\end{code}

\TypeT{b}
\begin{code}
% The prefix '+' is used to allow multiple paragraphs in the environment's body.
\NewDocumentEnvironment { twice }
  { O{\ttfamily} +b }
  {#2#1#2} {}

\begin{twice}[\itshape]
  Hello world!
\end{twice}
\end{code}

\TypeT{s}
\begin{code}
% 's' must be the first argument type.
\NewDocumentCommand\foo{sm}
{
  \IfBooleanTF{#1}{TRUE--#2}{FALSE--#2}
}

\foo{ARG}\par
\foo*{ARG}
\end{code}

\TypeT{t}; corresponding to optional type \texttt{s}:
\begin{code}
% 't' must be the first argument type.
% \NewDocumentCommand\foo{t{|}m}{
% or
\NewDocumentCommand\foo{t|m}
{
  \IfBooleanTF{#1}{TRUE--#2}{FALSE--#2}
}

\foo{ARG}\par
\foo|{ARG}
\end{code}

\TypeT{!}; Providing a simple example:
\begin{code}
% amsmath package signature: 
% \DeclareDocumentCommand \\ {!s !o}{...}
% When display breaks are enabled with \allowdisplaybreaks, the \\* command can be used to prohibit a pagebreak after a given line, as usual.

\begin{align}
  a = 1 \\[1em]
  b = 2
\end{align}

\begin{align}
  a = 1 \\*[1em]
  b = 2
\end{align}

\begin{align}
  a = 1 \\* [1em]
  b = 2
\end{align}

\begin{align}
  a = 1 \\ *[1em]
  b = 2
\end{align}
\end{code}

\TypeT{e}
\begin{code}
\NewDocumentCommand\foo{e:e|}
{
  \#1~ is:~ #1\par
  \#2~ is:~ #2\par
}
\NewDocumentCommand\foobar{e{:|}}
{
  \#1~ is:~ #1\par
  \#2~ is:~ #2\par
}

\foo:{ARG-A}|{ARG-B}\vskip2em
\foo|{ARG-B}:{ARG-A}

\dotfill\par
\foobar:{ARG-A}|{ARG-B}\vskip2em
\foobar|{ARG-B}:{ARG-A}
\end{code}

\TypeT{E}
\begin{code}
\NewDocumentCommand\foo{E{:|/}{{ARG-1}{ARG-2}{ARG-3}}}
{
  \#1~ is:~ #1\par
  \#2~ is:~ #2\par
  \#3~ is:~ #3\par
}

\foo:{ARG-A}/{ARG-C}
\end{code}

\section{Argument Processor}
Syntax: \verb|>{<processor>}| in the specification, \verb|>{\ProcessorB} >{\ProcessorA} m| would apply 
\verb|\ProcessorA| followed by \verb|\ProcessorB| to the tokens grabbed by the \texttt{m} argument.

\subsection{SplitArgument Processor}
Split the argument by the separator:
\begin{code}
\ExplSyntaxOn
\NewDocumentCommand\foo{>{\SplitArgument{2}{,}}m}
{ 
  \tl_set:Nn \l_tmpa_tl {#1}
  % \tl_show:N \l_tmpa_tl 
  % > \l_tmpa_tl={a}{b}{c}.
  \argsRead #1
}
\NewDocumentCommand\argsRead{mmm}
{
  \#1-(1)~ is:~ { #1 }\par
  \#1-(2)~ is:~ { #2 }\par
  \#1-(3)~ is:~ { #3 }\par
}
\ExplSyntaxOff

\foo{a, b, c}
\end{code}

Before \verb|#1| passed to command \verb|\argsRead|, it has been splited into 3 parts, and passed to 
\verb|\argsRead| as 3 arguments.(\verb|something like {#1.1}{#1.2}{#1.3} inside of #1|). Thus we can use 
\LaTeX3 based functions \verb|\use_i:nn {}{}, \use_ii:nn {}{}| to get each part.

\begin{code}
\ExplSyntaxOn
\NewDocumentCommand\foo{>{\SplitArgument{1}{,}}m}
{
  \#1-(1)~ is:~ \use_i:nn #1\par
  \#1-(2)~ is:~ \use_ii:nn #1
}
\ExplSyntaxOff

\foo{a, b}
\end{code}

\subsection{SplitList Processor}
\begin{code}
\ExplSyntaxOn
\NewDocumentCommand\foo{>{\SplitList{|}}m}
{ 
  \tl_set:Nn \l_tmpa_tl {#1}
  % \tl_show:N \l_tmpa_tl 
  % > \l_tmpa_tl={a}{b}{c}{d}.
  \use_iii:nnnn #1
}
\ExplSyntaxOff

\foo{a | b | c | d}
\end{code}


\subsection{ProcessList Processor}
\begin{code}
\ExplSyntaxOn
\NewDocumentCommand\foo{ >{\SplitList{;}}m }
{ 
  \ProcessList {#1} { \addBrace } 
}
\NewDocumentCommand\addBrace{m}{
  (#1)\par
}
\ExplSyntaxOff

\foo{a ; b ; c ; d}
\end{code}

\subsection{TrimSpaces Processor}
Use command: \verb|>{\TrimSpaces}m|


\subsection{ProcessedArgument Processor}
This processor provide a way to process the argument using our own function.
\begin{code}
\ExplSyntaxOn
\NewDocumentCommand\foo{ >{\_eq_and_braket:n}m }
{
  \#1~ is:~ #1
}
\cs_set:Npn \_eq_and_braket:n #1 
{
  \tl_set:Nn \ProcessedArgument {[=~#1~=]}  
  % or in this way
  % \def\ProcessedArgument{[=~#1~=]}
}
\ExplSyntaxOff

\foo{a}
\end{code}


\section{Expandable command and environment}
Parsing arguments expandably imposes a number of restrictions on both the type of
arguments that can be read and the error checking available:
\begin{itemize}
  \item The last argument (if any are present) must be one of the mandatory types \texttt{m, r, R, l, u}.
  \item All short arguments appear before long arguments.
  \item The mandatory argument types \texttt{l} and \texttt{u} may not be used after optional arguments.
  \item The optional argument types \texttt{g} and \texttt{G} are not available.
  \item The ``verbatim" argument type \texttt{v} is not available.
  \item Argument processors (using \texttt{>}) are not available.
  \item ..., checking for optional arguments is less robust than in the standard version.
\end{itemize}

\section{Access to the argument specification}
\begin{code}
\ExplSyntaxOn
\NewDocumentCommand\foo{oO{DEFAULT}R()me{:|}m}{}
\GetDocumentCommandArgSpec \foo
\tl_to_str:N \ArgumentSpecification
% \tl_show:N \ArgumentSpecification
% > \ArgumentSpecification=oO{DEFAULT}R()me{:|}m.
\ExplSyntaxOff
\end{code}
\end{document}