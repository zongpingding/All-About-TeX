\documentclass{article}
\usepackage[T1]{fontenc}
\usepackage[margin=1.5in]{geometry}
\setlength{\parindent}{0pt}
\usepackage{xcolor}
\definecolor{lstbg}{HTML}{F6F6F6}
\usepackage{listings}
\lstset{%
  language = TeX,
  basicstyle = \ttfamily,
  morekeywords = {},
  backgroundcolor = \color{lstbg},
  commentstyle =\color{gray}\texttt,
  keywordstyle = {\color{teal}},
  numbers = left, 
  numbersep = 5pt,
  numberstyle = \ttfamily,
}
\renewcommand{\labelitemii}{\(\circ\)}
\usepackage{hyperref}
\hypersetup{
  bookmarksnumbered,
  colorlinks = true,
  linkcolor = teal,
  urlcolor = red,
  citecolor = blue,
  pdfauthor = {Eureka},
  pdftitle = {LaTeX Hook Management Tutorial}
}



\title{\LaTeX{} Table Management\footnote{All implement can be found in ``latex.ltx''}}
\author{Eureka}
\date{\today}
\begin{document}
\maketitle
\tableofcontents

\section{Main commands}
The main commands for making a table of contents are:

\begin{lstlisting}
* \@starttoc{<ext>}

* \addcontentsline{<table>}{<type>}{<entry>} 

* \addtocontents{<table>}{<text>}

* \contentsline{<type>}{<entry>}{<page>}{<hyperref option>}

* \@dottedtocline{<level>}{<indent>}{<numwidth>}{<title>}{<page>}

* \numberline{<number>}
\end{lstlisting}

Firstly, we need to know that what is \textbf{contents(table)} ? Then we could know what the following commands will operator will. 
It is a file with extension \textbf{.toc}, \textbf{.lof}, \textbf{.lot}, etc. which contains the information 
about the table of contents, list of figures, list of tables, etc. respectively.

\vskip2em
When and How does these commands add these contentsline ? If you see the definition of \verb|\@sect|, you will see:
\begin{lstlisting}
\def\@sect#1#2#3#4#5#6[#7]#8{%
  ...
  \ifdim \@tempskipa>\z@
    ...
    \csname #1mark\endcsname{#7}%
    \addcontentsline{toc}{#1}{% The key command lies here !!
      \ifnum #2>\c@secnumdepth \else
        \protect\numberline{\csname the#1\endcsname}%
      \fi
      #7}%
  \else
    \def\@svsechd{%
      ...
      \addcontentsline{toc}{#1}{% The key command lies here !!
        \ifnum #2>\c@secnumdepth \else
          \protect\numberline{\csname the#1\endcsname}%
        \fi
        #7}}%
  \fi
  \@xsect{#5}}

% Remark:
% * counter definition: \newcount\c@secnumdepth 
%                       \newcount\c@tocdepth
\end{lstlisting}

To see these counter values, use:
\begin{lstlisting}
\makeatletter
\the\c@tocdepth
\the\c@secnumdepth

\setcounter{tocdepth}{2}
\setcounter{secnumdepth}{5}
\the\c@tocdepth
\the\c@secnumdepth
\makeatother

% Result:
% 3 3 2 5 
\end{lstlisting}

Please use commands \verb|\newcount| only in package code and with much care. The other counter is used in command \verb|\@dottedtocline|, see:
\begin{lstlisting}
\def\@dottedtocline#1#2#3#4#5{%
  \ifnum #1>\c@tocdepth \else
  ...}
\end{lstlisting}

Almost all the section commands, like \verb|\section|, \verb|\subsection|, etc. are defined by \verb|\@sect|. Thus they will call 
this command to add the contentsline. Something like:
\begin{lstlisting}
\contentsline {section}{\numberline {1}<sse I>}{<page>}{}%
\contentsline {subsection}{\numberline {1.1}<sss I>}{<page>}{}%
\contentsline {subsection}{\numberline {1.2}<sss II>}{<page>}{}%
\contentsline {subsection}{\numberline {1.3}<sss III>}{<page>}{}%

% Remark:
% * The \numberline was added for that '\ifnum #2 < \c@secnumdepth'
\end{lstlisting}

If you load package `hyperref', then the ToC file will look like:
\begin{lstlisting}
\contentsline {section}{<sse I>s}{<page>}{section.1}%
\contentsline {subsection}{<sss I>}{<page>}{subsection.1.1}%
\contentsline {subsection}{<sss II>}{<page>}{subsection.1.2}%
\contentsline {subsection}{<sss III>}{<page>}{subsection.1.3}%
\end{lstlisting}



\subsection{The command \textbackslash @starttoc}
This command can be used to define \verb|\tableofcontents|, \verb|\listoffigures| etc. For example: 
\begin{itemize}
\item \verb|\@starttoc{lof}| is used in \verb|\listoffigures|. 
\item \verb|\@starttoc{lot}| is used in \verb|\listoftables|.
\item \verb|\@starttoc{toc}| is used in \verb|\tableofcontents|.
\end{itemize}

The full definition of this command is:
\begin{lstlisting}
\def\@starttoc#1{%
  \begingroup
    \makeatletter
    \@input{\jobname.#1}% The key command lies here !!
    \if@filesw
      \expandafter\newwrite\csname tf@#1\endcsname
      \immediate\openout \csname tf@#1\endcsname \jobname.#1\relax
    \fi
    \@nobreakfalse
  \endgroup}
\end{lstlisting}


\subsection{The command \textbackslash addcontentsline}
Usually, user have to use this command to add her/his own entries to the table of contents. The syntax is:
\begin{lstlisting}
\addcontentsline{<table>}{<type>}{<entry>}
\end{lstlisting}

What does this command do for us ? For example, after use command:
\begin{lstlisting}
\addcontentsline{toc}{chapter}{References}
\addcontentsline{toc}{chapter}{Index}
\end{lstlisting}

There will be two entries(lines) in file: \verb|\jobname.toc| as follows:
\begin{lstlisting}
% The last empty brace is for hyperref option
\contentsline {chapter}{References}{1}{}%
\contentsline {chapter}{Index}{1}{}%
\end{lstlisting}

Thus you can simply see this command as: \textbackslash\textbf{addcontentsline} = \textbf{add} + \textbackslash\textbf{contentsline}

The full definition of this command is:
\begin{lstlisting}
\def\addcontentsline#1#2#3{%
  \addtocontents{#1}{\protect\contentsline{#2}{#3}{\thepage}{}%
  \protected@file@percent}}

% Remark: 
% * \protected@file@percent:a percent sign in toc file
\end{lstlisting}


\subsection{The command \textbackslash addtocontents}
This command just add a text to the \verb|<table>| file without any manipulation. Thus if you use a command like:
\begin{lstlisting}
\addtocontents{toc}{something to be added}
\end{lstlisting}

There will be a line in toc file, something like:
\begin{lstlisting}
\contentsline {<type-1>}{<entry-1>}{<page-1>}{}
something to be added
\contentsline {<type-2>}{<entry-2>}{<page-2>}{}
\end{lstlisting}

The full definition is:
\begin{lstlisting}
\long\def\addtocontents#1#2{%
  \protected@write\@auxout
    {\let\label\@gobble \let\index\@gobble \let\glossary\@gobble}%
    {\string\@writefile{#1}{#2}}}% The key command lies here !!
\end{lstlisting}

Thus you can add any text to not only toc file but also lof, lot, etc. by using this command.

\subsection{The command \textbackslash contentsline}
This commands lies in the \verb|<table>| file, \textbf{it is not a command in normal context}. But if you want to add the 
contentsline manually, you can use this command in some places, such as the very begining of document, chapter, etc. 

\textbf{Remark}: The fourth argument is used when `hyperref' was loaded. This argument is saveed in a 
macro \verb|\@contentsline@destination|. 

The full definition is:
\begin{lstlisting}
\def\contentsline#1#2#3#4{\gdef\@contentsline@destination{#4}%
  \csname l@#1\endcsname{#2}{#3}}

% By default:
\let\@contentsline@destination\@empty
\end{lstlisting}

\verb|\csname l@#1\endcsname| is the old interface for \verb|<table>| creating. They are defined in each document class. In 
\verb|article.cls|, they are defined as:
\begin{lstlisting}
\newcommand*\l@section[2]{%
  \ifnum \c@tocdepth >\z@
    \addpenalty\@secpenalty
    \addvspace{1.0em \@plus\p@}%
    \setlength\@tempdima{1.5em}%
    \begingroup
      \parindent \z@ \rightskip \@pnumwidth
      \parfillskip -\@pnumwidth
      \leavevmode \bfseries
      \advance\leftskip\@tempdima
      \hskip -\leftskip
      #1\nobreak\hfil
      \nobreak\hb@xt@\@pnumwidth{\hss #2%
                                \kern-\p@\kern\p@}\par
    \endgroup
  \fi}
\newcommand*\l@subsection{\@dottedtocline{2}{1.5em}{2.3em}}
\newcommand*\l@subsubsection{\@dottedtocline{3}{3.8em}{3.2em}}
\newcommand*\l@paragraph{\@dottedtocline{4}{7.0em}{4.1em}}
\newcommand*\l@subparagraph{\@dottedtocline{5}{10em}{5em}}
\end{lstlisting}


\subsection{The command \textbackslash @dottedtocline}
This command is used to create a dotted line in table of contents, for sections, for figures, for tables, etc.
If we'd like to custom this dotted line, just modify these 4 arguments.


The full definition is:
\begin{lstlisting}
\def\@dottedtocline#1#2#3#4#5{%
  \ifnum #1>\c@tocdepth \else
    \vskip \z@ \@plus.2\p@
    {\leftskip #2\relax \rightskip \@tocrmarg \parfillskip -\rightskip
      \parindent #2\relax\@afterindenttrue
      \interlinepenalty\@M
      \leavevmode
      \@tempdima #3\relax
      \advance\leftskip \@tempdima \null\nobreak\hskip -\leftskip
      {#4}\nobreak
      \leaders\hbox{$\m@th
          \mkern \@dotsep mu\hbox{.}\mkern \@dotsep
          mu$}\hfill
      \nobreak
      \hb@xt@\@pnumwidth{\hfil\normalfont \normalcolor #5}%
      \par}%
  \fi}
\def\numberline#1{\hb@xt@\@tempdima{#1\hfil}}
\end{lstlisting}

When and How this command generate the dotted line? This commands has not been used in \verb|latex.ltx| but in some other basic 
document class, such as article class, book class, etc. See the usage in \verb|article.cls|:
\begin{lstlisting}
\newcommand*\l@section[2]{%
  \ifnum \c@tocdepth >\z@
    \addpenalty\@secpenalty
    \addvspace{1.0em \@plus\p@}%
    \setlength\@tempdima{1.5em}%
    \begingroup
      \parindent \z@ \rightskip \@pnumwidth
      \parfillskip -\@pnumwidth
      \leavevmode \bfseries
      \advance\leftskip\@tempdima
      \hskip -\leftskip
      #1\nobreak\hfil
      \nobreak\hb@xt@\@pnumwidth{\hss #2%
                                \kern-\p@\kern\p@}\par
    \endgroup
  \fi}
\newcommand*\l@subsection{\@dottedtocline{2}{1.5em}{2.3em}}
\newcommand*\l@subsubsection{\@dottedtocline{3}{3.8em}{3.2em}}
\newcommand*\l@paragraph{\@dottedtocline{4}{7.0em}{4.1em}}
\newcommand*\l@subparagraph{\@dottedtocline{5}{10em}{5em}}
\newcommand\listoffigures{%
    \section*{\listfigurename}%
      \@mkboth{\MakeUppercase\listfigurename}%
              {\MakeUppercase\listfigurename}%
    \@starttoc{lof}%
    }
\newcommand*\l@figure{\@dottedtocline{1}{1.5em}{2.3em}}
\newcommand\listoftables{%
    \section*{\listtablename}%
      \@mkboth{%
          \MakeUppercase\listtablename}%
        {\MakeUppercase\listtablename}%
    \@starttoc{lot}%
    }
\let\l@table\l@figure
\end{lstlisting}


\textbf{Remark}: This command defines the page number format, the number align, the number width, the number color. See the 
following definition:
\begin{lstlisting}
\def\@dottedtocline#1#2#3#4#5{%
  ...
  \hb@xt@\@pnumwidth{\hfil\normalfont \normalcolor #5}%
  ...
}
\end{lstlisting}


\subsection{The command \textbackslash numberline}
This command is simple to use. This command pull page number to right most in a box of width \verb|\@tempdim|. The 
full definition is:

\begin{lstlisting}
% \hb@xt@ = \hbox to 
\def\numberline#1{\hb@xt@\@tempdima{#1\hfil}}
\end{lstlisting}

\textbf{Remark}: This command is the left label(such as 1.1, 1.1.1, etc.) in \verb|<table>|, Thus we can redefine this 
command the modify the left margin of toc, lof, lot entries.



\section{Conclusion}
\subsection{traditional way}
Then we can see how the \verb|table| is generated in \LaTeX{}:

\begin{itemize}
  \item Firstly, In each section command, the \verb|\addcontentsline| will be called, then a line will be added to the \verb|<table>| file.
    \begin{itemize}
      \item For the command \verb|\contentsline| in the \verb|<table>| file, the \verb|\csname l@#1\endcsname| will be called. The first 3 arguments 
        of command \verb|\@dottedtocline| are provided by it.
      \item Then the \verb|\@dottedtocline| will be called by \verb|\csname l@#1\endcsname|. The ``entry label(like 1.1, 1.1.1, etc) +
        entry name''[{\color{gray}2nd in \verb|\contentsline|, 4th in \verb|\@dottedtocline|}] and ``page number''[{\color{gray}3rd in \verb|\contentsline|, 5th in \verb|\@dottedtocline|}]
        will be grabbed by this command. 
      \item If package ``hyperref'' is loaded, then the last argument of \verb|\contentsline| will be saved in a macro \verb|\@contentsline@destination|, 
        which will be used by \verb|hyperref| package in the future.
      \item The ``entry label'' will be packed in a box of width \verb|@tempdima|, ``entry name'' will follow the ``entry label'' out of the box. 
        Then they will be put in the left side of the \verb|<table>| line; The page number will also be packed in a box of width \verb|\@pnumwidth| and a dotted line will be 
        added by command \verb|\leaders\hbox{$<item>$}\nobreak|.
    \end{itemize}
  \item Secondly, In the \verb|\@starttoc{toc}| command will be called by \verb|\tableofcontents|, 
  \item Finally, \verb|<table>| file will be inputed and the contents will be shown in the document.
\end{itemize}


\subsection{titletoc package}
If you use package `titletoc' package or `titlesec' package, then there will be something defferent.
\end{document}