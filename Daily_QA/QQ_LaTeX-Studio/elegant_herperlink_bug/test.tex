\begin{filecontents}[force]{\jobname.ist}
headings_flag 1
heading_prefix "\\par\\penalty-50\\textbf{"
heading_suffix "}\\par\\nobreak"
\end{filecontents}
\documentclass[cn,hang,blue,chinese,fancy,twocol,biber]{elegantbook}

\title{数学分析}
\subtitle{学习笔记}

\author{absinthe}
\date{\today}
\version{0.0}
\bioinfo{邮件:}{??}

\extrainfo{日月逝矣,岁不我与。——论语}


\cover{cover.jpg}
\logo{logo.jpg}

% 本文档命令
\usepackage{float}% 浮动体图表设置
\renewcommand\topfraction{.9}
\renewcommand\textfraction{0.35}
\renewcommand\floatpagefraction{0.8}
\setlength\columnseprule{0.01pt} % 分栏线
\setcounter{tocdepth}{3}

% \usepackage[xindy]{imakeidx} %在 mkind.ist 格式文件进行了修改
\makeindex[options={-s \jobname.ist}]
\usepackage{hyperref}
% \hypersetup{
%	colorlinks=true,
	%linkcolor=blue,
	%filecolor=blue,
	%urlcolor=blue,
	%citecolor=blue
%}
\makeindex
% 快捷命令
% bb 系列 黑板
\usepackage{amssymb}
\newcommand\bbR{\mathbb{R}}

% 新列表设置
% proenumerate 列表 性质(property)列表
\newlist{proenumerate}{enumerate}{1}
\setlist[proenumerate, 1]{itemsep=0pt,parsep=0pt,label=(\arabic*)}

% 实现\chapter随\part的编号增加而清零重新编号
\makeatletter \@addtoreset{chapter}{part} \makeatother 

\begin{document}
\maketitle
	
\tableofcontents
\mainmatter
\chapter{序列极限}
\section{实数系连续性的基本定理}
\subsection{戴德金分割定理}
\begin{theorem}{戴德金分割定理}\index{戴德金分割定理}
	对 $\bbR$ 的任一分划 $(A|B)$ , $B$ 中必有最小数.	
\end{theorem}
	
\subsection{确界存在定理}
\begin{theorem}{确界存在定理}\index{确界存在定理}
	非空有上界的实数集必有上确界; 非空有下界的实数集必有下确界.
\end{theorem}
\subsection{闭区间套定理}
\begin{theorem}{闭区间套定理}\index{闭区间套定理}
	设 $\{[a_n,b_n]\}$ 是一列闭区间, 并满足:
	\begin{proenumerate}
		\item $[a_n,b_n] \supseteq [a_{n+1},b_{n+1}],\; n=1,2,\cdots$;
		\item $\displaystyle\lim_{n\to\infty}(b_n-a_n)=0$,
	\end{proenumerate}
	则存在唯一的一点 $c\in\bbR$, 使得 $c\in[a_n,b_n],\; n=1,2,\cdots,$ 即
	\[\{c\}= \bigcap_{n=1}^{\infty}[a_n,b_n].\]
\end{theorem}
\backmatter
\appendix
\printindex

\end{document}