
% 生成最终的图象时把第一个文档类取消注释即可
\documentclass[10pt,varwidth]{standalone}
% \documentclass[12pt]{article}
% 1.必须添加varwidth选项,不然就会报错
\PassOptionsToPackage{quiet}{fontspec}
\usepackage{ctex}
\usepackage{geometry}
% 必须要保证绘图的纸张足够的大
\geometry{a2paper,left=1in,right=1in,top=1in,bottom=1in}
\usepackage{xifthen}
\usepackage{xfp}
\usepackage{xcolor}
\usepackage{pgfplots}
\usepackage{pgfplotstable}
\pgfplotsset{compat=1.16}
% 2.引用的tikz库
\usetikzlibrary {matrix, chains, trees, decorations}
\usetikzlibrary {arrows.meta, automata,positioning}
\usetikzlibrary {decorations.pathmorphing, calc}
\usetikzlibrary {backgrounds, mindmap,shadows}
\usetikzlibrary {patterns, quotes, 3d, shadows}
\usetikzlibrary {graphs, fadings, scopes}
\usetikzlibrary {arrows, shapes.geometric}
\usepgflibrary {shadings}

\tikzset{
    >={Latex[length=6mm, width=2mm]}
}


\definecolor{RiceYe}{HTML}{FFFFE4}



\begin{document}
% \vspace*{20em}

\begin{tikzpicture}[->]
    
    % 1.声明node
    \foreach \x/\name in {1, 2, 3, 4, 5}{
        % 1.批量命名所有的节点,产生从A11~A53共15个标记node
        \foreach \y in {1,2,3}{
            \ifnum\x>1 \ifnum\x<5
                \ifnum\y >1
                    \ifnum \x = 4
                        \node at (\x*4, {-(\y-1)*4}) (A\x\y)%
                            {${O_\fpeval{\y-1}}$};
                    \else
                        \node at (\x*4, {-(\y-1)*4}) (A\x\y)%
                            {$\fpeval{\y-1}$};
                    \fi
                \fi 
            \fi\fi
            \node at (\x*4, {-(\y-1)*4}) (A\x\y) {}; % 这里必须添加{},不然会解析错误
        }
    % 2.批量连接第3,4层节点
        \ifnum\x < 5 
            \ifnum\x > 2
                \foreach \y in {2,3}{
                    \foreach \nexty in {-4, -8}{
                        % shorten <= 1.5em 和 shorten >= 1.5em 分别将箭头的起点和终点缩短了 1.5em
                        % \draw[->, thick] (A2J) -- node[midway,above=1pt]{sigmoid 1}(A2k);
                        \draw[thick, shorten <= 1.5em, shorten >= 1.5em]%
                             (4*\x-4, -4*\y+4) -- node[pos=.4]{ $w_{\fpeval{\y-1}, \fpeval{-(\nexty/4)}}$ }(4*\x, \nexty);
                    }
                }
            \fi 
        \fi
    }

    % 3.给2,3,4层节点设置样式
    \foreach \x in {2, 3} {
        \foreach \y in {2,3}{
            \draw[thick] (A\x\y) circle(1);
        }
    }

    % 4.补充层名
    \foreach \ceilnum/\ceilname in {1/第$i-1$层, 2/第$i$层, 3/输出层:$O$}{
        \node[below=3em] at (A\fpeval{\ceilnum+1}1) {\ceilname};
    }
    % 不想用draw,但是使用node 也不方便时,便可以考虑使用\path命令
    % \path (A12) -- node[pos=.5, above] {输入} (A13);
    % \path (A52) -- node[pos=.5, above] {输出} (A53);
    \node[left] at (A12) {$t_1$};
    \node[left] at (A13) {$t_2$};
    \node[right] at (A52) {$e_1$};
    \node[right] at (A53) {$e_2$};
    \foreach \y in {2, 3}{
        \draw[thick, shorten >= 1.5em] (A1\y) -- (A2\y);
        \draw[thick, shorten <= 1.5em] (A4\y) -- (A5\y);
    }
    \foreach \n in {2, 3}{
        \draw[opacity=0.5, fill=orange!50!white, draw=none] (A\n2)+(-1.25, 3) rectangle ($(A\n3) + (1.25, -1.25)$);  
    }
    \draw[thick] (A51)+(-1,-1) rectangle ($(A53) + (1, -1)$);
    \node[below=3em] at (A51) {误差}; 

    % 几条bezier曲线
    \draw[blue, thick] (20,-4) .. controls (16,-1) and (11,-1) .. (10,-4);
    \draw[blue, thick] (20,-8) .. controls (16,-5) and (11,-5) .. (10,-8);
    % 图注
    \node[circle, fill=white, inner sep=1pt, draw=blue, line width=1pt] at (18, -2.77) (B1) {};
    \node[circle, fill=white, inner sep=1pt, draw=blue, line width=1pt] at (18, -6.77) (B2) {};
    \draw[->, thick, blue] (B1) -- ($(B2) + (0, -5em)$)node[below=1pt] {反向传播};

\end{tikzpicture}
\end{document}