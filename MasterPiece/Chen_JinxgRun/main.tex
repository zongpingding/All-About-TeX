\documentclass{article}
\usepackage[top=2.5cm, bottom=2.5cm, left=3cm, right=3cm]{geometry}
\PassOptionsToPackage{quiet}{fontspec}
\usepackage[12pt]{ctex}
\usepackage{framed}
\usepackage{amsmath}
\usepackage{enumitem}
\usepackage{amsthm}
\usepackage{zhnumber}
\usepackage{titlesec}
\newfontfamily{\russia}[Path=./Fonts/cmu/]{cmunrm.ttf}[
    BoldFont=cmunbx.ttf,
    ItalicFont=cmunbi.ttf
]
\usepackage{hyperref}
\hypersetup{
    colorlinks = true,
    urlcolor   = black,
    linkcolor  = black,
    citecolor  = black,
}


% 章节格式
\renewcommand{\thesection}{\zhnum{section}}
\renewcommand{\thesubsection}{}
\titleformat{\section}{\Large\centering\bfseries}{\thesection.~}{0pt}{}{}
\titleformat{\subsection}[runin]{\bfseries}{}{0pt}{}{}
% \newtheoremstyle{lemma}
%     {2pt}{2pt}{}
%     {0pt}{\bfseries}{}
%     {.25em}{\thmname{#1}~\arabic{subsection}.}
% \theoremstyle{lemma}
% \newtheorem{lemma}{引理}[section]

% command
\newcommand{\floor}[1]{\lfloor#1\rfloor}
\newcommand{\ee}{\mathrm{e}}
\newcommand{\dd}{\;\mathrm{d}}
\renewcommand{\proof}{\par \textbf{证}.~}



\title{大偶数表为一个素数及一个不超过二个素数的乘积之和}
\author{陈景润}
\date{{\small(中国科学院数学研究所)}}
\begin{document}
\maketitle
\thispagestyle{empty}
% 摘要
\begin{abstract}
    本文的目的在于用筛法证明了每一充分大的偶数是一个素数及一个不超过两个素数乘积之和.
    
    关于孪生素数问题亦得到类似的结果.
\end{abstract}
% 目录
\begin{center}
    \tableofcontents
\end{center}
\setcounter{page}{0}
\newpage

\section{引言}
把命题“每一个充分大的偶数都能表示为一个素数及一个不超过$a$个素数的乘积之和”简记为 $(1, a)$.

不少数学工作者改进了筛法及素数分布的某些结果,并用以改善$(1,a)$.现在我们将$(1, a)$发展历史简述如下:
\begin{itemize}
    \item $(1, c)$ --- Renyi
    \item $(1, 5)$ --- 潘承洞, {\russia Барбан}
    \item $(1, 4)$ --- 王元,潘承洞, {\russia Барбан}
    \item $(1, 3)$ --- {\russia Бухщтаб, Виноградов}, Bombieri
\end{itemize}

在文献[10]中我们给出了$(1,2)$的证明提要. 命 $p_x(1, 2)$为适合下列条件的素数 $p$的个数:
\begin{align*}
    x - p = p_1 \text{\quad 或 \quad} x - p = p_2p_3
\end{align*}
其中 $p_1, p_2, p_3$都是素数.用$x$表示一充分大的偶数.命 
\begin{align*}
    C_x = \prod_{\substack{p|x\\p>2}} \frac{p-1}{p-2}\prod_{p>2}(1-\frac{1}{(p-1)^2})
\end{align*} 
对于任给定的偶数 $h$ 及充分大的 $x$, 用 $x_k(1, 2)$表示满足下面条件的素数 $p$的个数:
\begin{align*}
    p\le x, \qquad p+h = p_1\quad \text{ 或 }\quad  p+h =p_2p_3
\end{align*}
其中 $p_1, p_2, p_3$都是素数. 

本文的目的在于证明并改进作者在文献[10]内所提及的全部结果,现在详细叙述如下.
\begin{itemize}
    \item \textbf{定理1.} ~ $(1,2)$及 $p_x(1, 2)\ge \frac{0.67xC_x}{(\log x)^2}$
    \item \textbf{定理2.} ~ 对于任意的偶数 $h$,都存在无限多的素数 $p$,使得 $p+h$的素因子的个数不超过两个及
        $x_h(1, 2) \ge \frac{0.67xC_x}{(\log x)^2}$
\end{itemize}

在证明定理1时,主要用到了本文中的引理8和引理9.在证明引理8时,我们使用较为
简单的数字计算方法;而证明引理9时,我们使用了Bombieri定理及Richert中的一个结
果.


\section{几个引理}
\subsection{引理1}
假设 $y\ge 0$, 而 $[\log x]$ 表示 $\log x$的整数部分, $x>1$, 
\begin{align*}
    \Phi(y) = \frac{1}{2\pi i}\int_{2-i^\infty}^{2+i^\infty}\frac{y^{\omega}\dd\omega}{\omega\left(1+\frac{\omega}{(\log x)^{1.1}}\right)^{[\log x]+1}}
\end{align*}
显见,当 $y\le 1$时,有 $\Phi(y) = 0$. 对于所有的 $y\ge 0$,则 $\Phi(y)$是一个非减函数. 当 $\log x\ge 10^4$ 及 
$y\ge 2\ee^{2(\log x)^{0.1}}$时,则有:
\begin{align*}
    1 - x^{-0.1} \le \Phi(y) \le 1
\end{align*}

\proof 我们先来证明 
\begin{align}
    \frac{\partial^r}{\partial\omega^r}\left(\frac{y^\omega}\omega\right)=\left(\frac{y^\omega}\omega\right)\left\{(\log y)^r+\sum_{i=1}^r\frac{(-1)^ir\cdots(r-i+1)(\log y)^{r-i}}{\omega^i}\right\}
\end{align}
成立. 显见, (1)式当 $r=1$和 $r=2$时都成立. 现在假定(1)式对于 $r=2,\cdots,s$时都成立,而证明对于 $S+1$也成立. 由于
\begin{align*}
    \frac{\partial^{s+1}}{\partial\omega^{s+1}}\left(\frac{y^w}\omega\right)
    & = \frac\partial{\partial\omega}\left\{y^w\left(\frac{(\log y)^s}\omega+\sum_{i=1}^s\frac{(-1)^iS\cdots(S-i+1)(\log y)^{s-i}}{\omega^{i+1}}\right)\right\} \\
    & = y^{\omega}\bigg\{
        \frac{(\log y)^{s+1}}{\omega}
        + \sum_{i=1}^{s}\frac{(-1)^{i}S\cdots(S-i+1)(\log y)^{s+1-i}}{\omega^{i+1}} \\
    &\qquad -\frac{(\log y)^{s}}{\omega^{2}}
        + \sum_{i=1}^s\frac{(-1)^{i+1}S\cdots(S-i+1)(i+1)(\log y)^{s-i}}{(i)^{t+2}} 
        \bigg\} \\
    & = \left(\frac{y^\omega}\omega\right)
        \bigg\{
        (\log y)^{s+1} - \frac{(S+1)(\log y)^s}\omega+\frac{(-1)^{s+1}(S+1)!}{\omega^{s+1}} \\
    &\qquad + \sum_{i=2}^s \left(\frac{(-1)^sS\cdots(S-i+1)(\log y)^{s+1-i}}{\omega^i} + \frac{(-1)^{\prime}S\cdots(S+2-i)i(\log y)^{s+1-i}}{\omega^i}\right)
        \bigg\} \\
    & = \left(\frac{y^\omega}\omega\right)\left\{(\log y)^{S+1} + \sum_{i=1}^{S+1}\frac{(-1)^i(S+1)\cdots(S+1-i+1)(\log y)^{s+1-i}}{\omega^i} \right\}
\end{align*}
故(1)式成立. 又当 $y\ge 1$时,我们有
\begin{align*}
    \Phi(y)
    & = 1+\left\{\frac{(\log x)^{1.1+1.1[\log x]}}{[\log x]!}\right\}\left\{\frac{\partial^{[\log x]}}{\partial\omega^{[\log x]}}\left(\frac{y^{\omega}}{\omega}\right)\right\}_{\omega=-(\log x)^{1.1}}  \\
    & = 1-e^{-(\log x)^{1.1}(\log y)}\sum_{\nu=0}^{[\log x]}\frac{\{(\log x)^{1.1}(\log y)\}^{\nu}}{\nu!} \\
    & = \left\{\frac{1}{\left[\log x\right]!}\right\}\int_{0}^{(\log x)!^{-1}(\log y)}e^{-\lambda}\lambda^{[\log x]}\dd\lambda
\end{align*}
因为 $0\le y\le 1$时,$\Phi(y)=0$.故由上式得到: 当$y\ge 0$时,则 $\Phi(y)$是一个非减函数. 又当 $y\ge \ee^{2(\log x)^{-1.0}}$时,有
\begin{align*}
    0 < 1-\Phi(y)
    & = \left\{\frac{1}{[\log x]!}\right\}\int_{(\log x)^{1.1}(\log y)}^{\infty}e^{-\lambda}\lambda^{[\log x]}\dd\lambda \\
    & = \left\{\frac{1}{[\log x]!}\right\}\int_{2[\log x]}^{\infty}e^{-\lambda}\lambda^{[\log x]}\dd\lambda \\
    & = \left\{\frac{\left(\left[\log x\right]\right)^{1+\left[\log x\right]}}{\left[\log x\right]!}\right\}
        \cdot \int_{2}^{\infty}e^{-\lambda[\log x]}\lambda^{[\log x]}\dd\lambda \\
    & = \left\{\ee^{-[\log x]}\frac{\left(\left[\log x\right]\right)^{1+\left[\log x\right]}}{\left[\log x\right]!}\right\}
        \cdot \int_{1}^{\infty}e^{-\lambda[\log x]}(1+\lambda)^{[\log x]}\dd\lambda \\
    & \le x^{-0.1}
\end{align*}
其中用到 $\log x\ge 10^4$ 及当 $\lambda\ge 1$时,有 $\ee^{\log(1+\lambda)}\le \ee^{\lambda\log 2}$.



\subsection{引理2}
令 $\ee(\alpha) = \ee^{2\pi i}\alpha, S(\alpha) = \sum_{n=M+1}^{M+N}{a_n\ee(n\alpha)}, Z = \sum_{n=M+1}^{M+N}{|a_n|^2}$,
其中 $a_n$是任意的实数. 我们用 $\sum_{\chi_q}^{*}$来表示和式之中经过且只经过 $q$模的所有原特征,则有
\begin{align}
    & \sum_{q\leq x}\frac{q}{\varphi(q)}\sum_{x_{q}}^{*}\Big|\sum_{n=M+1}^{M+N}a_{n}\chi_{q}(n)\Big|^{2}\le (X^{2}+\pi N)\sum_{n=M+1}^{M+N}\Big|u_{n}\Big|^{2} \\
    & \sum_{D<q\leq Q}\frac{1}{\varphi(q)}\sum_{x_{q}}^{*}\Big|\sum_{n=M+1}^{M+N}a_{n}\chi_{q}(n)\Big|^{2}\ll\Big(Q+\frac{N}{D}\Big)\sum_{n=M+1}^{M+N}|a_{n}|^{2}
\end{align}







\subsection{引理3}
当 $S=\sigma+it$和 $\sigma\ge \frac12$时,则有
\[
    \sum_{q\le Q}\sum_{x_{q}}^{*}\mid L(s,\chi_{q})\mid^{4}\ll Q^{2}\mid S\mid^{2}(\log Q)^{4} 
\]



\subsection{引理4}
当 $k$ 是无平方因子的奇数,而 $m\neq 1$时,我们有
\[
    \left|\sum_{x_k}^*\chi_k(m)\right|\le\left|(m-1,k)\right|   
\]



\subsection{引理5}
设 $x$ 是偶数, 则有 
\[
    \Omega \le \frac{M_1 + M_2}{1-e} + O(\frac{x}{(\log x)^{2.01}})    
\]

\subsection{引理6}
我们有 
\[
    M_2 \ll \frac{x}{(\log x)^{2.01}} 
\]

\subsection{引理7}
对于大偶数 $x$, 我们有
\[
    M_1 \le \left\{\frac{(8+24\varepsilon)xC_x}{\log x}\right\}
        \left\{\sum_{x^{1/10}<p_1\le x^{1/3}<p_2\le (\frac{x}{p_1}^{1/2})}^{}{\frac{1}{p_1p_2\log\frac{x}{p_1p_2}}}\right\}
\]
其中 $\displaystyle C_x = \prod_{\stackrel{p|x}{p>2}}\frac{p-1}{p-2}\prod_{p>2}\left(1-\frac{1}{(p-1)^2}\right)$


\subsection{引理8}
设$x$是大偶数,则有
\[
    \Omega \le \frac{3.9404xC_x}{(\log x)^2}    
\]

\subsection{引理9}
设$x$是大偶数,则有 
\[
    P_x(x, x^{1/10}) - (\frac12) \sum_{x^{1/10}<p\le x^{1/3}}^{}{P_x(x, p, x^{1/10})} 
    \ge \frac{2.6408xC_x}{(\log x)^2}   
\]
其中 $\displaystyle C_x = \prod_{\stackrel{p|x}{p>2}}\frac{p-1}{p-2}\prod_{p>2}\left(1-\frac{1}{(p-1)^2}\right)$



\section{结果}
显见,我们有
\begin{align}
    p_x(1, 2) \ge p_x(x, x^{1/10}) - (\frac12)\sum_{x^{1/10}<p\le x^{1/3}}^{}{p_x(x, p, x^{1/10})} - \frac{\Omega}{2} -x^{0.91}
\end{align}

由(28)式,引理8和引理9,即得到定理1
\[
    (1, 2)\quad \text{ 或 }\quad  P_x(1, 2)\ge \frac{0.67xC_x}{(\log x)^2}   
\]
的证明. 完全类似的方法可得到定理2的证明.

\end{document}
