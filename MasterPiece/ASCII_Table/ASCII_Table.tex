\documentclass{article}
\usepackage[margin=.5in, paperwidth=400mm, paperheight=280mm]{geometry}
\usepackage{xcolor}
\usepackage{tabularray}
\usepackage{ctex}
\usepackage{fancyhdr}
\pagestyle{fancy}
\def\headrule{}
\cfoot{\Large\sffamily American Standard Code for Information Interchange}
\usepackage{hyperref}
% ** NOTE **:
% 1. For that "tabularray environments need to see every & and \\ when splitting the table body with l3regex."
%    thus, your commands need to expand so that '&' and '\\' are visible to the tabularray parser.
% 2. Something should NOT be full expanded verse versa, for example, command '\lowascii' should 
%    NOT be fully expanded when tabularray is parsing the table body.
% 3. To display unicode symbol, you need to use a font that has these glyphs, while most fonts do not have
%    or the glyphs indexes are not right.
%    REF: https://tex.stackexchange.com/a/51400/294585
%   \newfontfamily{\lowasciifont}{OpenSans.ttf}                  ---> do NOT has these glyphs
%   \newcommand{\oldlowascii}[1]{{\lowasciifont\char\string"#1}} ---> some glyphs are not right
\newfontfamily{\lowasciifont}{FreeSans.otf} 
\newcommand{\lowascii}[1]{\innerlowascii#1\relax}
\def\innerlowascii#1#2#3\relax{%
  $\vcenter{
  \fontsize{4}{4}\sffamily
  \offinterlineskip
  \kern-1.2ex
  \sbox0{#1}\dimen0=\wd0 \box0
  \sbox0{#2}\moveright\dimen0\copy0 \advance\dimen0\wd0
  \if\relax\detokenize{#3}\relax\else\moveright\dimen0\hbox{#3}\fi}$}
\definecolor{Head}{HTML}{eaecf0}
\definecolor{Body}{HTML}{f8f9fa}
\definecolor{Lines}{HTML}{9aa2ab}
\definecolor{Blue}{HTML}{1645ad}
\definecolor{Red}{HTML}{b40000}
\hypersetup {
  colorlinks = true,
  urlcolor   = Blue,
}
\ExplSyntaxOn
\cs_generate_variant:Nn \prg_replicate:nn {e}
\cs_generate_variant:Nn \tl_count:n {e}
\cs_generate_variant:Nn \tl_range:nnn {e}
\cs_new:Npn \__oct_to_bin:n #1 
  {% to keep expandable, that's we in this way
    \tl_range:enn {\prg_replicate:en 
      {8-\tl_count:e {\int_to_bin:n {#1}}} 
      {0}\int_to_bin:n {#1}}{1}{4}
    \c_space_tl
    \tl_range:enn {\prg_replicate:en 
      {8-\tl_count:e {\int_to_bin:n {#1}}} 
      {0}\int_to_bin:n {#1}}{5}{8}
  }
\cs_set:Npn \__row_gen:n #1
  {% #1: oct number
    \__oct_to_bin:n {#1} & 
    #1 &
    \int_to_Hex:n {#1} &
    \char_generate:nn {#1}{12} \\
  }
\newcommand\controlSym[4]
  {% #1:oct number; #2:abbreviation; #3:like '^@'; #4:meaning 
    \__oct_to_bin:n {#1} & 
    #1 &
    \prg_replicate:en 
      {2-\tl_count:e {\int_to_Hex:n {#1}}} 
      {0}\int_to_Hex:n {#1} &
    #2 & 
    % \exp_after:wN \lowascii \exp_after:wN {\int_eval:n {2400+#1}} &
    \exp_not:N \lowascii{#2} &
    \tl_to_str:n {^#3} &
    #4 \\
  }
\newcommand\LoopRow[2]
  {%#1:start; #2: end
    \int_step_function:nnN {#1} {#2} 
      \__row_gen:n
  }
% \tl_set:Ne \l_tmpa_tl {\__oct_to_bin:n {33}}
% \tl_show:N \l_tmpa_tl
% \tl_set:Ne \l_tmpa_tl {\LoopRow{33}{36}}
% \tl_show:N \l_tmpa_tl
% --> INFO:
\hook_gput_next_code:nn {shipout/background}
  {
    \put(\paperwidth, -\paperheight){
      \makebox(0, 0)[br]{\parbox{16em}{
        \begin{itemize}
          \item Reference:~ from~ \href{https://www.runoob.com/w3cnote/ascii.html}{RUNOOB}
          \item Re-Typeset:~ by~ \href{https://github.com/zongpingding}{Eureka}
        \end{itemize}}
      }
    }
  }
\ExplSyntaxOff


\begin{document}
% \SetTblrOuter[tblr]{ expand=\AllRows}
\SetTblrOuter[tblr]{ baseline=T}
\SetTblrInner[tblr]{
  width=.4\linewidth,
  colspec = {ccccccc},
  columns = {leftsep=6pt,rightsep=6pt},
  % colspec = {X[1.5, c]X[1, c]X[1.5, c]X[1, c]},
  row{1}={fg=Blue, bg=Head, font=\sffamily\small},
  row{2-Z}={bg=Body, font=\sffamily\small},
  column{even}={fg=Blue},
  hlines={Lines, .5pt},
  vlines={Lines, .5pt},
}
\centering
\begin{tblr}[expand=\expanded]{
  width=.75\linewidth,
  colspec = {ccccccl},
  column{even}={fg=black},
  column{2}={fg=Blue},
  row{1}={fg=Blue, bg=Head, font=\sffamily\lowasciifont\small},
  column{6}={font=\sffamily\lowasciifont\small},
  column{Z}={fg=Blue},
  cell{1}{Z} = {c}
}
二进制 & 十进制 & 十六进制 & 缩写 & 
  \parbox{4em}{\centering Unicode\\ \textcolor{black}{表示法}} & 
  \parbox{4em}{\centering 脱出字符\\ \textcolor{black}{表示法}} & 名称/意义 \\
\expanded{\controlSym{0}{NUL}{@}{空字符(NULL)}}
\expanded{\controlSym{1}{SOH}{A}{\textcolor{red}{标题开始}}}
\expanded{\controlSym{2}{STX}{B}{\textcolor{red}{本文开始}}}
\expanded{\controlSym{3}{ETX}{C}{\textcolor{red}{本文结束}}}
\expanded{\controlSym{3}{EOT}{D}{\textcolor{red}{传输结束}}}
\expanded{\controlSym{5}{ENQ}{E}{\textcolor{red}{请求}}}
\expanded{\controlSym{6}{ACK}{F}{\textcolor{red}{确认回应}}}
\expanded{\controlSym{7}{BEL}{G}{\textcolor{red}{响铃}}}
\expanded{\controlSym{8}{BS}{H}{退格}}
\expanded{\controlSym{9}{HT}{I}{水平定位符号}}
\expanded{\controlSym{10}{LF}{J}{换行键}}
\expanded{\controlSym{11}{VT}{K}{\textcolor{red}{垂直定位符号}}}
\expanded{\controlSym{12}{FF}{L}{\textcolor{red}{换页键}}}
\expanded{\controlSym{13}{CR}{M}{CR(字符)}}
\expanded{\controlSym{14}{SO}{N}{取消变换 (Shift Out)}}
\expanded{\controlSym{15}{SI}{O}{启用变换 (Shift In)}}
\expanded{\controlSym{16}{DLE}{P}{\textcolor{red}{跳出数据通讯}}}
\expanded{\controlSym{17}{DC1}{Q}{\textcolor{red}{设备控制}一 (\textcolor{red}{XON 激活软件速度控制})}}
\expanded{\controlSym{18}{DC2}{R}{\textcolor{red}{设备控制}二}}
\expanded{\controlSym{19}{DC3}{S}{\textcolor{red}{设备控制}三 (\textcolor{red}{XOFF 停用软件速度控制})}}
\expanded{\controlSym{20}{DC4}{T}{\textcolor{red}{设备控制}四}}
\expanded{\controlSym{21}{NAK}{U}{确认失败回应}}
\expanded{\controlSym{22}{SYN}{V}{\textcolor{red}{同步用暂停}}}
\expanded{\controlSym{23}{ETB}{W}{\textcolor{red}{区块传输结束}}}
\expanded{\controlSym{24}{CAN}{X}{\textcolor{red}{取消}}}
\expanded{\controlSym{25}{EM}{Y}{\textcolor{red}{连线介质中断}}}
\expanded{\controlSym{26}{SUB}{Z}{替换}}
\expanded{\controlSym{27}{ESC}{[}{退出键}}
\expanded{\controlSym{28}{FS}{\ }{\textcolor{red}{文件分割符}}}
\expanded{\controlSym{29}{GS}{]}{\textcolor{red}{组群分隔符}}}
\expanded{\controlSym{30}{RS}{^}{\textcolor{red}{记录分隔符}}}
\expanded{\controlSym{31}{US}{_}{\textcolor{red}{单元分隔符}}}
\expanded{\controlSym{127}{DEL}{?}{Delete 字符}}
\end{tblr}
\quad
\begin{tblr}[expand=\expanded]{}
  二进制 & 十进制 & 十六进制 & 图形 \\
  0010 0000 & 32 & 20 &  (space) \\
  \expanded{\LoopRow{33}{63}}
\end{tblr}
\quad
\begin{tblr}[expand=\expanded]{}
  二进制 & 十进制 & 十六进制 & 图形 \\
  \expanded{\LoopRow{64}{95}}
\end{tblr}
\quad
\begin{tblr}[expand=\expanded]{}
  二进制 & 十进制 & 十六进制 & 图形 \\
  0110 0000 & 96 & 60 & \textasciigrave\\
  \expanded{\LoopRow{97}{126}}
\end{tblr}
\end{document}
