\documentclass[12pt]{article}
\PassOptionsToPackage{quiet}{fontspec}
% \usepackage{fontspec}
\usepackage[a4paper, total={6.5in, 10in}]{geometry}
\usepackage{amsmath}
\usepackage{mathspec}
\usepackage{tikz}
\usepackage{ctex}
\usepackage{mathpazo}
\usepackage{pgfplots}
\pgfplotsset{compat=1.17}
\usepackage{xcolor}
\usepackage{framed}
\usepackage{float}
\usepackage{titlesec}

\renewcommand{\abstractname}{Introduce}
\setmainfont{comic.ttf}
% 先清空中文字体设置,再重新设置中文字体
\renewcommand\CJKrmdefault{}
\setCJKmainfont{STKAITI.TTF}



\title{求和符号使用方法}
\author{Eureka}
\date{\today}
\begin{document}
\maketitle    

\begin{abstract}
    Mathematicians just love sigma notation for two reasons. First, it provides a convenient
way to express a long or even infinite series. But even more important, it looks really
cool and scary, which frightens nonmathematicians into revering mathematicians and
paying them more money.

\hfill ---Calculus II for Dummies
\end{abstract}


\tableofcontents
\clearpage


\section{前言}
常见的求和符号的出处或者说是形式,主要有如下三个:
\begin{itemize}
    \item 纯粹的{\bf 代数}求和表达式,形如 $\sum\limits_{n}{\frac{1}{n(n+1)}}$,通常在 \textbf{求和\&求积}中出现, 
    \item 包含组合数的形式出现,常常含有 $\binom{n}{k}$,通常在 {\bf 组合求和}中出现
    \item 和数论中的乘性函数 $\varphi, \mu$等结合出现,形如 $\sum\limits_{d|n}$等,通常在{\bf 乘法数论}中出现
\end{itemize}

尽管在表面上这三个形式不同,但是它们背后有着同样的核心思想:{\bf 交换求和}
\begin{framed}
    假设有一个二元函数 $f(a, b)$,那么就有
    \begin{align}
        \sum_{a\in A}^{}{\sum_{b\in B}^{}{f}} 
        = \sum_{b\in B}^{}{\sum_{a\in A}^{}{f}}
    \end{align} 
\end{framed}

这个看似显然的结果在超过一般的情况下都是十分具有创造性的,所以任何情况下你看到
双重求和都应该考虑交换求和顺序。{\bf 如果遇到单重的求和,考虑把它转化成双重求和},
比如后面的生成函数。还有一些情况下,你应该考虑{\bf 改变求和的变量(表达式)}
比如下面这几个例子:
\begin{align*}
    \sum_{a\ge 0}^{}{\sum_{b\ge 0}^{}{f}} 
    = \sum_{k\ge 0}^{}{\sum_{\mathop{a, b}\limits_{a+b=k}}^{}{f}}
\end{align*}

\begin{align*}
    \sum_{a\in A}^{}{\sum_{b\in B}^{}{a^2(b+1)}}
 =  \sum_{a\in A}^{}{a^2\left(\sum_{b\in B}^{}{(b+1)}\right)}
 =  \left(\sum_{a\in A}^{}{a^2}\right)\cdot \left(\sum_{b\in B}^{}{(b+1)}\right)
\end{align*}

\section{代数求解}
\subsection{伸缩\&部分\;求和}
这个是最没有挑战和难度的求和,只要看到多项式分母(polynomial denominators)就可以考虑。
下面是一个Stanford考试题目:
\begin{align*}
    \sum_{n\ge 1}^{}{\frac{7n+32}{n(n+2)}\cdot \left(\frac34\right)^n}
   & = \sum_{n\ge 1}^{}{\left(\frac{16}{n}-\frac{9}{n+2}\right)\cdot\left(\frac34\right)^n} \\
   & = \lim_{n\to \infty}{\frac{16}{1}\times \left(\frac34\right)^1 + \frac{16}{2}\times \left(\frac34\right)^2
      - \frac{16}{n+1}\times \left(\frac34\right)^{n+1} - \frac{16}{n+2}\times \left(\frac34\right)^{n+2}} \\
   & = \frac{33}{2}
\end{align*}

\subsection{交换求和顺序}
下面是一个概率论中的例子:
Let $X$ and $Y$ be random variables (不一定独立), 那么有
\begin{align*}
    \mathbb{E}(X+Y) 
    & = \sum_{n\ge 0}^{}{nP(X+Y=n)} 
        = \sum_{n\ge 0}^{}{\sum_{a+b=n}^{}{nP(X=a,Y=b)}}\\
    & = \sum_{a, b\ge 0}^{}{{(a+b)nP(X=a,Y=b)}}   \\
    &= \sum_{a\ge 0}^{}{\sum_{b\ge 0}^{}{aP(X=a,Y=b)}}
        + \sum_{a\ge 0}^{}{\sum_{b\ge 0}^{}{bP(X=a,Y=b)}}\\
    & = \sum_{a\ge 0}^{}{aP(X=a)} + \sum_{b\ge 0}^{}{bP(Y=b)}\\
    & = \mathbb{E}(X) + \mathbb{E}(b)
\end{align*} 

下面我们来看一个在数论中关于整除的求和式子,实际上这个引理我们还会在后面用到。
\begin{framed}
    Let $n\ge 1$ be an Integer, 那么
    \begin{align}
        \sum_{d\big|n}^{}{\varphi(d)}   = n
    \end{align}

    注意:其中  $\varphi$满足 $\varphi * 1 = \mathrm{id}$
\end{framed}

又一个使用交换求和的例子,设  $n$ 是一个整数,证明:
\begin{align*}
    \sum_{k\ge 1}^{}{\varphi(k)\left\lfloor\frac{n}{k}\right\rfloor} = \frac12n(n+1)
\end{align*}

\subsection{Fourier 有限和}
一个经典的例子就是计算
\begin{align*}
    \sum_{k\ge 1}^{}{\binom{1000}{2k}}
\end{align*}

求解思路:注意到 
$\displaystyle \frac{1^n+(-1)^n}{2} = \left\{\begin{aligned} & 1,\quad n\mbox{是偶数}\\ & 0,\quad  n\mbox{是奇数}\end{aligned}\right.$
于是我们有:
\begin{align*}
    \sum_{k\ge 1}^{}{\binom{1000}{2k}} \\
    & = \frac12 \left(\sum_{n\ge 0}^{}{\binom{1000}{n}} + \sum_{n\ge 0}^{}{\binom{1000}{n}}\times (-1)^n\right)\\
    & = (1+1)^{1000} + (1-1)^{1000}\\
    & = 2^{999}
\end{align*}


\section{``模''求和}
\section{多重求和}
\subsection{乘性函数}
\subsection{Dirichlet 卷积}
\subsection{M\"obius 反演}



\section{生成函数}
\subsection{基本应用}
\subsection{线性递推序列}
\subsection{常见的生成函数}
\subsection{Snake Oil method}

\section{习题和解答}

\end{document}