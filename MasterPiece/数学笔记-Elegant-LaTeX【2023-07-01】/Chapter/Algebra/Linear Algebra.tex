\section{线性代数}
\subsection{线性相关}

$x^{2}$与$x|x|$在$C[-1,1]$是否线性相关?

\begin{proof}

    $c_{1} x^{2}+c_{2} x|x|=0 $, 
    在 $[0,1]$上$c_{1}=1,c_{2}=-1$,
    在 $[-1,0]$上$c_{1}=1,c_{2}=1$,
    所以它们在$C[-1,1]$是线性相关的. 
\end{proof}

\subsection{矩阵转置} 
\begin{align*}
    R =\begin{bmatrix}
        A\\
        B
    \end{bmatrix}
    \Longrightarrow
    R^T=\begin{bmatrix}
        A^T\\
        B^T
    \end{bmatrix}^T 
    = \left[A^T, B^T\right]
\end{align*}

\subsection{左行右列}
其实可以看做是,行列式的最左边乘以一个运算,这个运算是第二行减去第一行的$A$倍。
由此我们可以看出:分块矩阵也是符合左行右列的规则的。
\begin{align*}
    \left|
        \begin{matrix}
            E & O\\ 
            -A & E
        \end{matrix}
    \right|
    \cdot 
    \left|
        \begin{matrix}
            E & B\\ 
            A & E 
        \end{matrix}
    \right|
    = \left|
        \begin{matrix}
            E & B \\ 
            O & E - AB 
        \end{matrix}
    \right|
\end{align*}

\subsection{过渡矩阵}

\[AX = BY\]

因为$A\to B$的过度矩阵为$P = A^{-1}B$。
举例:令前后的基为$\{\xi_1 = 4, \xi_1^{'} = 2\}$, 一个向量在变换前后的坐标分别为$b_1 = 5, b_1^{'} = 10$
因为$4\times 5 = 2\times 10$,可以认为是基从$5\to10$,这个坐标变换就是$p^{-1} = \frac{1}{2}x$.所以我们可以知道基变换
就是$P = 2x$

\subsection{向量空间}
比如 $\mathrm{A}$ 有特征值 $\lambda_i$, 对应的特征向量为 $\mathrm{p}_i$
我把 $\mathrm{p}_i$ 正交化后的得到的正交基记为 $\xi_i$

疑问一: 那么 $\xi_i$ 是对应于 $A$ 的特征值为 $\lambda_i$ 的特征向量吗? 

疑问二:为什么$p_i$单位正交化后的$\xi_i$, 
\begin{align*}
    \xi 
    & = \biggl(\xi_1,\, \xi_2,\, \xi_3,\, \cdots,\, \xi_i\biggr)
      \Longrightarrow  A=\xi^T \operatorname{diag}\left(\lambda_1, \lambda_2, \cdots, \lambda_i\right) \xi
\end{align*}

\begin{proof}

当 $\mathrm{A}$ 是对称实矩阵时, 只有当 $\lambda_i$ 为重根时才需要对求出的特征向量正交化, 且正交化后的向量仍然是原矩阵的特征向量。

如 $\lambda_i$ 为二重根, 则有 $\mathrm{p}_1, \mathrm{p}_2$ 两个特征向量与之对应。
正交化后即有,
\begin{align*}
    \xi_i=p_1, \xi_2=p_2-\frac{[p_1,\xi_1]}{[\xi_1,\xi_1]}\xi_1=p_2-kp_1
\end{align*}

故 
\begin{align*}
    A \xi_2
    & = A\left(\mathrm{p}_2-k p_1\right)=A p_2-k A p_1 \\
    & =\lambda_i p_2-k \lambda_i p_2 = \lambda_i\left(\mathrm{p}_2-k p_1\right)
\end{align*}
当 $A$ 不是对称矩阵时, 正交化后的向量就不是原矩阵的特征向量.至于为何是这样的结构, 记住!
\end{proof}

\subsection{线性空间相关问题}
线性空间V上的所有线性变换所组成的线性空间$\tau$。那么$\tau$里边的所有元素
对V中的所有元素所用后还在V中吗?

怎么求线性空间的零元素?

\subsection{特征值问题}
\begin{theorem}[特征值的性质]
\begin{align}
    \prod_{i=1}^n \lambda_i=|A|\hspace*{0.3\linewidth}\sum_{i=1}^n a_{i i}=\sum_{i=1}^n \lambda_i
\end{align}
\end{theorem}

\begin{proof}

    \num{1} 从特征多项式考虑:若关于矩阵A的特征多项式 $f(\lambda)$ 有根:
    $\lambda_1, \lambda_2, \cdots, \lambda_n$.(可能有重根,但是不影响)

    那么根据代数学基本定理可以知道, 若关于矩阵A的特征多项式可以分解为:
    \begin{align*}
        |\lambda E - A|&=f(\lambda)=\left(\lambda-\lambda_1\right)\left(\lambda-\lambda_2\right) \cdots\left(\lambda-\lambda_n\right)
    \end{align*}

    当 $\lambda=0$ 时, 即有 
    \[
        |-1\cdot A|=(-1)^n|A|=(-1)^n \prod_{i=1}^n \lambda_i \Longrightarrow \prod_{i=1}^n \lambda_i=|A|
    \]
    第一个式子我们就得到了, 同时也可以得到
    \[
        LHS = \lambda^n+\left[-\sum_{i=1}^n \lambda_i\right] \lambda^{n-1}+\cdots
    \]
    
    \num{2} 从行列式的展开式来考虑: 
    \begin{align*}
        |\lambda E - A|
        & = \begin{bmatrix}
                \lambda-a_{11} &  \cdots & \cdots & -a_{n1}\\
                \cdots & \lambda-a_{22} & \cdots & \cdots\\
                \cdots\\
                -a_{nn}  &  \cdots  & \cdots & \lambda-a_{nn}\\
        \end{bmatrix}
        = \left[(\lambda-a_{11})(\lambda-a_{22})\cdots(\lambda-a_{nn})\right]+\cdots\\
        & = \lambda^n + \left[-\sum_{i=1}^{n} {a_{i i}} \right]\lambda^{n-1}+\cdots
    \end{align*}
    上边只展开一部分的原因:$\lambda^n, \lambda^{n-1}$ 只会出现在主对角线的那个式子中, 即
    \ensuremath{\displaystyle \prod_{i=1}^{n}{(\lambda-a_{ii})}\makebox[2em][l]{\kaishu \zihao{8}\textcolor{red}{主对角线}}}
    中,故有
    \[
        \sum_{i=1}^n a_{i i}=\sum_{i=1}^n \lambda_i
    \]
\end{proof}

