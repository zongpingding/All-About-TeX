\section{子空间为直和的几个等价条件}

\subsection{证明思路}
首先说明直和的几个等价条件:

\begin{framed}
\noindent
\num{1} 定义:$V_1\oplus V_2$\\
\num{2} 零向量 $\theta$ 的分解唯一\\
\num{3} $V_1\bigcap V_2=\theta$\\
\num{4} $\dim(V_1+V_2)=\dim(V_1)+\dim(V_2)$\\
\num{5} 若\seq[n_1]{\alpha}和\seq[n_2]{\beta}分别为向量空间$V_1,V_2$的一组基,
        那么\seq[n_1]{\alpha}~,~\seq[n_2]{\beta}是 $V_1+V_2$的一组基
\end{framed} 

\bigskip
我们只需要证明:$1 \Rightarrow 2 \Rightarrow 3 \Rightarrow 4 \Rightarrow 5 \Rightarrow 1$即可
(注:$\Rightarrow $表示能够推出)

\begin{proof}{${1} \Rightarrow  {2}$}

   当 $V_1\oplus V_2$时, $\theta = \alpha_1 +\alpha_2, ~~\alpha_1 \in V_1,\alpha_2\in V_2$的分解唯一\par 
   又因为 $\alpha_1 = \alpha_2=\theta$是上述方程的一个解。\par 
   因为解的唯一性,我们可以知道 $\alpha_1=\alpha_2=\theta$\par
   于是我们可以得到: $\theta =\alpha_1+\alpha_2$,当且仅当 $\alpha_1=\alpha_2=\theta$时成立\par
   证毕 $\square$
\end{proof}

\begin{proof}{${2} \Rightarrow {3}$}

    $\forall \alpha \in V_1\bigcap V_2$, 因为 $V_1\bigcap V_2$为一个子空间, 所以 $-\alpha\in V_1\bigcap V_2$\par 
    又因为子空间的封闭性,所以我们有 $\alpha+(-\alpha)=\theta\in V_1\bigcap V_2$.\par 
    根据上面 ${1} \Rightarrow {2}$的证明我们可以知道:$\alpha =-\alpha=\theta$\par
    最后根据 $\alpha$的任意性,可以知道 $v_1 \bigcap V_2=\theta$\par 
    证毕 $\square$
\end{proof}

\bigskip
\begin{proof}{${3} \Rightarrow {4}$}

    根据维数公式,我们可以得到: $\dim(V_1+V_2) =\dim(V_1) + \dim(V_2) -\dim(V_1\bigcap V_2)$.
    对于任意的 $\alpha\in V_1 \bigcap V_2$, 必然有 $\alpha =\theta$.(因为两个向量空间 $V_1, V_2$ 的交集为空集).
    故此结论显然\par 
    证毕 $\square$
\end{proof}

\bigskip
\begin{proof}{ ${4} \Rightarrow {5}$ }

    设\seq{\alpha},~\seq{\beta}分别为 $V_1, V_2$的基。根据子空间和的定义我们有:
    \[
        \seq[n_1]{\alpha},~\seq[n_2]{\beta}
    \]
    是 $V_1 +V_2$的一组基.设 $\forall \alpha\in V_1 +V_2$, 则:
    \[
        \alpha = \underbrace{\sum_{i=1}^{n_1}{k_i\alpha_i}}_{\in V_1} + \underbrace{\sum_{i=1}^{n_2}{l_i\beta_i}}_{\in V_2}  =\theta   
    \]
    根据 ${3} \Rightarrow {4}$的结论,我们可以知道:
    \[
        \sum_{i=1}^{n_1}{k_i\alpha_i} = \sum_{i=1}^{n_2}{l_i\beta_i}  =\theta
    \]
    同时,又因为 \seq[n_1]{\alpha}, \seq[n_2]{\beta}是向量空间 $V_1, V_2$的基,所以可以得到:
    \[
        k_i= l_i=0,~i \in \{1, 2, \cdots\}
    \]
    同时又因为 \im{\mathbf{span}(\alpha_i) + \mathbf{span}{(\beta_i)} = V_1 + V_2},且 $\alpha_1, ~\beta_i$线性无关,所以可以得到结论\par 
    证毕 $\square$
\end{proof}

\begin{proof}{${5} \Rightarrow {1}$}

    设 $\forall \alpha  = \beta_i +\gamma_i \in V_1 + V_2,~ \beta_i\in V_1, \gamma_i \in V_2$.同时假设
    \begin{align*}
        \alpha =  \beta_1 + \gamma_1\\
        \alpha = \beta_2 + \gamma_2        
    \end{align*}

    因为 $\alpha =\theta$, 所以可以得到:
    \[
        (\gamma_1-\gamma_2) +  (\beta_1 -\beta_2) = \theta    
    \]
    根据上面 ${1} \Rightarrow {2}$的结论可以知道:
    \begin{align*}
        \beta_1 = \beta_2\\
        \gamma_1 = \gamma_2    
    \end{align*}
    于是乎,$\alpha$的分解就是唯一的\par 
    证毕 $\square$
\end{proof}


\begin{formal}{blue!20}
  {\bf 备注: }我们在证明的过程中多次用到 ${1} \Rightarrow {2}$的结论,但是这个并没有逻辑问题. 
\end{formal}