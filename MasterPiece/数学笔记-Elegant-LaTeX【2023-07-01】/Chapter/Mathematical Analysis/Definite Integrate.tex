\section{定积分与黎曼和}
\begin{definition}
    设$f(x)$是定义在$[a,b]$上的有界函数,在$[a, b]$上任取分点$\{x_i\}^n_{i=0}$,做一种划分:
    \[
        P:a=x_0<x_1<x_2<\cdots<x_{n-1}<x_n=b
    \]
    并且任取点$\xi_i\in[x_{i-1},x_i]$.记小区间的长度为$\Delta x_{i-1}=x_i-x_{i-1}$,并令$\lambda=\max\limits_{1\le i\le n}(\Delta x_i)$,若当$\lambda\to 0$时,极限
    \begin{equation}
        \lim\limits_{\lambda\to 0}{\sum_{i}^{n}{f(\xi_i)\Delta x_i}}\nonumber
    \end{equation}
    存在,并且与划分P无关,又对$\xi_i$的取法无关,则称$f(x)$在$[a, b]$上Riemann可积,和式
    \begin{equation}
        S_n= \sum_{i=1}^{n}{f(\xi_i\Delta x_i)}\nonumber
    \end{equation}

    称为Riemann和,其极限$I$称为$f(x)$在$[a, b]$上的定积分,记为:
    \begin{equation}
    I=\int_{a}^{b}{f(x) \dd x}\nonumber
    \end{equation}
\end{definition}
下边给出几个基本的求取范例,其实本质思想都是相同的\\
\vspace*{3pt}


\begin{framed}
    \begin{align*}
        & I_1=\int_{a}^{b}{e^x \mathrm{d}x}
        & \displaystyle I_2= \int_{a}^{b}{x^2 \mathrm{d}x}
    \end{align*}
\end{framed}

\begin{formal}{blue!20}
    对于是否Riemann可积,其实就是需要我么们去计算积分的上限和下限。
\end{formal}

\begin{proof}\textsf{第一题}

对于积分上限,我们需要计算的表达式即为:
\[
    \lim_{n\to\infty}{\frac{b-a}{n}e^a\cdot \sum_{i=1}^{n}{e^{\frac{i\cdot(b-a)}{n}}}}\nonumber \\
\]

\begin{align*}
    {\rm LSH}
    &=\lim_{n\to\infty}
    {
        e^a\cdot\frac{b-a}{n}\frac{e^{\frac{b-a}{n}}[1-e^{\frac{b-a}{n}\cdot n}]}{1-e^{\frac{b-a}{n}}}
    }
    =\lim_{n\to\infty}
    {
        e^a\cdot\frac{b-a}{n}\cdot\frac{e^{ \frac{b-a}{n} }[e^{b-a}-1]}{\lim\limits_{n\to\infty}{   e^{  \frac{b-a}{n} }   }-1}
    }\\
    &=\lim_{n\to\infty}
    {
        e^a\cdot\frac{b-a}{n}\cdot
        \frac{e^{\frac{b-a}{n}}[e^{b-a}-1]}  {\frac{b-a}{n}}
    }
    =\lim_{n\to\infty}
    {
        e^a\cdot e^{\frac{b-a}{n}}\cdot[e^{b-a}-1] 
    }\\
    &=e^b-e^a
\end{align*}   


对于积分下限,我们需要计算的表达式即为:
\[
    \lim_{n\to\infty}{\frac{b-a}{n}e^a\cdot \sum_{i=1}^{n}{e^{\frac{(i-1)\cdot(b-a)}{n}}}}\nonumber
\]


\begin{align*}
    {\rm LSH}
    &=\lim_{n\to\infty}
    {
        e^a\cdot\frac{b-a}{n}\frac{e^{0}[1-e^{\frac{b-a}{n}\cdot n}]}{1-e^{\frac{b-a}{n}}}
    }
    =\lim_{n\to\infty}
    {
        e^a\cdot\frac{b-a}{n}\cdot
        \frac{e^{0}[e^{b-a}-1]}  {\frac{b-a}{n}}
    }\\ 
    &=\lim_{n\to\infty}
    {
        e^a\cdot e^{0}\cdot[e^{b-a}-1] 
    }
    =e^b-e^a\nonumber
\end{align*}

由此我们可以得出,积分上限和积分下限相等,所以Riemann和为:
\[
I_1=e^b-e^a
\]
\end{proof}


\begin{proof}\textsf{第二题}

首先做分割,$T=\{a, a+\frac{a-b}{n}, \cdots, a+\frac{(n-1)(b-a)}{n}, b\}$, 并取$\xi_i=$,此时的Riemann和为:

\begin{align*}
    I_2 
    & = \int_{a}^{b}{\frac{1}{x^2} \dd x}
        = \lim_{n\to\infty}{\sum_{i=1}^{n}{f(\xi_i)\frac{a-b}{n}}}\\
    & = \lim_{n\to\infty}{\sum_{i=1}^{n}{f(\xi_i)}}
\end{align*}
    
因为$\xi_i\in [x_{i-1}, x_i]$ 而且要求 $f(\xi_i)$ 作为通项,
要便于我们求和,即 $\frac{1}{\xi_i^2}$ 便于求和.
考虑到 $\frac{1}{2\times 3}=\frac{1}{2}-\frac{1}{3}$ ,受到启发,
所以我们便取 $\xi_i=\sqrt[]{x_{i-1}x_i}$ 的形式, 所以我们取
\begin{align*}
    \xi_i=\sqrt[]{[\frac{(i-1)(b-a)}{n}+a]\cdot[\frac{i\cdot(b-a)}{n}]}
\end{align*} 


C为一个待求解的常数。因为 $na+i(a-b)-na-(i-1)(a-b)=a-b\Longrightarrow {\rm C}=\frac{1}{a-b}$, 所以我们可以得到:
\begin{align*}
    C = \frac{1}{a-b}
\end{align*}

带入原式我们有:
\begin{align*}
    \mathrm{LHS} 
    & = \int_{a}^{b}{ \frac{1}{x^2}\dd x}=\lim_{n\to\infty}{\frac{a-b}{n}\sum_{i=1}^{n}{\frac{1}{x_{i-1}x_i}}}
      = \lim_{n\to\infty}{\frac{a-b}{n}\sum_{i=1}^{n}{{\rm C}\cdot[\frac{1}{a+\frac{(i-1)(a-b)}{n}}-\frac{1}{a+\frac{i\cdot(a-b)}{n}}]}}\\
    & = \lim_{n\to\infty}{\frac{1}{a-b}\cdot (a-b)\sum_{i=1}^{n}{[\frac{1}{a+\frac{(i-1)(a-b)}{n}}-\frac{1}{a+\frac{i\cdot(a-b)}{n}}]}}
      = \sum_{i=1}^{n}{[\frac{1}{na+(i-1)(a-b)}-\frac{1}{na+i\cdot(a-b)}]}\\
    & = \lim_{n\to\infty}\sum_{i=1}^{n}[(\frac{1}{na+0(a-b)}-\frac{1}{na+(a-b)})+(\frac{1}{na+(a-b)}-\frac{1}{na+2(a-b)})\\
    & \quad+\cdots+(\frac{1}{na+(n-1)(a-b)}-\frac{1}{na+n(a-b)})]\\
    & = \lim_{n\to\infty}{[\frac{1}{na+0(a-b)}-\frac{1}{na+n(a-b)}]}
      =\lim_{n\to\infty}{[\frac{1}{n}(\frac{1}{a}-\frac{1}{2a-b})]} = 0
\end{align*} 

有错,之后再检查检查!!
\end{proof}