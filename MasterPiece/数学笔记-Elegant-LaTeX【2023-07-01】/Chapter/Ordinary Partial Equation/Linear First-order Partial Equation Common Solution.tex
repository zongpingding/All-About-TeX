\newcommand{\coef}{\ensuremath{\displaystyle \mathrm{Exp}\{{-\int_{}^{}{p(x){\rm d}x}}\}}}
\newcommand{\ee}{\ensuremath{\mathrm{e}}}

\section{一阶线性微分方程的通解}
\begin{theorem}[一阶线性微分方程的通解]
     \begin{align}
          \mbox{标准形式:}\quad \frac{\dd y}{\dd x} + p(x)=q(x)&\nonumber\\
          &\mbox{通解:}\quad y=\mathrm{Exp}\{{-\int_{}^{}{p(x){\rm d}x}}\}\cdot\int_{}^{}{\biggl(\mathrm{Exp}\{{\int_{}^{}{p(x) {d\rm }x}}\} \cdot q(x)+C\biggr) {\rm d}x}
    \end{align}
\end{theorem}

\begin{proof}

首先两边同时乘以积分因子 \coef 即有:
\begin{align*}
     & \coef \cdot\frac{\dd y}{\dd x} + \coef \cdot p(x) = q(x)\cdot \coef \\
     \Longrightarrow\quad  & \frac{\dd }{\dd x}\biggl(y\cdot \coef \biggr)=q(x)\cdot\coef \\
     \Longrightarrow\quad  & y=\coef \cdot\int_{}^{}{\biggl(\coef \cdot q(x)\biggr)\dd x}
\end{align*}

下边说明第一个积分 
\[
    \coef = \gamma(x)
\]

不用加任意常数$C$, 假设
\[
    \coef = \gamma(x) + C    
\]

带入上边的式子即有
\begin{align*}
     y & =\quad  \ee^{-\gamma(x)-C}\cdot \int_{}^{}{e^{\gamma(x)+C}\cdot q(x) \dd x}\\
     & \Longrightarrow\quad  y=e^{-C}\cdot e^{-\gamma(x)}\cdot e^C\cdot \int_{}^{}{e^{\gamma(x)}\cdot q(x)\dd x}\\
     & \Longrightarrow\quad  y=e^{-\gamma(x)}\cdot\int_{}^{}{e^{\gamma(x)}\cdot q(x)\dd x}
 \end{align*} 

所以从上边我们可以看出,第一个积分可以不加常数C. 但是注意:上边只是非齐次方程的特解,
我们还应该加上齐次方程的通解,所以可以得到原微分方程的通解为:

\begin{framed}
    \begin{align*}
        y & = \ee^{-\int_{}^{}{p(x)\dd x}}\cdot\int_{}^{}{[\ee^{\int_{}^{}{p(x) \dd x}} \cdot q(x)+C] \dd x}
        = \mathrm{Exp}\{{-\int_{}^{}{p(x) \dd x}}\}\cdot\int_{}^{}{\biggl(\mathrm{Exp}\{{\int_{}^{}{p(x) {d\rm }x}}\} \cdot q(x)+C\biggr) \dd x}
    \end{align*}
\end{framed}
\end{proof}

\clearpage

\subsection{两个错误的带入公式的例子}
\begin{framed}
\[
    xy'+(x+1)y=\ee^x \hspace*{.5\linewidth} x'' -x = C_0   
\]
\end{framed}
\noindent{\bf 第一题}

{\bf 错解}

先求出齐次方程的通解
\[
    y={\rm Exp}\{\int_{}^{}{\frac{1+x}{x} dx}\}
\]
原来非齐次方程的解即为:
\begin{align*}
    y & = \ee^{\int_{}^{}{\frac{1+x}{x} \dd x}}[\int_{}^{}{e^{\int_{}^{}{\frac{1+x}{x} \dd x}}\cdot \frac{e^x}{x} \dd x}]
        = \ee^{\ln |x|+x}[\int_{}^{}{\ee^{\ln |x|+x} \cdot \frac{\ee^x} {x}\dd x}]\\
      & = |x|\ee^x\cdot\biggl(\int_{}^{}{\ee^{2x} \dd x}\biggr)
        = x\ee^x\cdot(\frac{1}{2}\ee^{2x}+C)
\end{align*}

\begin{formal}{blue!20}
    错因:积分因子乘过去没有取倒数
\end{formal}
{\bf 正解}

首先求得积分因子
\[
    \ee^{\int_{}^{}{\frac{x+1}{x} \dd x}}=x\ee^x    
\]
然后使用公式可以知道:
\begin{align*}
    & y = \frac{1}{x\ee^x}\cdot \int_{}^{}{\ee^x\cdot x\ee^x \dd x}
        = \frac{\ee^{-x}}{x}\cdot (\frac{1}{2}x\ee^x-\frac{1}{4}\ee^{2x}+C)
        = (\frac{1}{2}-\frac{1}{2x})\ee^x + C\cdot \frac{\ee^{-x}}{x}
\end{align*} 

仍然有错:原方程两边同时除以 $x$ 时没有对右边的 $e^x$ 除以 $x$.
只需要把第二种错误的解法稍微改善以下就可以得到正解
\begin{formal}{blue!20}
\[
    \displaystyle y = \frac{\ee^x}{2x}+C\cdot\frac{\ee^{-x}}{x}
\]
\end{formal}  

\noindent{\bf 第二题}

求解如下的二阶微分方程
\[
    x'' -x = C_0  \Longleftrightarrow x'' - x' + x' - x = 0
\]
即 $g'(t) - g'(t) = C_0$, 其中: $g(t) = x'(t) -x(t)$. 可以猜测 $g(t)$的形式如下:
\[
    g(t) = -C_0 + c\ee^{-t}    
\]
验证
\[
    g'(t) - g(t) = c\ee^{-t} -(-C_0 + c\ee^{-t}) = C_0    
\]
然后求解得到: $x'(t) - x(t) = c\ee^{-t} -C_0$.

但是如果直接代入公式是得不到正确 $g(t)$, 理由如下:
\begin{align*}
    g(t) = \ee^{-\int_{}^{}{-1}\dd t}\cdot \int_{}^{}{\ee^{\int_{}^{}{-1}}\dd t}
         = C_0 \ee^{t} (\ee^{-t}+\ee^{c_1}) 
         = C_0 + \ee^{c_1}\cdot C_0  e^t
         = C_0 + C\ee^t
\end{align*} 