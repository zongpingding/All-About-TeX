\section{非齐次方程的常数变易法}
考虑非齐次微分方程
\[
    \frac{\dd^nx}{\dd t^n} + \frac{\dd^{n-1}x}{\dd t^{n-1}}\cdot a_1(t) + \frac{\dd^{n-2}x}{\dd t^{n-2}}\cdot a_2(t) + \cdots + \frac{\dd x}{\dd t}\cdot a_{n-1}(t) + x\cdot a_n(t) = f(t)
\]

简要的记作(注: $a_0(t)=1$)
\begin{align}
    &x^{(n)}\cdot a_{0}(t) + x^{(n-1)}\cdot a_{1}(t) + x^{(n-2)}\cdot a_{2}(t) + \cdots + x^{(n-(n-1))}\cdot a_{n-1}(t) + x^{(0)}\cdot a_{n}(t) = f(t) 
    \label{非齐次方程}
\end{align}

对应的齐次方程记作:
\begin{align} 
    & \hspace*{3em}x^{(n)}\cdot a_{0}(t) + x^{(n-1)}\cdot a_{1}(t) + x^{(n-2)}\cdot a_{2}(t) + \cdots + x^{(n-(n-1))}\cdot a_{n-1}(t) + x^{(0)}\cdot a_{n}(t) = 0
    \label{齐次方程}
\end{align}
设$x_1(t), x_2(t), x_3(t), \cdots, x_n(t)$是齐次方程的基本解组, 因而我们设:
\begin{align*}
    x = c_1x_1(t) + c_2x_2(t)+\cdots + c_n x_n(t) \label{常数变易结果}
\end{align*}
    
然后使用常数常数变易法之后我们设:
\begin{align}
    x = c_1(t)x_1(t) + c_2(t) x_2(t) + \cdots + c_n(t) x_n(t)
\end{align}
    
对$x$进行多次求导的过程如下:\par
$
\begin{aligned}
        % \rule[0pt]{0.8\linewidth}{.6pt}\\
        \num{1}\quad  x = \sum_{i=1}^{n}{c_i(t)x_i(t)} \\
        % \rule[0pt]{0.8\linewidth}{.6pt}\\
\end{aligned}
$

$
\begin{aligned}
    \num{2}\quad  x' = \sum_{i=1}^{n}{\left[c_i(t)x_i'(t)+c_i'(t)x_i(t)\right]}&=\sum_{i=1}^{n}{c_i(t)x_i'(t)} + \sum_{i=1}^{n}{c_i'(t)x_i(t)} \\
    &\mathrm{Let}\quad \sum_{i=1}^{n}{c_i'(t)x_i(t)}=0 \label{约束方程(1)} \notice{约束方程(1)}\\
    &\Rightarrow x'=\sum_{i=1}^{n}{c_i(t)x_i'(t)}
\end{aligned} 
$


$
\begin{aligned}
    \num{3}\quad  x'' = (x')' = \sum_{i=1}^{n}{\left[c_i(t)x_i'(t)\right]'} &= \sum_{i=1}^{n}{c_i'(t)x_i'(t)} + \sum_{i=1}^{n}{c_i(t)x_i''(t)}  \\
    & \mathrm{Let}\quad \sum_{i=1}^{n}{c_i'(t)x_i'(t)}=0 \label{约束方程(2)} \notice{约束方程(2)}\\
    &\Rightarrow x''=\sum_{i=1}^{n}{c_i(t)x_i''(t)}
\end{aligned} 
$

$\mathbf{\vdots}$

$
\begin{aligned}
    \num{n-1}\quad  x^{(n-1)} = \sum_{i=1}^{n}{\left[c_i(t)x_i^{(n-2)}(t)\right]'} &= \sum_{i=1}^{n}{c_i'(t)x_i^{(n-2)}(t) + \sum_{i=1}^{n}{c_i(t)x_i^{(n-1)(t)}}}  \\
    &\mathrm{Let}\quad \sum_{i=1}^{n}{c_i'(t)x_i^{(n-2)}(t)}=0  \label{约束方程(n-1)} \notice{约束方程(n-1)}\\
    & \Rightarrow x^{(n-1)}=\sum_{i=1}^{n}{c_i(t)x_i^{(n-1)}(t)}
\end{aligned} 
$

$
\begin{aligned}
    \num{n}\quad  x^{(n)} = (x^{(n-1)})' &= \sum_{i=1}^{n}{\left[c_i'(t)x_i^{n-1}(t)\right]'} + \sum_{i=1}^{n}{c_i(t)x_i^{(n)}(t)} \\
    & \mathrm{Don't}\quad \mathrm{L}et\sum_{i=1}^{n}{c_i'(t)x_i^{(n-1)}(t)}=0\\
    & \Rightarrow x^{(n)} = \sum_{i=1}^{n}{\left[c_i'(t)x_i^{n-1}(t)\right]'} + \sum_{i=1}^{n}{c_i(t)x_i^{(n)}(t)}
\end{aligned} 
$

所以我们可以得出$x^{(n)}$的通式

\begin{align}
    &x^{(m)} = \sum_{i=1}^{n}{c_i(t)}x_i^{(m)}(t), ~~~~m \in [0, n) \nonumber\\
    &x^{(n)} = \sum_{i=1}^{n}{\left[c_i'(t)x_i^{n-1}(t)\right]'} + \sum_{i=1}^{n}{c_i(t)x_i^{(n)}(t)} \label{求导规律}
\end{align}



又因为$x_1(1), x_2(t), \cdots, x_n(t)$是\ref{齐次方程}的解(它们的线性组合也是\ref{齐次方程}的解),所以我们可以得到
\begin{align}
    \sum_{i=1}^{n}{x^{(n-i)}(t)a_i(t)}=0, ~~x(t) \in \{x_1(1), x_2(t), \cdots, x_n(t)\}
    \label{齐次解}
\end{align}

\noindent \textbf{关于下述思路的来源}\par
\begin{align*}
    & \underbrace{\left[\sum_{i=1}^{n}{c_i(t)x_i^{(n)}(t)}\right]a_0(t)}_{\mycircled{1}} + \underbrace{\left[\sum_{i=1}^{n}{c_i(t)x_i^{(n-1)}(t)}\right]a_1(t)}_{\mycircled{2}} +\cdots+\underbrace{\left[\sum_{i=1}^{n}{c_i(t)x_i^{(0)}(t)}\right]a_n(t)}_{\mycircled{3}} =f(t)\\
\end{align*}
下边我们把\mycircled{1}, \mycircled{2}, \mycircled{3}分别展开在一个如下的表格的对应列中:\par
\vspace*{2em}
\begin{figure}[!htb]
    \centering
    \begin{tabular}{p{13em}p{14em}p{13em}}
    \hline\\
    \mycircled{1} & \mycircled{2} & \mycircled{3}\\
    \hline\\
    $c_1(t)x_1^{(n)}(t)\cdot a_0(t)$ & $c_1(t)x_1^{(n-1)}(t)\cdot a_1(t)$ & $c_1(t)x_1^{(0)}(t)\cdot a_n(t)$\\
    \vspace*{1em}\\
    $c_2(t)x_2^{(n)}(t)\cdot a_0(t)$ & $c_2(t)x_2^{(n-1)}(t)\cdot a_1(t)$ & $c_2(t)x_2^{(0)}(t)\cdot a_n(t)$\\
    \vdots & \vdots & \vdots\\
    $c_n(t)x_n^{(n)}(t)\cdot a_0(t)$ & $c_n(t)x_n^{(n-1)}(t)\cdot a_1(t)$ & $c_n(t)x_n^{(0)}(t)\cdot a_n(t)$\\
    \hline\\
    \vspace*{1em}
    \end{tabular}
    \label{裂项}
    \caption{裂项说明}
\end{figure}



最后我们运用\ref{求导规律}和\ref{齐次解}, 然后再把\textsf{基本解组}$x_i(t)$的线性组合带入\ref{齐次方程},可以把这个方程化为:
\begin{align*}
    & = x^{(n)}\cdot a_{0}(t) + x^{(n-1)}\cdot a_{1}(t) + x^{(n-2)}\cdot a_{2}(t) + \cdots + x^{(n-(n-1))}\cdot a_{n-1}(t) + x^{(0)}\cdot a_{n}(t)\\
    & = \sum_{i=0}^{n}{x^{(n-i)}(t)a_i(t)}
      = x^{(n)} + \sum_{i=1}^{n}{x^{(n-i)}(t)a_i(t)} ~~~~\mathtext{注:因为}x^n\mathtext{比较特殊, 所以把它单独拿出来}\\
    & = \textcolor{cyan}{\overbrace{\sum_{i=1}^{n}{c_i(t)x_i^{(n)}(t)} + \sum_{i=1}^{n}{c_i'(t)x_i^{(n-1)}(t)}}^{x^{(n)}}} + \sum_{i=1}^{n}{x^{(n-i)}(t)a_i(t)}\\
    & = \textcolor{cyan}{\left[\sum_{i=1}^{n}{c_i(t)x_i^{(n)}(t)}\right]a_0(t)} + \left[\sum_{i=1}^{n}{c_i(t)x_i^{(n-1)}(t)}\right]a_1(t) +\cdots\\
    & \hspace*{15em} + \left[\sum_{i=1}^{n}{c_i(t)x_i^{(1)}(t)}\right]a_{n-1}(t) + \left[\sum_{i=1}^{n}{c_i(t)x_i^{(0)}(t)}\right]a_n(t) + \textcolor{cyan}{\sum_{i=1}^{n}{c_i'(t)x_i^{(n-1)}(t)}}\\
    & = \sum_{j=0}^{n}{\left\{ \left[ \sum_{i=1}^{n}{c_i(t)x_i^{(n-j)}(t)}\right]a_j(t)\right\}} + \textcolor{cyan}{\sum_{i=1}^{n}{c_i'(t)x_i^{(n-1)}(t)}}\\
    & = \sum_{j=0}^{n}{\left\{ \sum_{i=1}^{n}{\left[a_j(t)\cdot c_i(t)x_i^{(n-j)}(t)\right] }\right\}} + \textcolor{cyan}{\sum_{i=1}^{n}{c_i'(t)x_i^{(n-1)}(t)}}
      = \sum_{j=0}^{n}{\sum_{i=1}^{n}{\left[a_j(t)\cdot c_i(t)x_i^{(n-j)}(t)\right] }} + \textcolor{cyan}{\sum_{i=1}^{n}{c_i'(t)x_i^{(n-1)}(t)}}\\
    & = \sum_{i=1}^{n}{\sum_{j=0}^{n}{\left[a_j(t)\cdot c_i(t)x_i^{(n-j)}(t)\right] }} + \textcolor{cyan}{\sum_{i=1}^{n}{c_i'(t)x_i^{(n-1)}(t)}}
      = \sum_{i=1}^{n}{\left\{ c_i(t)\left[\sum_{j=0}^{n}{a_j(t)x_i^{(n-j)}(t)}\right]\right\}} + \textcolor{cyan}{\sum_{i=1}^{n}{c_i'(t)x_i^{(n-1)}(t)}}\\
    & = \sum_{i=1}^{n}{\left\{ c_i\cdot 0\right\}}+ \textcolor{cyan}{\sum_{i=1}^{n}{c_i'(t)x_i^{(n-1)}(t)}}
      = \textcolor{cyan}{\sum_{i=1}^{n}{c_i'(t)x_i^{(n-1)}(t)}}
      = f(t)
\end{align*}

于是我们得到了另外一个约束方程
\begin{align}
    \sum_{i=1}^{n}{c_i'(t)x_i^{(n-1)}(t)} = f(t) \label{约束方程(n)} \notice{约束方程(n)}
\end{align}
结合之前的$(n-1)$个方程(\ref{约束方程(1)}, \ref{约束方程(1)}, $\cdots$, \ref{约束方程(n-1)}和\ref{约束方程(n)})。我们就得到了含有$n$个未知数$c_i(t), ~~i = (1, 2, \cdots, n)$的$n$个方程, 它们组成一个线性方程组, 其系数行列式就是
$W[x_1(t), x_2(t), \cdots, x_n(t)]$, 它不等于0。 所以如下的方程组的解是唯一确定的。

\begin{center}
$
\left\{
    \begin{aligned}
        &\sum_{i=1}^{n}{c_i'(t)x_i(t)} & = 0\\
        &\sum_{i=1}^{n}{c_i'(t)x_i'(t)} & = 0\\
        &\sum_{i=1}^{n}{c_i'(t)x_i^{(n-2)}(t)} & = 0\\
        &\cdots&\\
        &\sum_{i=1}^{n}{c_i'(t)x_i^{(n-1)}(t)} & = f(t)
    \end{aligned}
\right.
$    
\end{center}

怎么求解这个线性方程组:
\begin{itemize}
    \item 1. 直接硬算这个高阶微分方程
    \item 2. 因为$x(t)$有已知,所以带入一个特定的$t$只值去求解
\end{itemize}

\noindent\textbf{正解}\par
由于所有的$x_i(t)$都是已知的,相当于线性方程组的系数矩阵已知,并且系数矩阵行列式
\[
    W[x_1(t), x_2(t), \cdots, x_n(t)]\neq 0
\]
所以可以直接运用克拉默法则求得
\[
    c_i(t) = \varphi_i(t), ~~~ i = 1, 2, 3, \cdots, n
\]
积分可以得到:
\[
    c_i(t)= \int_{}^{}{\varphi_i(t) \dd t} + \gamma_i, ~~~ i = 1, 2, 3, \cdots, n
\]
这里的$\gamma_i$为任意的常数。将得到的$c_i(t)(i = 1, 2, 3, \cdots, n)$的表达式带入\ref{常数变易结果}及可以得到方程\ref{非齐次方程}的解为:

\begin{theorem}
    \begin{align}
        x = \sum_{i=0}^{n}{\gamma_i x_i(t)} + \sum_{i=1}^{n}{x_i(t)\int_{}^{}{\varphi_i(t) \dd t}}
        \label{非齐次方程的解}        
    \end{align}
\end{theorem}

\begin{formal}{blue!20}
    这里求解就不用加常数C了,C已经再前边体现了
\end{formal}
显然表达式\ref{非齐次方程的解}是方程\ref{非齐次方程}的通解。但是为了得到一个解,只需要给常数$\gamma_i(i = 1, 2, 3, \cdots, n)$以
确定的值,比如$\gamma_i =0(i = 1, 2, 3, \cdots, n)$时, 即得到解
\[
    \sum_{i=1}^{n}{x_i(t)\int_{}^{}{\varphi_i(t) \dd t}}
\]

之前没有计算出结果的一个解法
\begin{align*}
    {\rm LHS} & = x^{(n)}\cdot a_{0}(t) + x^{(n-1)}\cdot a_{1}(t) + x^{(n-2)}\cdot a_{2}(t) + \cdots + x^{(n-(n-1))}\cdot a_{n-1}(t) + x^{(0)}\cdot a_{n}(t)\\
    & = x^{(n)} + \sum_{i=1}^{n}{\left[a_i(t)\cdot \sum_{j=1}^{n}{c_i(t)x_j^{(n-i)}(t)}\right]}
      = x^{(n)} + \sum_{i=1}^{n}{\left[\sum_{j=1}^{n}{a_i(t)\cdot x_j^{(n-i)}(t) c_i(t)}\right]}\\
    & = x^{(n)} + \sum_{i=1}^{n}{\left[\sum_{j=1}^{n}{0\cdot c_j(t)}\right]}
      = x^{(n)}
      =f(t)
\end{align*}


