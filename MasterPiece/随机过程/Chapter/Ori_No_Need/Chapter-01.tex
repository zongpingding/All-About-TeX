% some newcommand defined
\newcommand{\om}{\ensuremath{\Omega}\,}
\newcommand{\f}{\ensuremath{\sigma}-field\,}


\s{1.1}{Events and Probability}

\begin{definition}
let \om be a non-empty set , A \f or  $\sigma$-Algebra \C{F} on \om is a family
of subset of \om, such that\footnote[1]{These three as follows are Necessary and Sufficient condition(充分必要条件); 
另外,sufficient condition: 充分条件; necessary conditio: 必要条件}: 
\begin{enumerate}[(i)]
    \item \im{\ns \in \C{F}}
    \item \im{A \in \C{F} \Rightarrow A^c = \Omega\backslash A\in \C{F}}\mbox{(对取余封闭)}
    \item \im{A_i,i = 1, 2, \cdots, \in \C{F} \Rightarrow \bigcup_{i=1}^\infty A_i  \in \C{F}}\mbox{(对可列并封闭)}
\end{enumerate} 
\end{definition}

\remark let \C{F} be a \f on \om. Then\footnote[2]{necessary condition}:

\begin{enumerate}[(i)]
    \item \im{\ns, \Omega \in \C{F}}
    \item \im{A, B\in \C{F} \Rightarrow A\backslash B\in\C{F}}(对求差运算封闭)
    \item \im{A_i \in \C{F}, i=1, 2, \cdots \Rightarrow\bigcap_{i=1}^\infty A_i\in \C{F}}(对可列并封闭)
    \item \im{A_i \in \C{F}, i=1, 2, \cdots, n, \Rightarrow\bigcup_{i=1}^n A_i(\bigcap_{i=1}^n A_i)\in \C{F}}(对有限交,有限并封闭)
\end{enumerate}

Let \om be a non-empty set, we write:
\begin{enumerate}[(i)]
    \item  \im{\C{F}_0 = \{\ns, \Omega\}} (trivial \f: 平凡 $\sigma$ 域)
    \item \im{\C{F}_1 = \{\ns, \Omega, A, A^c\}, A\in \Omega} (the \f generated by A)
        \newline Denoted by \im{\C{F}_A = \sigma(A)}
        \newline This is also the smallest \f contains A
    \item \im{\C{F}_2 = 2^\Omega = \{ A:A\in \Omega\}} 
\end{enumerate}

The \im{\C{F}_0, \C{F}_1, \C{F}_2} are \f, then we have:
\begin{align*}
    \C{F}_0\in \C{F}_1 \in \C{F}_2
\end{align*} 

Let \om be a non-empty set, and \C{F} be a \f on \om. Then \im{(\Omega, \C{F})}
is called a measurable space and any set in \C{F} is called a measurable set.

The intersection of $\sigma$-fields are again a \f, but the union of $\sigma-fields$ 
may not be a \f. 

For example: Let $\Omega = \{1, 2, 3\}$, $\C{F}_1 = \{\ns, \{1\}, \{2, 3\}, \Omega\}$,
$\C{F}_1 = \{\ns, \{2\}, \{1, 3\}, \Omega\}$. Then both $\C{F} = \C{F}_1, \C{F}_2$ is \f, but
$\C{F}_1\cup\C{F}_2 = \{\ns, \{1\}, \{2, 3\}, \{2\}, \{1, 3\}, \Omega\}$  is not a \f:

for that , $A_1= \{1\}\in \C{F}, A_2=\{2\}\in \C{F}$, then $A_1\cup A_2=\{1\}\cup \{2\} = \{1, 2\}\notin \C{F}$, Which shows that $A_1, A_2\in \C{F}$, but $A_1\cup A_2\notin\C{F}$.
So \C{F} is not a \f.

Beorel \f is Denoted by \im{\C{B}(\mathbb{R})}\footnote[3]{又叫做实数集上的 $\sigma-$代数}.
So it's called Borel \f on $\mathbb{R}$, and each set in $\C{B}(\mathbb R) $ is called Borel Set
\footnote[4]{存在 $\mathbb R$的子集不是 Borel 集}.
These are serval equivalent definitions as follows:

\begin{itemize}
    \item The \f generated by intervals.
    \item The Smallest \f which contains all interval.
    \item $\sigma(\{-\infty, x]: x\in \mathbb R)$
    \item $\sigma(\{a, b]: a, b \in\mathbb R)$
\end{itemize}

\begin{definition}
    Let \im{(\Omega, \C{F})} be  a measurable space, a Probability measure P on \im{(\Omega, \C{F})}  
is a \textbf{set function}:
\begin{align}
    P: & \C{F} \longrightarrow [0, 1]\notag\\
       & A  \longrightarrow P(A) 
\end{align} 
\end{definition}

% \newpage

Such that:
\begin{enumerate}[(i)]
    \item (非负性) \im{P(A) \le 0}
    \item (规范性) \im{P(\Omega) = 1} 
    \item (可列可加性) \im{A_i \in \C{F}, i = 1, 2, \cdots}, and  \im{A_i\cap A_j=\ns(i\neq j)}
        \begin{align}
            P(\bigcup_{i=1}^\infty A_i) = \sum_{i=1}^{\infty}{P(A_i)}
        \end{align}
\end{enumerate} 

At this time, the triple \im{(\Omega, \C{F}, P)} is Call a \textbf{Probability space}.
Moreover, an event ia said to occur almost surely is short for \textbf{a.s.} if \im{P(A) = 1}


\remark Let \im{(\Omega, \C{F}, P)} is a Probability space:
\begin{enumerate}[(i)]
    \item \im{P(\ns) = 0}
    \item If \im{A_i \in \C{F}, i= 1, 2, \cdots, n}, and \im{A_i\cap A_j=\ns}, Then
        \begin{align*}
            P(\bigcup_{i=1}^n A_i) = \sum_{i=1}^{n}{P(A_i)} \notag\\ 
        \end{align*}
        \emp{Notice:} {In Particularly}, $P(A) + P(A^c) = 1.$
    \item \im{B\subset A \Rightarrow P(A-B) = P(A) - P(AB)}.
        for any sets \im{A_1, A_2, \cdots, A_n},
        \begin{align}
            P(\bigcup_{i=1}^n A_i) & = \sum_{i=1}^{n}{P(A_i)} - \sum_{1\le i<j\le n}{P(A_i A_j)} + \sum_{1\le i < j < k\le n}^{}{P(A_iA_jA_k)}\notag\\
             & - \cdots + (-1)^{n+1}P(A_1A_1\cdots A_n)
        \end{align}
    \item If \im{P(A_i) =1, i = 1, 2, \cdots}, Then
        \begin{align*}
            P(\bigcup_{i=1}^\infty A_i) = P(\bigcap_{i=1}^\infty A_i) = 1
        \end{align*}
    \item (从下连续性 \im{\nearrow}) If \im{A_i \in \C{F}, i = 1, 2, \cdots}, and \im{A_1\supset A_2\cdots\supset A_n}, then:
        \begin{align*}
            P(\bigcup_{i=1}^nA_i) = \lim_{n\to\infty}{P(A_n)}
        \end{align*}
    \item (从上连续性 \im{\searrow}) If \im{A_i \in \C{F}, i = 1, 2, \cdots}, and \im{A_1\subset A_2\cdots\subset A_n}, then:
        \begin{align*}
            P(\bigcap_{i=1}^nA_i) = \lim_{n\to\infty}{P(A_n)}
        \end{align*}
\end{enumerate}

\begin{definition}
    The {\bf upper limit set :} 
    \begin{align}
        \overline{\lim_{n\to\infty}}{A_n} & =\lim_{n\to\infty}{\mathrm{sup}A_n} = \bigcap_{N=1}^\infty{\bigcup_{n=N}^\infty{A_n}} \notag\\
                                          & = \{\omega\in \Omega: \omega\in A_n, \mbox{\rm occur Infty when }n>N_0\} 
    \end{align}
    The {\bf lower limit set :} 
    \begin{align}
        \mathop{ \underline{\lim}}\limits_{n\to\infty}{A_n} & =\lim_{n\to\infty}{\mathrm{sup}A_n} = \bigcup_{N=1}^\infty{\bigcap_{n=N}^\infty{A_n}} \notag\\
                              & = \{\omega\in \Omega, \exists N_0\in N, \mbox{\rm such that for any } n>N_0, \omega\in A_n\} \notag\\            
                    & = \{\omega\in \Omega: \omega\notin A_n, \mbox{\rm occur each when }n>N_0\} 
    \end{align}
    

    \im{\Longrightarrow \mathop{ \underline{\lim}}\limits_{n\to\infty}{A_n} \subset \overline{\lim_{n\to\infty}}{A_n}}
\end{definition}

\begin{lemma}
    Let \im{A_1, A_2, \cdots, A_n} be a sequence of Events such that \im{P(A_1) + P(A_2)+\cdots+P(A_n) < +\infty}, and Let 
    \im{B_n = A_n\cup A_{n+1}\cup\cdots}, Then
    \begin{align*}
        P(B_1\cap B_2\cdots) = 0
    \end{align*}
    Let \im{A_1, A_2, \cdots, A_n} be a sequence of independent Events such that \im{P(A_1) + P(A_2)+\cdots+P(A_n) = +\infty}, and Let 
    \im{B_n = A_n\cup A_{n+1}\cup\cdots}, Then
    \begin{align*}
        P(B_1\cap B_2\cdots) = 1
    \end{align*}
\end{lemma}

\begin{proof}
    Let \im{P(B_n, i.o.) }{ denotes }\im{ P(A_1\cap A_2\cdots)}, The first simple proof:
    \begin{align*}
        0 \le P(B_n, i.o.) \le P(\bigcap_{N=1}^\infty\bigcup_{n=N}^\infty A_n) 
        & = \lim_{n\to \infty}{P(\bigcup_{n=N}^\infty A_n)} \le \lim_{N\to \infty}{\sum_{n=N}^{\infty}{P(A_n)}} \notag\\
        & = \sum_{n=1}^{\infty}{P(A_n)} -\lim_{N\to\infty}{\sum_{n=1}^{N-1}{P(A_n)} }\notag\\
        & = \sum_{n=1}^{\infty}{P(A_n)} - \sum_{n=1}^{\infty}{P(A_n)}   \notag\\
        & = 0                           
    \end{align*} 
    
    
    The second difficult proof:
    \begin{align*}
        1 \ge P(B_n, i.o.) = P(\bigcap_{N=1}^\infty{\bigcup_{n=N}^\infty}A_n) 
        & = \lim_{N\to \infty}{P(\bigcup_{n=N}^\infty A_n)} = 1 - \lim_{N\to \infty}{P(\bigcap_{n=N}^\infty A_n^c)}\notag\\
        \mbox{(According to Independence)}
        \Rightarrow LHS & = 1 - \lim_{N\to \infty}{\prod_{n=N}^\infty{[1- P(A_n)]}} \notag \\
        & \ge 1 - \lim_{N\to\infty}{\prod_{n=N}^\infty{{e}^{-P(A_n)}}} \notag \\
        & = 1 - \lim_{N\to \infty}{e^{-\sum_{n=N}^{\infty}{P(A_n)}}}\notag\\
        & = 1 - 0\notag\\
        & = 1
    \end{align*}
\end{proof}