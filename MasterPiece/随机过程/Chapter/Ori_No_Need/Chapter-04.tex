\newcommand{\sg}{\ensuremath{\sigma}}


\section{Mind Map}

The Mind map of this Chapter is as follows
\begin{align}
    E(\xi|B) \lr E(\xi|\eta=y_i) \lr E(\xi|\eta) \lr E(\xi|\sg(\eta)) 
    \lr E(\xi|\C{G}) \notag
\end{align}

\remark For some sub \sg-field \C{G} $\in$ \C{F}

\s{2.1}{Conditioning on Events}

\begin{definition}
    For any $\xi \in \C{L}^1[\om, \C{F}, P]$ and $\A B\in\C{F}$ such that $P(B)>0$, the 
    Conditional Expectation of $\xi$ given by $B$ is defined by
    \begin{align}
        E(\xi|B) = \frac{E(\xi\F{1}_B)}{P(B)} = \frac{\dis\int_{B}^{}{\xi \mathrm{d}p}}{P(B)}\hspace*{3em}
        \lrr E(\xi\F{1}_B) = E(\xi|B)\cdot P(B)\notag
    \end{align}

    \remark\\
    (1) Taking $B=\om$ we have $\dis E(\xi|\om) = \frac{E(\xi\F{1}_\om)}{P(\om)} = E(\xi)$ \\
    (2) If $\xi(\omega) = \F{1}_A(\omega) = \left\{
    \begin{aligned}
        & 1, & \omega \in A\\
        & 0, & \omega \notin A
    \end{aligned}\right.$
    ,\quad then $\dis E(\F{1}_A|B) = \frac{E(\F{1}_A\F{1}_B)}{P(B)} = \frac{E(\F{1}_{A\cap B})}{P(B)} = \frac{P(A\cap B)}{P(B)} =P(A|B)$
\end{definition} 

\emp{Example 2.1}

Three coins, 10p, 20p and 50p are tossed. The values of those coins that land
heads up are added to work out the total amount $\xi$. What is the expected total
amount $\xi$ given that two coins have landed heads up?

\emp{Solution}

$\om = \{(HHH),(HHT), (HTH),(THH), (HTT), (THT), (TTH), (TTT)\}$, this is the sample space.
"H" $\lr$ 'Head', \qquad "T" $\lr$ 'Tail'

Then all possible values of $\xi$ are:

\noindent$\xi(HHH) = 80, \xi(HHT) = 30, \xi(HTH) = 60, 
\xi(THH) = 70, \xi(HTT) = 10, \xi(THT) = 20$\\
$\xi(TTH) = 50, \xi(TTT) = 0$

Write $B$ = \{All Coins have land Heads Up\}$ = \{(HHT), (HTH), (THH)\}$, We have
\begin{align*}
    E(\xi\F{1}_B) = \frac18\left(\xi(HHT)+\xi(HTH)+\xi(THH)\right) = \frac18 (30+60+70) = 20
\end{align*}

Thus, $\dis E(\xi|B) = \frac{E(\xi\F{1}_B)}{P(B)} = \frac{20}{\frac38} = \frac{160}{3}$

\newpage

\s{2.2}{Conditioning on a Discreate r.v.}

Let $\eta$ be a discreate r.v. with values $y_1, y_2, \cdots, y_i, \cdots$, the table as follows:

\begin{center}
    \begin{tabular}{c|ccccc}
        $\eta$ & $y_1$ & $y_2$ & $\cdots$ & $y_i$ & $\cdots$ \\
        \hline
        $P$ & $P(\eta = y_1)$ & $P(\eta = y_1)$ & $\cdots$ & $P(\eta = y_1)$ & $\cdots$ \\  
    \end{tabular}    
\end{center}


\emp{Notice:} $\dis E(\xi|\eta =y_k) = E(\xi |\{\eta=y_k\})$ 

\begin{definition}
    Let $\xi \in \C{L}^1(\om, \C{F}, P)$ and $\eta$ be a  discreate r.v. .Then
    the Conditioning of $\xi$ given $\eta$ is r.v. .$E(\xi|\eta)$  such that
    \begin{align}
        E(\xi|\eta)(\omega) = E(\xi|\eta=y_k)\ if\ \eta(\omega) = y_k
    \end{align}
    That's To say if $\eta$ is given, Then $\xi$ is \textbf{Only}
\end{definition}


\remark
\begin{center}
    \begin{tabular}{c|ccccc}
        $E(\xi|\eta)$ & $E(\xi|\eta=y_1)$ & $E(\xi|\eta=y_2)$ & $\cdots$ & $E(\xi|\eta=y_i)$ & $\cdots$ \\
        \hline
        $P$ & $P(\eta = y_1)$ & $P(\eta = y_1)$ & $\cdots$ & $P(\eta = y_1)$ & $\cdots$ \\  
    \end{tabular} 
\end{center}

\emp{Example 2.2}

Three coins 10p, 20p, 50p; $\xi=$ the total amount shown by the Three coins.
$\eta=$ the total amount shown by the 10p and 20p coins \textbf{Only}. Then Find out 
The $E(\xi|\eta)$\,?

\emp{Solution}

It's Clear that $\eta$ is a discreate r.v. with all possible values $0, 10, 20, 30$ and 

\vspace*{1em}
$
\begin{aligned}
    & \{\eta=0\} = \{(TTH), (TTT)\} \lrr P(\eta=0) = \frac14\\
    & \{\eta=0\} = \{(HTH), (HTT)\} \lrr P(\eta=10) = \frac14\\
    & \{\eta=0\} = \{(THH), (THT)\} \lrr P(\eta=20) = \frac14\\
    & \{\eta=0\} = \{(HHH), (HHT)\} \lrr P(\eta=30) = \frac14\\
\end{aligned}
$

\emp{Moreover}

\noindent$
\left\{
\begin{aligned}
    & E(\xi|\F{1}_{\{\eta=0\}}) = \frac{1}{8} \left(\xi(TTH)+\xi(TTT)\right)\\
    & E(\xi|\F{1}_{\{\eta=10\}}) = \frac{1}{8} \left(\xi(HTH)+\xi(HTT)\right)\\
    & E(\xi|\F{1}_{\{\eta=20\}}) = \frac{1}{8} \left(\xi(THH)+\xi(THT)\right)\\
    & E(\xi|\F{1}_{\{\eta=30\}}) = \frac{1}{8} \left(\xi(HHH)+\xi(HHT)\right)
\end{aligned}
\right.
\lrr
\left\{
\begin{aligned}
    & E(\xi|\eta=0) = \frac{E(\xi\F{1}_{\{\eta=0\}})}{P(\eta=0)} = \frac{\frac{50}{8}}{\frac{1}{4}} = 25\\
    & E(\xi|\eta=10) = \frac{E(\xi\F{1}_{\{\eta=10\}})}{P(\eta=10)} = \frac{\frac{70}{3}}{\frac{1}{4}} = 35\\
    & E(\xi|\eta=20) = \frac{E(\xi\F{1}_{\{\eta=20\}})}{P(\eta=20)} = \frac{\frac{90}{8}}{\frac{1}{4}} = 45\\
    & E(\xi|\eta=30) = \frac{E(\xi\F{1}_{\{\eta=30\}})}{P(\eta=30)} = \frac{\frac{110}{8}}{\frac{1}{4}} = 55
\end{aligned}
\right.
$

\emp{FurtherMore}
\begin{center}
    \begin{tabular}{p{6em}|p{6em}p{6em}p{6em}p{6em}}
        $E(\xi|\eta)$ & 25 & 35 & 45 & 55 \\
        \hline
        $P$ & $\frac14$ & $\frac14$ & $\frac14$ & $\frac14$\\  
    \end{tabular} 
\end{center}

\newpage
From the Table before, We can conclude a general Formula
\begin{align}
    E(E(\xi|\eta)) = E(\xi)
\end{align}

\emp{Example 2.3}

$\C{L}[\om, \C{F}, P]$ be a Probability space, where $\om = [0, 1],\ \C{F} = \C{B}([0, 1])$,
$P = $ Lebesgue measure on $[0, 1]$, Define two r.v.'s on $om$ as follows:
\begin{align*}
    &  \xi(x) = 2x^2, x \in [0, 1]
    & \eta(x) = 
    \left\{
        \begin{aligned}
            & 1, x\in [0, \frac{1}{3}]\\
            & 2, x\in (\frac13, \frac23]\\
            & 0, x\in (\frac23, 1]
        \end{aligned}
    \right.
\end{align*}

\emp{Solution}\\
(1) if $x\in [0, \frac13]$, then $\dis E(\xi|\eta)(x) = E(\xi|\eta=1) = \frac{E(\xi\F{1}_{\{\eta=1\}})}{P(\eta=1)} = \frac{\int_{0}^{\frac13}{2x^2 \mathrm{d}x}}{\frac13} = \frac{2}{27}$\\
(2) if $x\in (\frac13, \frac23]$, then $\dis E(\xi|\eta)(x) = E(\xi|\eta=2) = \frac{E(\xi\F{1}_{\{\eta=2\}})}{P(\eta=2)} = \frac{\int_{0}^{\frac13}{2x^2 \mathrm{d}x}}{\frac23-\frac13} = \frac{14}{27}$\\
(3) if $x\in (\frac23, 1]$, then $\dis E(\xi|\eta)(x) = E(\xi|\eta=0) = \frac{E(\xi\F{1}_{\{\eta=0\}})}{P(\eta=0)} = \frac{\int_{0}^{\frac13}{2x^2 \mathrm{d}x}}{1-\frac23} = \frac{38}{27}$

\emp{Moreover}
\begin{center}
    \begin{tabular}{p{6em}|p{6em}p{6em}p{6em}}
        $E(\xi|\eta)$ & $\frac{2}{27}$ & $\frac{14}{27}$ & $\frac{38}{27}$ \\
        \hline
        $P$ & $\frac13$ & $\frac13$ & $\frac13$\\  
    \end{tabular} 
\end{center}

\emp{Some property}

\begin{enumerate}[(i)]
    \item $E(a\xi+b\eta|\zeta) = aE(\xi|\zeta)+bE(\eta|\zeta)$
    \item If $\eta = \F{C(onstant)}$, then $E(\xi|\eta) = E(\xi)$
    \item $\xi$ and $\eta$ are independent $\lrr E(\xi|\eta) = E(\xi)$
    \item $E(\F{1}_A\F{1}_B)(\omega) = 
        \left\{
            \begin{aligned}
                & P(A|B),&\omega\in B\\
                & P(A|B^c), &\omega \notin B
            \end{aligned}
        \right.
    $
    \item $\dis E(E(\xi|\eta)) = E(\xi)$
\end{enumerate}

\emp{Proposition 2.1}

If $\xi \in \C{L}^1[\om, \C{F}, P]$ and $\eta$ be a discreate r.v., then
\begin{enumerate}[(i)]
    \item $E(\xi|\eta)$ is $\sg(\eta)$-measurable $\lrr E(\xi|\eta)$ {\kaishu 的取值由} $\eta$ {\kaishu 所唯一确定} 
    \item For $\dis \A A \in \sg(\eta), \quad \int_{A}^{}{E(\xi|\eta) \mathrm{d}p} = \int_{A}^{}{\xi \mathrm{d}p}$
\end{enumerate}

\begin{proof}
    suppose that $\eta$ has {\bf countablly many}  pairwise disjoint values $y_1, y_2, \cdots$

    Then $\{\eta = y_i\}\cap\{\eta=y_j\} = \ns\,(i\neq j)$ and $\dis \bigcup_{i=1}^{+\infty}\{\eta=y_i\} = \om$, if $\eta(\omega) = y_k$, Then, 

    $E(\xi|\eta)(\omega) = E(\xi|\eta=y_k)$, which means that $E(\xi|\eta)$ is $\sg(\eta)$-measurable
\end{proof}

\emp{Moreover}

For each $i$, we have
\begin{align*}
    \int_{\{\eta=y_i\}}^{}{E(\xi|\eta) \mathrm{d}p} 
    & = \int_{\{\eta=y_i\}}^{}{E(\xi|\eta=y_i) \mathrm{d}p} \\
    & = E(\xi|\eta=y_i)\cdot P(\eta=y_i) = E(\xi|\F{1}_{\{\eta=y_i\}}) \\
    & = \int_{\{\eta=y_i\}}^{}{\xi \mathrm{d}p}
\end{align*}

\emp{Notice:} $\dis E(\xi|\F{1}_A) = \int_{A}^{}{\xi \mathrm{d}p}$

For general, $A\in \sg(\eta)$, Write $A = \bigcup_{k=1}^{+\infty}{\{\eta=y_{i_k}\}}$, Then 
\begin{align}
    \int_{A}^{}{E(\xi|\eta) \mathrm{d}p} 
    & = \int_{\bigcup\limits_{k=1}^{+\infty}{\{\eta=y_{i_k}\}}}^{}{E(\xi|\eta) \mathrm{d}p} = \sum_{k=1}^{+\infty}{\int_{\bigcup\limits_{k=1}^{+\infty}{\{\eta=y_{i_k}\}}}^{}{E(\xi|\eta) \mathrm{d}}p}\notag \\
    & = \sum_{k=1}^{+\infty}{\int_{\bigcup\limits_{k=1}^{+\infty}{\{\eta=y_{i_k}\}}}^{}{\xi \mathrm{d}}p} = \int_{A}^{}{\xi \mathrm{d}p}
\end{align}

\emp{Notice:} If $A\cap B=\ns$, then $\dis \int_{A\cup B}^{}{\xi \mathrm{d}p} = \int_{A}^{}{\xi \mathrm{d}}p + \int_{B}^{}{\xi \mathrm{d}p}$

\newpage
\s{2.3}{Conditioning on an Arbitrary r.v.\quad($E(\xi|\eta)$)}

\begin{definition}
    Let $\xi \in \C{L}^1[\om, \C{F}, P]$ and $\eta$ be an Arbitrary r.v. . Then the Expectation 
    of $\xi$ given $\eta$ is defined to be a r.v. $E(\xi\eta)$ such that
    \begin{enumerate}[(i)]
        \item $E(\xi|\eta)$ is $\sg(\eta)$-measurable
        \item For $\dis \A A \in \sg(\eta), \int_{A}^{}{E(\xi|\eta) \mathrm{d}}p = \int_{A}^{}{\xi \mathrm{d}p}$.
        i.e., $E(\F{1}_AE(\xi|\eta)) = E(\F{1}_A\xi)$
    \end{enumerate}

    \remark \\
    (1) $E(\F{1}_A|\eta) = P(A|\eta)$.\\
    (2) $\xi=\xi'$ a.s. $\lrr E(\xi|\eta) = E(\xi'|\eta)$ a.s.
\end{definition}

\begin{lemma}
    Let $(\om, \C{F}, P)$ be a Probability space and $\C{G} \in \C{F}$ be a sub $\sg$-field.
    If $\xi$ is a \C{G}-measurable r.v. and for any $B \in \C{F}$
    \begin{align*}
        \int_{B}^{}{\xi \mathrm{d}p} = 0
    \end{align*}  
    Then $\xi = 0$ a.s. 
\end{lemma}

\remark If $\xi$ and $\eta$ are two \C{G}-measurable r.v.'s and for any $B\in \C{G}$
\begin{align*}
    \int_{B}^{}{\xi \mathrm{d}p} = \int_{B}^{}{\eta \mathrm{d}p}
\end{align*}
then $\xi = \eta$ a.s.

\emp{Notice:} In fact, $\dis \int_{B}^{}{(\xi-\eta)}{ \mathrm{d}p} = 0,\quad \lrr$\quad we have $\xi - \eta = 0$ a.s.


