
\newcommand{{\pr}}[1]{{\bf Problem #1}}
\newcommand{\zs}[2]{\ensuremath{{#1}^{#2}}}
\newcommand{\e}[1]{\ensuremath{\exp\{{#1}}}
\newcommand{\fs}[2]{\ensuremath{\frac{#1}{#2}}}
\newcommand{\dfs}[2]{ensuremath{\frac{\mathrm{d}#1}{\mathrm{d}#2}}}
\newcommand{\pfs}[2]{\ensuremath{\frac{\partial #1}{\partial #2}}}
\newcommand{\abs}[1]{\ensuremath{|#1|}}
\newcommand{\dpb}[1]{\ensuremath{\displaystyle{#1}}}
\newcommand{\imd}[2][1]{%
\lower-.2em\hbox{
    \scalebox{#1}[#1]{%
    \ensuremath{\displaystyle{#2}}%
}}}
% \newcommand{\im}[1]{\ensuremath{\displaystyle{#1}}}
% \newtheorem{theorem}{Theorem}
% \newtheorem{lemma}{Lemma}
% \newtheorem{proof}{Proof}
% \newtheorem{definition}{Definition}
\newtheorem{proposition}{Proposition}
\newtheorem{example}{Example}
% \newtheorem{remark}{Remark}
\newenvironment{pro}[1]{{\bf Problem #1}}{\par}

\sector{Process of Poisson and Brownian}

\begin{Definition}
    Let $\{\xi (t),t\ge 0 \}$ be a stochastic process defined on a filtered probability space
    $(\Omega,\mathcal{F} ,\{\mathcal{F} \}_{t\ge 0 },P)$.If
    \begin{enumerate}
        \item for $\forall t \ge 0,\xi (t) $ is $\mathcal{F}_t$-measure
        \item for $\forall t\ge 0$,E\abs{\xi(t)} < $+\infty$
        \item for $\forall 0\le s \le t,E(\xi(t) | \mathcal{F}_s ) = \xi(s)$a.s.
    \end{enumerate}
    then $\{\xi(t)\} $ is called an $\{\mathcal{F}\}$-martingale.   
    
    {\bf Remark:}对于上鞅和下鞅的情形,只需要将 $\mathit{3}$中的 $'='$分别改为$'<'$和$'>'$即可.
\end{Definition}

{\bf Recall}
    {\it Let $\lambda$ > 0.A r.v. $\xi\sim P(\lambda)$ if
    \[P(\xi =k) = \frac{\lambda^k}{k!}\exp(-\lambda k) \]}

\begin{Definition}
    A Poisson process $\{N(t),t\ge 0\}$ is a stochastic process with the following properties:
    \begin{enumerate}
        \item N(0) = 0
        \item (Independent increments\footnotemark[1]) For $\forall\; 0\le s \le t,N(t) - N(s)$ is independent of $\mathcal{F}_s = \sigma (N(u),0\le u \le s)$
        也就是说,从$s$时刻到$t$时刻的增量与s时刻之前的信息无关.
        \item (stationary increments 平稳增量) For $\forall\; 0 \le s \le t,N(t)-N(s)\sim P(\lambda(t-s))$,i.e.

            \[P(N(t)-N(s) = n)= \frac{(\lambda (t-s))^n}{n!}\mathrm{e}^{-\lambda(t-s)} \]
        \item (Step function paths)The path N(t) ,$t\ge 0$ are increasing function of t changing only by jumps of size 1.
        $N(t)-t$的图像是一个阶梯函数,相邻两段直接跳跃高度是1.
    \end{enumerate}    
\end{Definition}

其中平稳增量\footnote[1]{Independent increments:独立增量; stationary increments:平稳增量},稳定增量这两个性质用的很多,一定要注意。
% \footnote[\vphantom{1}]{Independent increments:独立增量; stationary increments:平稳增量}

\begin{theorem}
    $\{x_n,n\ge 1 \}$ are i.i.d with the distribution \mbox{\rm Exp}($\lambda $).And $T_n \sim \Gamma  (n,\lambda) $    
\end{theorem}
\begin{proof}
    It's obvious.
\end{proof}

\clearpage
{\bf Some Fact}:\par
1. Let $x_i$ is the waitting time from $N(i-1)$ to $N(i)$,we have $T_n = \sum_{i=1}^n x_i$\par
2. $N(t) = \sup\{n\ge 0:T_n\le t\}$

{\bf Note}\par
\begin{enumerate}
    \item "independent increments" mean that for $\forall n \ge 1$ and $0=t_0<t_1<\ldots<t_n$,the r.v.
    $\{N(t_k) - N(t_{k-1}),k=1,2,\cdots,n \}$are independent.
    \item  By defintion--(3), Let $s=0$),for $\forall t\ge 0$,we have
    \[P(N(t) = k) = \frac{(\lambda t)^k}{k!} \mathrm{e}^{-\lambda t} \]
    \item Two Formulas, by which will be used to prove.  
        \begin{align*}
            & W(t+s) - W(t)\sim N(0,s) \\
            & {\rm If\;} \xi \sim N(\mu, \sigma^2), {\rm then\;} e^{\theta\cdot\xi} = e^{\mu\cdot\theta+\frac12\sigma^2\theta^2}
        \end{align*}
\end{enumerate}    





\clearpage
\subsection*{Exercise}{\it
Let $\{N(t),t\ge 0 \}$ be a Poisson process with parameter $\lambda t$.Take natural filtered.
\begin{enumerate}
    \item[(i)] Find $P\{N(1)=1,N(2)=2\},P(N(2)=2|N(1)=1),\\
        \hspace*{2em} E[N(t)\cdot N(t+s)],Cov(N(t),N(s)),{\imd[1.2]{\rho}_{N(t),N(s)}}$.
    \item[(ii)] Show that $E[N(t)-\lambda t,t\ge 0]$ is a $ \{ \mathcal{F}_t|t>0 \}$-Martingale.
\end{enumerate}
}


% $\bullet $ 1 Hello
\subsection*{Solution}{\it
%  \begin{align*}
%    \bullet 1 \hspace*{2em} P\{N(1)=1,N(2)=2\} &= P(N(1) = 1,N(2)-N(1) = 1)\\
%     &=P(N(1) = 1)P(N(2)-N(1) = 1) \\
%     &= (\frac{\lambda}{1!} e^{-\lambda })^2 = \lambda ^2 e^{-2\lambda} 
% \end{align*}

% $\big |_{i=1}$

% $\left(Hell\mathop{o}\limits^{\mathop{\bigg|}\limits^{+\infty}}\right)$

% 常见的 mathop $\sum , \prod$

\begin{enumerate}
    \item we have
        \begin{align*}
            &\begin{aligned}
                \bullet\; P\{N(1)=1,N(2)=2\} &= P(N(1) = 1,N(2)-N(1) = 1)\\
                &=P(N(1) = 1)P(N(2)-N(1) = 1) \\
                &= (\frac{\lambda}{1!} e^{-\lambda })^2 = \lambda ^2 e^{-2\lambda}  
            \end{aligned}\\[1.2em]
            &\begin{aligned}
                \bullet\;P\{N(2) = 2|N(1) = 1\} = \frac{P(N(1) =1 ,N(2) =2)}{P(N(1) = 1)} = \lambda e^{-\lambda}
            \end{aligned}\\[1.2em]
            &\begin{aligned}
                \bullet\;E[N(t)\cdot N(t+s)] &= E[N(t)\cdot (N(t+s)-N(t)+N(t))]\\
                &=E[N(t)\cdot (N(t+s)-N(t))] + E[N(t)^2]\\
                &=E[N(t)]\cdot E[N(t+s)-N(t)] + E[N(t)^2]\\
                &=\lambda t\lambda s + \lambda t+(\lambda t)^2\\
                &=\lambda t(\lambda t + \lambda s + 1)
            \end{aligned}\\[1.2em]
            &\begin{aligned} 
                \bullet\;\mbox{If } s<t,then:\mathrm{Cov}(N(t),N(s)) &= \mathrm{Cov}(N(t)-N(s)+N(s),N(s))\\
                &=\mathrm{Cov}(N(t)-N(s)+N(s),N(s))\\
                &=\mathrm{Cov}(N(t)-N(s),N(s))+\mathrm{Cov}(N(s),N(s))\\
                &=0 + \mathrm{Var}(N(s))\\
                &= \lambda s
            \end{aligned}\\
            &\mbox{so then } \mathrm{Cov}(N(t),N(s)) = \lambda \min\{s,t \}\\
            &\begin{aligned}
                \bullet{}\;\rho_{\scriptscriptstyle N(t), N(s)} 
                    = \frac{{\rm Cov}(N(t), N(s))}{\sqrt{{\rm Var}(N(t))\cdot {\rm Var}(N(s))}}
                    = \frac{\min\{t, s\}}{\sqrt{s\cdot t}} 
            \end{aligned}
        \end{align*}
    \item we have
        \begin{align*}
            &\begin{aligned}
                \bullet\; \mbox{For } \forall t\ge 0 , N(t) - \lambda t \mbox{ is } \mathcal{F}_t\mbox{-measure}.    
            \end{aligned}\\[1.2em]
            &\begin{aligned}
                \bullet\; \mbox{For }\forall t \ge 0,E\abs{N(t)-\lambda t}\le E[N(t)] + \lambda t  = 2\lambda t < \infty
            \end{aligned}\\[1.2em]
            &\begin{aligned}
                \bullet\; \mbox{For } \forall 0 \le s \le t,\mbox{ we have }\\
                E[N(t)-\lambda t|\mathcal{F}_s] 
                &= E[N(t)|\mathcal{F}_s]  - \lambda t \\
                &= E[N(t)-N(s)+N(s)|\mathcal{F}_s] -\lambda t\\
                &= E[N(t)-N(s)|\mathcal{F}_s] + N(s) - \lambda t\\
                &= \lambda (t-s) + N(s) - \lambda t\\
                &= N(s) - \lambda s
            \end{aligned}\\[1.2em]
        \end{align*}     
\end{enumerate}
}

\begin{Definition}
    A stochastic process $\{W(t),t\ge 0\}$ is called a Brownian motion or Wiener process if 
    \begin{enumerate}
        \item W(0) = 0
        \item (Independent increments) For $\forall\; 0\le s \le t,W(t) - W(s)$ is independent of $\mathcal{F}_s = \sigma (W(u),0\le u \le s)$.
        \item (stationary increments) For $\forall\; 0 \le s \le t,W(t)-W(s)\sim N(0,t-s)$
        \item (continuous sample paths) the sample paths t $\rightarrow$ W(t) are a.s. continuous
    \end{enumerate}
\end{Definition}

{\bf Remarks}
\begin{enumerate}
    \item A Brownian motion $\{W(t),t\ge 0\}$ is a Markov process.In fact,
    \begin{align*}
        E[\mathrm{e}^{\theta W(t+s)} | \mathcal{F}_t] &=E[\mathrm{e} ^{\theta(W(t+s) - W(t) + W(t))} | \mathcal{F}_t]\\
        &=E[\mathrm{e} ^{\theta(W(t+s) - W(t) )} ]\mathrm{e}^{\theta W(t)}\footnotemark[1]\\
        &=\mathrm{e} ^{\frac{1}{2}\theta ^2 s + \theta W(t)}\\
        &=E[\mathrm{e}^{\theta W(t+s)}|W(t)]
    \end{align*} 
    \item By(3.stationary increments),for $\forall t \ge 0$ ,we have $W(t)\sim N(0,t) $, and the following properties:
    \begin{framed}
        \hspace*{3em}
        $\begin{aligned}
            &E(W(t)) = 0,&  &E|W(t)| = \sqrt{\frac{2t}{\pi}}\\
            &E[W^2(t)] = Var(W(t)) = t,& &E[W^4(t)] = 3t^2\\
            &{\rm Cov}(W(t),W(s)) = \min \{s,t\},& &\rho_{\scriptscriptstyle W(t),W(s)} = \frac{\min\{s,t\}}{\sqrt{st}}
        \end{aligned}$
    \end{framed}
    \item For $\forall\; 0\le s < t$,we have:
    \[\alpha W(s) + \beta W(t) = (\alpha + \beta)W(s) + \beta (W(t) - W(s)) \sim N(0,(\alpha+\beta)^2s+\beta^2 (t-s)) \]
    因为$W(s) \sim N(0,s),W(t)-W(s) \sim N(0,t-s)$,
    且 $f(x)=\frac{1}{\sqrt{2\pi}\sigma}\mathrm{Exp}\{\frac{\left(x-\mu\right)^2}{2\sigma^2}\}\bigg|\mu=0$
    \item 任意两个 Brownian motion联合 \lrr 二维正态分布(但是两个一般的正态分布就不能满足)
\end{enumerate}



\noindent{\bf Example 1}\par
W(1) + W(2) = 2W(1) + W(2) - W(1)  $\sim $N(0,5)\par
Thus $P(W(1) + W(2) \le 1) =\varPhi (\frac{1}{\sqrt{5}}) $ 

\noindent{\bf Example 2}\par
W(1) + W(2) + W(3) + W(4) = 4W(1) + 3(W(2)-W(1)) + 2(W(3)-W(2)) +W(4) - W(3) $\sim$ N(0,30)


\begin{Definition}
    A process is called Gaussian if all its finite-dimensional distribution
are multivariate normal.
\end{Definition}

\begin{theorem}
    A stochastic process $\{W(t),t\ge 0 \}$ is a Brownian motion $\Longleftrightarrow  \{W(t),t\ge 0 \}$ is a Gaussian process
with $E[W(t)]$ = 0 and {\rm Cov}$(W(t),W(s))$ = $\min \{s,t\}$.
\end{theorem}

{\bf Exercise}:Let $\{W(t),W(s)   \}$be a B.M.Write $Z(t) = \frac{1}{c} W(c^2t)$,where c > 0 is a constant.Is $Z(t)$ a B.M.?

{\bf Solution}:Yes.It's obvious that $\{Z(t),t\ge 0 \}$ is a Gaussian process.Moreover,
\begin{itemize}
    \item $ E[Z(t)] = \frac{1}{c} E(W(c^2 t))$ = 0
    \item $ {\rm Cov}(Z(t),Z(s)) = \frac{1}{c^2}{\rm Cov}(W(c^2s),W(c^2t)) = \min\{t,s\}$ 
\end{itemize}
\subsection*{Sample paths of B.M.}
Almost every sample path $W(T),0\le t \le T$ is:
\begin{enumerate}
    \item a continuous function of t 
    \item not monotone in any interval,not differentiable at any point.
    \item not differentiable at any point
    \item Has infinite variation (无穷变差) on any interval,no matter how shall it is.
    \item \mbox{[W,W](t) = t}.
\end{enumerate}
{\bf Theorem} Let $\{W(t),t\ge 0\}$ be a B.M.and $\mathcal{F}_t$ is natural filtered.Then the following process are all martingale:
\begin{enumerate}
    \item W(t).
    \item $W(t)^2 - t$.
    \item Z(t) := $\mathrm{e} ^{\theta W(t) - \frac{1}{2}\theta^2 t }$,where $\theta$ is a constant
\end{enumerate}
{\bf Proof}:(1)It's Obvious.
(2).For $\forall t \ge 0,E|W(t)| = $
(3).For $\forall $
\begin{enumerate}
    \item \begin{enumerate}
        \item For $\forall t \ge 0,W(t)$ is $\mathcal{F}_t$-measure.
        \item For $\forall t \ge 0,E\abs{W(t)}<\sqrt{\frac{2t}{\pi}} < \infty$
        \item For $\forall 0 \le s \le t$ ,
            \begin{align*}
                E[W(t)\big|\mathcal{F}_s] 
                &= E[W(t)-W(s)+W(s)\big|\mathcal{F}_s]\\
                &= E[W(t)-W(s)\big|\mathcal{F}_s] + W(s)\\
                &= W(s)\mbox{ a.s.}
            \end{align*}
    \end{enumerate}
    \item \begin{enumerate}
        \item For $\forall t \ge 0,W(t)^2 - t$ is $\mathcal{F}_t$-measure.
        \item For $\forall t \ge 0,E|W(t)^2-t| < 2t < \infty $
        \item For $\forall 0\le s < t$,we have
        \begin{align*}
            E[W(t)^2 - t\big|\mathcal{F}_s]&= E[(W(t)-W(s)+W(s))^ 2\big|\mathcal{F}_s ] - t\\
            &=E[(W(t)-W(s))^2\big|\mathcal{F}_s]+ 2W(s)E[W(t)-W(s)\big|\mathcal{F}_s] + W(s)^2 - t\\
            &=t-s+0+W(s)^2-t = W(s)^2 - s\mbox{ a.s.}
        \end{align*}
    \end{enumerate}
    \item \begin{enumerate}
        \item For $\forall t \ge 0 , Z(t)$ is $\mathcal{F}_t$-measure.
        \item For $\forall t \ge 0,E|Z(t)| = E[e^{\theta W(t)}]\cdot e^{-\frac{1}{2}\theta^2 t } = 1 < \infty$
        \item For $\forall 0\le s < t$,we have
        \begin{align*}
            E[Z(t)\big|\mathcal{F}_s] &= E[e^{\theta W(t)}\big|\mathcal{F}_s]\cdot e^{-\frac{1}{2}\theta^2t}\\
            &= E[e^{\theta (W(t) - W(s))}]\cdot e^{\theta W(s) - \frac{1}{2}\theta ^2 t}\\
            &= e^{\frac{1}{2}\theta^2(t-s)+\theta W(s) - \frac{1}{2}\theta^2 t}\\
            &= e^{\theta W(s) - \frac{1}{2} \theta ^2s} = Z(s)
        \end{align*}   
    \end{enumerate}
\end{enumerate}

% \newcommand{\C}[1]{\ensuremath{\mathcal{#1}}}

% \C{F}



