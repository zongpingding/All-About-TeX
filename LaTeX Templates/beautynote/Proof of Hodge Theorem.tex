\documentclass[twoside,openany,12pt]{beautynote} 
% Input Some Information of the doc
\doctitle{Complex Geometry}
\docsubtitle{The Proof of Hodge Theorem}
\dockeywords{Hodge theorem, Proof}

% Hyperref always required second to last.
\RequirePackage{hyperref}
\makeatletter
\hypersetup{%
    % hidelinks,
    pdfstartview=Fit,%
    pdfmenubar=true,%
    pdftoolbar=true,%
    bookmarksopen=false,%
    colorlinks=true,
    linkcolor=black,
    citecolor=purple,
    pdftitle={\@docsubtitle},%
    pdfauthor={\@author},%
    pdfsubject={\@doctitle},%
    pdflang={\languagename},%
    pdfkeywords={\@dockeywords},%
    pdfproducer={pdfTeX}}
\makeatother

% Cleveref as the last one.
\RequirePackage{cleveref}
%%%%%%%%%%%%%%%%%
\author{Ethan Lu}
\footext{}
\copyrightpage%
{Faculty of Pure Mathematics}% Your Faculty
{XX University}% Your University
{Press of XX University}% Your Publisher
{01A75, 00B50}% AMS
{XX}% Your City
% If you do not want to fill one of the fields, please leave it like this: {}
\usepackage{appendix}
%%%%%%%%%% Start TeXmacs macros
\newcommand\embold[1]{\bf #1}
\newcommand\slanted[1]{\itshape #1}
\newcommand{\assign}{:=}
\newcommand{\comma}{{,}}
\newcommand{\mathd}{\mathrm{d}}
\newcommand{\tmcolor}[2]{{\color{#1}{#2}}}
\newcommand{\tmop}[1]{\ensuremath{\operatorname{#1}}}
\newcommand{\tmverbatim}[1]{\text{{\ttfamily{#1}}}}
%%%%%%%%%% End TeXmacs macros
\begin{document}
% Titlepage
\maketitle\clearpage
%%%%%%%%%%%%%%%% Copyright-Page %%%%%%%%%%%%%%%%%%%%%%%
\copyrights
\pagestyle{\auxsettings}
\makeatletter
\thispagestyle{copyright}
\ifdefempty{\@faculty}{}{\noindent{\large\textsc{\@faculty}} \\}
\ifdefempty{\@university}{}{{\large\textsc{\@university}} \\[1em]}
\ifdefempty{\@publisher}{}{\textit{Published by:} \@publisher \\}
\ifthenelse{\boolean{copyright}}{\textit{Copyright by:} \textsc{\templateauthor }\\}{} 
\ifdefempty{\@ams}{}{\textit{AMS Classification (2023):} \@ams.\\}
\vfill
\ifdefempty{\@city}{}{\noindent\@city, on \today\\}
\copyright\,\the\year\, \textsc{The Authors}
\doclicenseThis
\cleardoublepage
\makeatother
%%%%%%%%%%%%%%%% Copyright-Page %%%%%%%%%%%%%%%%%%%%%%%
% Toc
    \tableofcontents
% Main Contents
\pagestyle{\defaultsettings}
\chapter{The proof of Hodge decomposition}
\section{Proof of Hodge decomposition}

\subsection{Proof of the Hodge Theorem 1: Local Theory}

Hilbert space provides an effective representation for Fourier series
and Fourier transform based on polynomial representation on any orthogonal
system, which is also one of the core concepts of functional analysis.

We are looking for the element of smallest norm in the affine subspace $\psi +
\overline{\partial} A^{p, q - 1} (M) \subset A^{p, q} (M)$. Clearly such an
element can be found in the closure of $\psi + \overline{\partial} A^{p, q -
1} (M)$ in the completion $\mathcal{L}^{p, q} (M) $ of the pre-Hilbert space
$A^{p, q} (M)$, simply by orthogonal projection. The problem is to show that
the element found in this way in fact lies in $A^{p, q} (M)$.

\begin{lemma}
  There is an inclusion
  \[ C^s (T) \subset H_s . \]
\end{lemma}

\begin{proof}[Using Induction Method]
  Firstly we show that $C^0 (T) \subset H_0  (s = 0)$, then through
  $(D_{\alpha} \varphi)_{\xi} = \xi^{\alpha} \varphi_{\xi}$,i.e. $\|D^{\alpha}
  \varphi \|_0^2 = \sum_{\xi} | \xi^{\alpha} |^2 | \varphi_{\xi} |^2$, it
  suffice to show the conclusion.
\end{proof}

Some remarks:

\begin{align*}
  D^{\alpha} e^{- i < \xi, x >} & = \overline{D_1^{\alpha_1} \cdots
  D_n^{\alpha_n} e^{i \sum_{j = 1}^n \xi_j^{\alpha_j} x_j}}\\
  & = \overline{ \frac{1}{i^n} \cdot \frac{\partial^n}{\partial  x_1 \cdots
  \partial x_n} \left( e^{i \sum_{j = 1}^n \xi_j^{\alpha_j} x_j} \right)}\\
  & = (\xi_{1 }^{\alpha_1} \cdots \xi_n^{\alpha_n}) \overline{e^{i \sum_{j =
  1}^n \xi_j^{\alpha_j} x_j}}\\
  & = \xi^{\alpha} e^{- i < \xi, x >} .
\end{align*}

\begin{lemma}[Soblev Lemma]
  
  \[ {\color[HTML]{800080}H_{s + [n / 2] + 1} \subset C^s T}, \]
  that is , every $u \in H_{s + [n / 2] + 1}$ is the Fourier series of a
  function $\varphi \in C^s T$, and this series converges uniformly to
  $\varphi$.
\end{lemma}

\begin{proof}
  First,consider the case $s = 0$; let
  \[ u = \sum_{\xi \in \mathbb{Z}^n} u_{\xi} e^{i \langle \xi, x \rangle}
     \tmcolor{brown}{\in H_{[n / 2] + 1}} \]
  with
  \[  \sum_{\xi} (1 +\| \xi \|^2 )^{[n / 2] + 1} | u_{\xi} |^2 < \infty . \]
  
  
  Observe that the partial sums
  \[ S_R (x) = \sum_{\| \xi \| \leqslant R} u_{\xi} e^{i < \xi, x >} \]
  are continous, and for ${R \leqslant R'}^{}$, one has
  \[  | S_R (x) - S_{R'} (x) | \leqslant \|u\|_{[n / 2] + 1} \sum_{\| \xi \|
     \neq 0} \frac{1}{\| \xi \|^{n + 1}}, \]
  where $\sum_{\| \xi \| \neq 0} \frac{1}{\| \xi \|^{n + 1}}$ is a convergent
  positive series. From which follows that $S_R (x)$ converges uniformly to
  $\varphi \in C^0 (T)$ with $\varphi_{\xi} = u_{\xi}$. (Here $S_R (x)$ is a
  Cauchy sequnce.)
  
  Now we proeed by induction on $s$. Since the proof for general $n$ involves
  only inessential formalism beyond what we have just done together with the
  one-variable case, we shall complete the argument only when $n = 1$.
  
  So by induction, we suppose $H_{s + 1} \subset C^s (T)$ (Induction
  hypothesis) and
  \[ u = \sum_{\xi \in \mathbb{Z}} u_{\xi} e^{i \xi x} \]
  (When $n = 1$, $\xi = (\xi), x = (x)$, then $< \xi, x > = \xi x$.)
  
  satisfies $u \in H_{s + 2} \subset H_{s + 1}$, i.e.
  \[  \sum_{\xi} | \xi |^{2 s + 4} | u_{\xi} |^2 < \infty . \]
  Set
  \[ v = \sum_{\xi \neq 0} i \xi u_{\xi} e^{i \xi x}  (= u' ?) . \]
  Then $v \in H_{s + 1}$, and therefore is a function in $C^s (T)$ by
  induction hypothesis. And the convergence being uniform, we may integrate
  term-by-term:
  \[ C^{s + 1} (T) \ni \int_0^x v (t) \mathd t = \sum_{\xi} u_{\xi} e^{i \xi
     x} = u (x) - u_0, \]
  so $u' (x) = v (x)$ and $u \in C^{s + 1} (T)$. (From $v \in H_{s + 1}
  \subset C^s (T) \Longrightarrow u \in H_{s + 2} \subset C^{s + 1} (T)$, just
  need to prove $u' (x) = v (x) .$)
  
  
\end{proof}

\begin{lemma}[Rellich Lemma]
  For $s > r$, the inclusion
  \[ H_s \subset H_r \]
  is compact.
\end{lemma}

\begin{proof}
  Given a bounded sequnce $\{ u_k \}$ in $H_s$, we want to find a convergent
  sequence in $H_r$.
  
  Firstly, the lemma is equivalent to \tmcolor{red}{the limit of every Cauchy
  sequence of $H_r$ is in $H_r$}. Thus, we can give a bounded sequence $\{ u_k
  \} $in $H_s$, then one can find a Cauchy sequence induced by $\{ u_k \}$ in
  $H_r$.
  
  Since $\{ u_k \} \subset H_s$, we have
  \[ \|u_k \|_s^2 = \sum_{\xi} (1 +\| \xi \|^2)^s | u_{k, \xi} |^2 \leqslant
     C, \; \exists C > 0, \]
  and since $\{ u_k \} \subset H_r$, similary, we have
  \[ \|u_k \|_r^2 = \sum_{\xi} (1 +\| \xi \|^2)^r | u_{k, \xi} |^2 \leqslant
     C', \; \exists C' > 0. \]
  
  
  Since $H_s \subset H_r$, for all $k$ we have
  \[ \sum_{\xi} (1 +\| \xi \|^2)^r | u_{k, \xi} |^2 \leqslant \sum_{\xi} (1
     +\| \xi \|^2)^s | u_{k, \xi} |^2 \leqslant C. \]
  Then for fixed $\xi$ the sequence $\{ (1 +\| \xi \|^2)^{r / 2} u_{k, \xi}
  \}_k$ is bounded and hence has a Cauchy subsequence. {\color[HTML]{B5005A}By
  the standard diagonalization}, then , we can {\color[HTML]{B4005A}find
  subsequence $\{ u_k \}$ such that $\{ (1 +\| \xi \|^2)^{r / 2} u_{k, \xi}
  \}_k$ is Cauchy sequence for every $\xi$.}
  
  Now the remaining task is to show that the $\{ u_k \}$ what we find is dose
  a Cauchy sequence, which will lead us to verify the $\{ u_k \}$satisfy the
  cauchy principle of convergence of norm version! Thus, we have
  \begin{eqnarray}
    \|u_k - u_l \| & = & \sum_{\| \xi \| \leqslant R} (1 +\| \xi \|^2)^r |
    u_{k, \xi} - u_{l, \xi} |^2 + \sum_{\| \xi \|> R} (1 +\| \xi \|^2)^r |
    u_{k, \xi} - u_{l, \xi} |^2 \nonumber\\
    & = & \sum_{\| \xi \| \leqslant R} (1 +\| \xi \|^2)^r | u_{k, \xi} -
    u_{l, \xi} |^2 + \sum_{\| \xi \|> R} \frac{(1 +\| \xi \|^2)^s }{(1 +\| \xi
    \|^2)^{s - r}} | u_{k, \xi} - u_{l, \xi} |^2 \nonumber\\
    & < & \varepsilon . \nonumber
  \end{eqnarray}
  
\end{proof}

\begin{remark}
  Since for fixed $\xi$, $\{ u_{k, \xi} \}_k$ is a Cauchy sequence, then we
  have $| u_{k, \xi} - u_{l, \xi} |  < \varepsilon$ for any $k, l \geqslant N,
  \exists N > 0$. Then we just need to consider the left unique variable
  $\xi$, which is divided into two cases that if  $\| \xi \| \leqslant R$ or
  not.
\end{remark}

\begin{definition}[Weak Solution]
  A function $\varphi \in L^2 (T) = H_0$ is siad to be a
  weak solution to $\Delta_d
  \varphi = \psi$ if
  \[  (\Delta_d \eta, \varphi) = (\eta, \psi), \]
  for all $\eta \in C^{\infty} (T)$.
\end{definition}

\begin{definition}[Self-Disjoint]
  If a function $\varphi \in L^2 (T) = H_0 \cap C^{\infty} (T)$, the
  Laplacian $\Delta_d$ is
  self-disjoint, i.e.
  \[  (\eta, \Delta_d \varphi) = (\eta, \psi) \Longleftrightarrow \Delta_d
     \varphi = \psi, \]
  for all $\eta \in C^{\infty} (T)$.
\end{definition}

{\color[HTML]{B4005A}Weak solutions are easy to find by Hilbert space
  methods, and {\bf the point is to prove regularity.}}


we first note that the weak solutions of the homogeneous equation $\Delta_d
\varphi = 0$ satisfy  $$(\| \xi \|^2 e^{i < \xi, x >}, \varphi_{\xi}) = 0 \text{ for
all $\xi$.}$$  Thus the weak harmonic space consists of the constant functions,
defined by $\varphi_{\xi} = 0$ for $\xi \neq 0$.($\Rightarrow \; \varphi_0 =
0$ is a necessary condition for (\ref{1}) to have a weak solution $\varphi$.)
\begin{equation}
  \Delta_d \varphi = \psi, \label{1}
\end{equation}
Next, we observe that {\color[HTML]{870043}{\shadowtext{(\ref{1}) makes sense when
$\psi \in L^2 (T) = H_0$ by} $$\Delta_d^2 \varphi = \Delta_d  (\Delta_d \varphi)
= \Delta_d \psi = 0 .$$} A necessary condition for it to have a weak solution
is that $\psi_0 = 0$, i.e. $\psi$ should be
orthogonal to the harmonic space}. ($\psi$ is
a weak solution to (\ref{1}))

Now assuming this to be the case,
\[ {\color[HTML]{000080}\varphi = - \sum_{\xi \neq 0} \frac{1}{\| \xi \|^2}
   \psi_{\xi} e^{i < \xi, x >}} \]
{\shadowtext{gives a formal Fourier series solution to (\ref{1})}}. Since
clearly, $\psi \in L^2 (T) \Longrightarrow \varphi \in L^2 (T)$, it is a weak
solution. 

\begin{definition}[Green's Operator]
  For $\psi \in L^2 (T)$, if we define the Green's operator by
  \begin{equation}
    G (\psi) = - \sum_{\xi \neq 0} \frac{1}{\| \xi \|^2 } \psi_{\xi} e^{i < x
    \comma x >}, \label{2}
  \end{equation}
  then
  \[ G : \; H_s \longrightarrow \; H_{s + 2 } \]
  is a bounded linear operator. 
  
  {\color[HTML]{008080}\shadowtext{(By
  Rellich Lemma, $H_{s + 2} \subset H_s$ is compact, which leads to the
  boundedness.)}} {\color[HTML]{008080}(As $G \in
  \mathfrak{F}$ (the space of formal Fourier series), then clearly it is a
  linear operator for $\mathfrak{F}$ is a linear space.)} In case
  {\color[HTML]{800080}$\psi$ is perpendicular to the harmonic
  space}{\color[HTML]{AA007F}, and $\varphi = G (\psi)$ provides a weak
  solution to (\ref{1}).} By Sobolev lemma, if $\psi \in C^{\infty} (T)$ then
  $\varphi \in C^{\infty} (T)$ and $\varphi$ is a solution to (\ref{1}) in the
  usual sense. Finally, by the Rellich lemma
  \[ G \; : \; L^2 (T) \longrightarrow L^2 (T), \]
  is a compact and self-disjoint operator. The spectral decomposition for $G$
  on $L^2 (T)$ is just Fourier series.
\end{definition}

The operator
\[ I + \Delta_d \; : \; H_s \longrightarrow H_{s - 2} \]
is trivially, since
{\color[HTML]{800080}$\Delta$ is second order}.

More importantly,

the identity
\[ {\color[HTML]{B4005A}\| (I + \Delta_d) \varphi \|_{s - 2}^2 =\| \varphi
   \|_s^2} \]
allows us to {\color[HTML]{800080}invert $I + \Delta_d$ on $L^2
(T)$} using {{\color[HTML]{800080}the closed graph theorem}}.

The identity may be calculated as follows:
\begin{align*}
  \| (I + \Delta_d) \varphi \|_{s - 2}^2 & =  \sum_{\xi \neq 0} (1 +\| \xi
  \|^2)^{s - 2} | ((I + \Delta_d ) \varphi)_{\xi} |^2\\
  & =  \sum_{\xi \neq 0} (1 +\| \xi \|^2)^{s - 2} | \varphi_{\xi} +
  (\Delta_d \varphi)_{\xi} |^2\\
  & =  \sum_{\xi \neq 0} (1 +\| \xi \|^2)^{s - 2} | \varphi_{\xi} +\| \xi
  \|^2 \varphi_{\xi} |^2\\
  & =  \sum_{\xi \neq 0} (1 +\| \xi \|^2)^{s - 2} \cdot (1 +\| \xi \|^2)^2
  \cdot | \varphi_{\xi} |^2\\
  & =  \sum_{\xi \neq 0} (1 +\| \xi \|^2)^s | \varphi_{\xi} |^2\\
  & =  \| \varphi \|^2_s
\end{align*}
Where $(\Delta_d \varphi)_{\xi}$ is the Fourier coefficient of $\Delta_d
\varphi = - \sum_{\xi} \varphi_{\xi} \| \xi \|^2 e^{i < \xi, x >}$.

\subsection{ Distribution}

\begin{definition}[The Condition of continuity for Distribution]
  The \tmcolor{red}{linear} functions
  \[ \lambda \; : \; C^{\infty} (T) \longrightarrow \mathbb{C}, \]
  which are {{\color[HTML]{B4005A}continuous}} in the sense that
  \[  {\color[HTML]{B4005A}| \lambda (\varphi) | \leqslant
    C_{\lambda} \sup_{[\alpha] \leqslant k, x \in T} | D^{\alpha} \varphi (x)
    | \quad \text{ for some $k$.}}\]
\end{definition}

Each distribution generates a formal Fourier series $\sum \lambda_{\xi} e^{i <
\xi, x >}$ where
\[ \lambda_{\xi} = \lambda (e^{i < \xi, x >}) . \]
{\color[HTML]{B4005A}It follows from the definition of
continuity of $\lambda$ and the above estimate on $\sup_{x \in T} | D^{\alpha}
\varphi (x) |$ that each distribution $\lambda$ is a continuous linear
function on $H_s$ for some $s$.} The pairing
\[  (u, v) = \sum_{\xi} u_{\xi} v_{\xi} \]
identifies $H_{- s}$ with the dual of $H_s$, so that $\lambda \in H_{- s}$
with its formal Fourier series given above. If we denote by $\mathcal{D} (T)$
the space of distributions, then we conclude that
\[ \mathcal{D} (T) = H_{- \infty} . \]

\begin{definition}[The derivatives of a distribution]
  The derivatives of a distribution are defined by
  \[ D^{\alpha} \lambda (\varphi) = \lambda (D^{\alpha} \varphi) . \]
  The Fourier coefficients of $D^{\alpha} \lambda$ are $(D^{\alpha}
  \lambda)_{\xi} = \xi^{\alpha} \lambda_{\xi}$.
\end{definition}

With this definition, a distribution is obtained by taking a finite
number of derivatives of a continuous function.

{\color[HTML]{B4005A}
  A distribution $\lambda$ is said to be in $L^2$ in case $\lambda \in H_0
  \subset H_{- \infty}$. Then we may describe the Soblev spaces by:
  $H_s$ consists of all distributions $\lambda$ such that the
  distributional derivatives $D^{\alpha} \lambda$ are in $L^2$ for $[\alpha]
  \leqslant s$.}

\subsection{ The proof of Hodge Theorem}

The essential idea is to produce the Green's operator by
Hilbert-space trick, and then to use the basic
estimate to show that it is a compact smoothing
operator.

\shadowtext{Firstly, we need to define the proper Green's operator $G$ by
Hilbert-space trick!}

\subsubsection{Extending Sobolev $s$-norm given by $L^2$-norm on torus $T$ to
vector bundles over manifolds}

On a torus $T$ we define its Sobolev $s$-norm by $L^2$-norm
\[  \sum_{[\alpha] \leqslant s} \int_T | D^{\alpha} \varphi |^2 \tmop{dx}, \]
which now can be extended to vector bundles over manifolds so that the Sobolev
lemma and Rellich lemma both remain valid.

\begin{definition}[\shadowtext{Sobolev Space}]\label{def:Sobolevspace}
  Fixed $1 \leqslant p \leqslant \infty$ and let $k$ be a nonzero integer.
  $W^{k, p} (U)$ consists of all locally integrable functions $u \; : \, U
  \longrightarrow \mathbb{R}$, such that for any multiindex $\alpha$ with $|
  \alpha | \leqslant k$, there exits (weak derivative) $D^{\alpha} u$ in the
  weak sense and $D^{\alpha} u \in L^p  (U)$.
\end{definition}

\begin{remark}
  If $p = 2$, we can write
  \[ H^k  (U) \assign W^{k, 2} (U), \quad k = 0, 1, \cdots . \]
  It is a {\shadowtext{{Hilbert space}}}, which is easy to verify that
  $H^0  (U) = L^2  (U)$. In this article, we denote $H^k (U)$ by $H_k$.
\end{remark}

{\color[HTML]{B4005A}Suppose
that $U \subset V \subset \mathbb{R}^n$ are open sets in $\mathbb{R}^n$ with
each relatively compact in the next. Functions with compact support in $U$ may
be considered as functions on a torus $T$!}\quad (*)

{\color[HTML]{B4005A}\shadowtext{The global Sobolev $s$-norm of sections $f \in
C^{\infty} (M, E)$}} is defined by
\[ \|f\|_s^2 = \sum_{k \leqslant s} \int_M \| \text{} \nabla^k f\|_0^2 d x,
\]
where
\[ \nabla^k f = \nabla (\nabla (\cdots (\nabla f) \cdots)) . \]
Denote $\mathcal{H}_s (M, E)$ the completion of $C^{\infty} (M, E)$ in above
norm. Since by (*), the global Sobolev norm induces a norm equivalent to the
usual Sobolev norm on sections compactly supported in a neighborhood of a
point, by using a partition of unity we may conclude that

\begin{lemma}[{\color[HTML]{B4005A}\shadowtext{Global Sobolev Lemma}}]\label{lem:globalsobolevlem}
  $\mathcal{H}_{s + [n / 2] + 1} (M, E) \subset C^s (M, E)$, the
  sections of differentiability class $s$ on $M$, and
  \[  \bigcap_s \texttt{\tmverbatim{}} \mathcal{H}_s (M, E) = C^{\infty} (M,
     E) . \]
\end{lemma}

\begin{lemma}[{\color[HTML]{B4005A}\shadowtext{Global Rellich Lemma}}]\label{lem:globalrellichlem}
  For $s > r$ the inclusion
  \[  \mathcal{H}_s (M, E) \subset \mathcal{H}_r (M, E) \]
  is a compact operator.
\end{lemma}

\subsubsection{ The basic estimate}

\begin{theorem}[{\color[HTML]{B4005A}\shadowtext{Garding Inequality}}]\label{thm:garding}
  For $\varphi \in A^{p, q} (M)$
  \[ \| \varphi \|_1^2 \leqslant C\mathcal{D} (\varphi) \quad  (C > 0) . \]
\end{theorem}
\begin{lemma}[{\color[HTML]{B4005A}\shadowtext{Regularity Lemma I}}]\label{lem:rugularitylem}
  Suppose that {\color[HTML]{800080}$\varphi \in \mathcal{H}_s^{p, q} (M)$},
  and that {\color[HTML]{800080}$\psi \in \mathcal{H}_0^{p, q} (M)$ is a weak
  solution} of the equation
  \[ {\color[HTML]{800080}\Delta \psi = \varphi} \]
  in the sense that
  \[  (\psi, \Delta \eta) = (\varphi, \eta) \]
  for all $\eta \in A^{p, q} (M)$. Then {\color[HTML]{800080}$\psi \in
  \mathcal{H}_{s + 2}^{p, q} (M)$}.
\end{lemma}

If $\varphi \in \mathcal{H}_0^{p, q} (M)$ is an eigenfunction for $\Delta$,
i.e. $\Delta \varphi = \lambda \varphi$ holds in the weak sense, where
$\lambda$ is the eigenvalue relatively to $\varphi$. Then
{\color[HTML]{B4005A}by Regularity Lemma, $\varphi \in
\mathcal{H}_s^{p, q} (M)$ for all $s$} ($\Delta$ is a minus second order
operator.). And by Global Sobolev
Lemma, one know that {\color[HTML]{B4005A}any
eigenfunctions for $\Delta$ is smooth.}

{\color[HTML]{B4005A}Note that any eigenvalue $\lambda
\geqslant 0$, and $\lambda = 0 \Longleftrightarrow \varphi$ is harmonic in the
weak sense. But by the Regularity and} Sobolev lemmas any such weak
harmonic form is smooth and harmonic in the usual sense.

\begin{lemma}[The basic estimate lemma]\label{lem:basic-estimate}
  Given $\varphi \in \mathcal{H}_0^{p, q} (M) $, there exists a unique $\psi
  \in \mathcal{H}_1^{p, q} (M)$ such that
  \[  (\varphi, \eta) =\mathcal{D} (\psi, \eta) = (\psi, (I + \Delta) \eta)
  \]
  for all $\eta \in A^{p, q} (M)$. The mapping
  \[  \psi = T (\varphi) \]
  from $\mathcal{H}_0^{p, q} (M)$ to $\mathcal{H}_1^{p, q} (M)$ is
  {\color[HTML]{B4005A}{\shadowtext{{bounded}}}}, and therefore the
  mapping
  \[ T \; : \; \mathcal{H}_0^{p, q} (M) \longrightarrow
     \mathcal{H}_0^{p, q} (M) \]
  is {\color[HTML]{B4005A}{\shadowtext{{compact and self-disjoint}}}}.
\end{lemma}



According to {\color{purple}{\bf\itshape the spectral theorem for compact, self-adjoint operators}} there
is a {\color{purple}{\bf\itshape Hilbert-space decomposition}}
\[ \mathcal{H}_0^{p, q} (M) = \bigoplus_m E (\rho_m), \]
where $\rho_m$ are the eigenvalues of $T$ and $E (\rho_m)$ are the
finite-dimensional eigenspaces. Since $T$ is one-to-one,
all $\rho_{m }\neq 0$; moreover, the equation
\[ T \varphi = \rho_m \varphi \]
is the same as
\[  (\varphi, \eta) = (\rho_m \varphi, (I + \Delta) \eta) \quad  (\eta \in
   A^{p, q} (M)), \]
which implies that
\[ \Delta \varphi = \left( \frac{1 - \rho_m}{\rho_m} \right) \varphi \]
in the weak sense. It follows that the eigenspaces for $T$ and $\Delta$ are
the same and are finite-dimensional vector spaces consisting of smooth forms.
The eigenvalues $\lambda_m$ for $\Delta$ and $\rho_m$ for $T$ are related by
\[ \rho_m=\frac{1}{1+\lambda_m}.\]

Assume that 
\[0=\lambda_0<\lambda_1<\cdots,\]
where $\lambda_m\uparrow \infty, \rho_m\downarrow 0$ as $m\to\infty$. The harmonic space $\mathcal{H}^{p,q}(M)$ corresponds to $\lambda_0=0$.
For $\varphi\in \mathcal{H}^{p,q}(M)^{\bot }$, we have
\[\|\Delta\varphi\|_0\geqslant\lambda_1 \|\varphi\|_0\quad (\lambda_1>0).\]
and if we define the Green's operator by 
\[
\begin{cases}
  G=0, & \text{on } \mathcal{H}^{p,q}(M),\\ 
    G\varphi=\frac{1}{\lambda_m}\varphi, &\varphi\in E\br{\frac{1}{1+\lambda_m}},\end{cases}\]
then {\color{purple}\shadowtext{\bf\itshape$G$ is a compact, self-adjoint operator with spectral decomposition}}
\[ \mathcal{H}^{p,q}(M)=\mathcal{H}^{p,q}(M)\oplus (\oplus_m E(\rho_m)),\]
where 
\[G\varphi=\left(\frac{\rho_m}{1-\rho_m}\right)\varphi.\]
\prob[Summarization]{deepGreen}
By \ref{lem:globalsobolevlem} and \ref{lem:globalrellichlem} we show that 



% Bibliography
\clearpage
\pagestyle{\auxsettings}
\printbibliography[heading=bibintoc]
\end{document} 