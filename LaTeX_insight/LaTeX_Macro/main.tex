\documentclass[fontset=windows, 12pt]{article}
\usepackage{Article}

\title{\LaTeX 的内部宏}
\author{Eureka}
\date{\today}



%% 定义宏的区域
%% 1. 特殊的宏名
\newcommand{\+}[1]{\ensuremath{\mathscr{#1}}}
\newcommand{\2}[1]{\ensuremath{\mathbb{#1}}}


\begin{document}
\maketitle


\begin{tformal}{分数的技巧}
    \begin{verbatim}
\frac{a}{b} = \frac ab
    \end{verbatim}

\noindent \textbf{实际演示:}\par 

$\frac{1}{2}$ 和 $\frac12$
\end{tformal}

%% 特殊的宏名
\begin{tformal}{特殊的宏名}
\begin{verbatim}
%% 命令的定义
\newcommand{\+}[1]{\ensuremath{\mathscr{#1}}}
\newcommand{\2}[1]{\ensuremath{\mathbb{#1}}}
%% 命令的使用:
% 样例一: \+A
% 样例二: \+AB
% 样例三: \2A
% 样例四: \2AB
\end{verbatim}


$\bullet$  第一个演示效果: \+A. \\
$\bullet$  第二个演示效果: \+AB. \\
$\bullet$  第三个演示效果: \2A. \\
$\bullet$  第四个演示效果: \2AB. \\
\end{tformal}

\newpage
\centerline{\textbf{Macros2$\varepsilon$中的内容}}

\section{Constants}
\subsection{Number Constants}
主要的相关内容, 他们主要和以下的几个命令相关
\begin{itemize}
    \item 1. \verb |\countdef|
    \item 2. \verb |\chardef|
    \item 3. \verb |\mathchardef|
\end{itemize}
几个例子
\begin{tformal}{数字常数}
    \makeatletter
    \@ne $\ne$ @
    \makeatother
\end{tformal}

\section{Variables}

\section{Macros}

\section{\TeX 是怎么产生的}

\textbf{下面为一些尝试的命令}

命令一:\verb |T\kern-0.2em\lower.25em\hbox{E}\kern-0.1em X|

命令二:\verb |\ensuremath{\mathrm{L\kern-.325em{\scriptstyle{A}}\kern-.17em}}\TeX|

% 命令三:L\kern-.325em\hbox{\check@mathfonts\fontsize\sf@size\z@\math@fontsfalse\selectfont A}\kern-.17em\TeX

运行效果:T\kern-0.2em\lower.25em\hbox{E}\kern-0.1em X~~~~\ensuremath{\mathrm{L\kern-.325em{\scriptstyle{A}}\kern-.17em}}\TeX

正如上面的代码所示,主要就是一个lower宏,使用+,-可以实现上下偏移。例如,我们把lower后面的
正号变为 -时,E机会出现在上面, \LaTeX 中的A同理

效果演示一:T\kern-0.1em\lower-.25em\hbox{E}\kern-0.1em X 

效果演示二:L\kern-0.2em\lower0.3em\hbox{A}\kern-0.2emT\kern-0.1em\lower-.25em\hbox{E}\kern-0.1em X

\clearpage
下面展示原理:
\begin{itemize}
    \item 1. \textbackslash kern:  其中 -  表示向左移动
    \item 2. \textbackslash lower: 其中 - 表示向上移动
    \item 3. \textbackslash raise: 其中 - 表示向下移动
    \item 4. \textbackslash hbox: 表示水平盒子
\end{itemize}

\begin{tformal}{\LaTeX 的原始命令}
\begin{verbatim}
\DeclareRobustCommand{\LaTeX}{L\kern-.36em%
    {\sbox\z@ T%
        \vbox to\ht\z@{\hbox{\check@mathfonts
                            \fontsize\sf@size\z@
                            \math@fontsfalse\selectfont
                            A}%
                    \vss}%
    }%
    \kern-.15em%
    \TeX}
\end{verbatim}
\end{tformal}

\end{document}