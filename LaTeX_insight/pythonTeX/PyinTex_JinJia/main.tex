\documentclass[fontset=windows]{article}
\PassOptionsToPackage{quiet}{fontspec}
\usepackage[a4paper, total={6.5in, 10in}]{geometry}
\usepackage{lmodern}
\usepackage{mathtools}
\usepackage{amsmath}
\usepackage{ctex}
\usepackage{xcolor}
\title{Python In \LaTeX}
\author{}
\date{}

% 注: 在命令行输入多行内容时,使用Shift+Enter换行(不执行)

\begin{document}
\maketitle
    Hello{} \LaTeX

    \section{PythonInTex}
    \subsection{普通的替换}
    ``{{ 'hello world'|title }}''

    ``{{ 'hello world'.upper() }}''

    \subsection{设置计数器}
    %   我们把Settings类传到st变量中了:tmptex = Template(tex).render(st=Settings)
    第一次使用计数器:设置从5开始{{ st.getinc('a', 5) }}

    第二次使用计数器:{{ st.getinc('a') }}

    \subsection{日期设置}
    CTeX默认使用$\backslash$ today:\today

    调用python生成日期:{{ st.today() }}

    指定日期格式: {{ st.today('%Y-%m-%d:%H-%M-%S') }}
    

    \subsection{大写数字的转换}
    原数:0123456789

    汉字:{{ st.num2cn('0123456789', isbig=False) }}

    汉字:{{ st.num2cn('0123456789', isbig=True) }}

    嵌套书写: {{ st.num2cn(st.today('%Y年%m月%d日 %H时%M分%S秒')) }}

    \subsection{直接使用python语法}
    {{ st.loop(8, 'Hello') }}

    \subsection{测试素数}
    测试定义的素数函数:{{ st.IsPrime(13) }}

    测试定义的素数函数:{{ st.IsPrime(16) }}

    \subsection{使用循环}
    %% 注意:%和{之间不能有空格
    循环和判断:
        
            
            {{num}} > 3
            
            {{num}} < 3
            
        
    


    \section{反思}
    实际上在引用Python后,很容易完成某些工作。因为python擅长于处理字符串,特别是在Jinja2模板库的
    帮助下。这样一来,就有一个问题,在TeX中许多功能的包只完成一些简单的替换,是不是可以用Python
    的这种形式来完成文字处理这一部分(即\textcolor{red}{\textbf{宏替换}}),而仅仅是把TeX作为一个排版的引擎。

\end{document}