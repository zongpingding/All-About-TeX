\documentclass[fontset=windows, 12pt]{article}
\PassOptionsToPackage{quiet}{fontspec}
\usepackage[a4paper, total={6.5in, 10in}]{geometry}
\usepackage{amsmath}
\usepackage{tikz}
\usepackage{ctex}
\usepackage{pgfplots}
\pgfplotsset{compat=1.17}
\usepackage{xcolor}
\usepackage{float}
\usepackage{fontspec}

\newcommand{\Testsentence}[1]{Hello This is a sentence Cretaed By Font Who's is #1}

% 1. Apalu Font
\newfontfamily{\Apalu}[Path=./Apalu/]{Apalu-2.otf}
\newfontfamily{\ApalU}[Path=./Apalu/]{Apalu-3.ttf}

% 2. LeagueGothic Font
\newfontfamily{\LeagueGothic}[Path=./LeagueGothic/]{LeagueGothic-Regular.ttf}

% 3. Philosopher
\newfontfamily{\Philosopher}[Path=./Philosopher/]{Philosopher-Regular.ttf}

% 4. LibreSemiSansSSK
\newfontfamily{\LibreSemiSansSSK}[Path=./LibreSemiSansSSK/]{LibreSemiSansSSK-Regular.ttf}

% 5. Note-Script-Medium
\newfontfamily{\NoteScriptMedium}[Path=./Note-Script-Medium/]{Note-Script-Medium-2.ttf}


%  中文字体
% 1. YuanYunMingTi
\setCJKfamilyfont{YuanYunMingTi}[Path=./YuanYunMingTi/]{GenWanMinJP-Regular-4.ttf}


\title{OnlineFont}
\author{Eureka}
\date{\today}
\begin{document}
\maketitle

\section{英文的字体}
\subsection{Apalu字体}

    注意这款字体不能进行商用,自己私下用用就行了。
    这两个Apalu字体差不多,主要就是otf和ttf的区别

    {\Apalu \Testsentence{Alpau-2}}

    {\ApalU \Testsentence{Apalu-3}}

\subsection{LeagueGothic}
    这个无衬线字体还是很经典的,可以免费商用。
    拥有粗体,斜体,粗斜体

    {\LeagueGothic \Testsentence{LeagueGothic}} 

\subsection{Philosopher 字体}
    Philosopher 也是一款免费开源的字体,这款字体是依据SIL Open Font License 1.1授权协议免费公开.
    拥有粗体,斜体,粗斜体

    {\Philosopher \Testsentence{Philosopher}}


\subsection{LibreSemiSansSSK 字体}
    这款字体不能进行商用,具有regular和Bold两种字体

    {\LibreSemiSansSSK \Testsentence{LibreSemiSansSSK}}

\subsection{Note-Script-Medium 字体}
    一款看着很舒服的英文手写体,个人感觉比calligraphi字体好看,,但是还是不能够商用。
    提供了一种字体。

    {\NoteScriptMedium \Testsentence{Note-Script-Medium}}


\section{免费开源的中文字体}
\subsection{YuanYunMingTi 字体}
    YuanYunMingTi这款字体看着版正,没别的

    % {\YuanYunMingTi \Testsentence{YuanYunMingTi}}

    {\CJKfamily{YuanYunMingTi} 这是一段测试中文字体效果的句子,这个字体是YuanYunMingTi}

\end{document}