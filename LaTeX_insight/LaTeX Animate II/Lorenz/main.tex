\documentclass{article}
\usepackage{multido}
\usepackage{pst-3dplot}
\usepackage{pst-ode}
\usepackage{animate}

\begin{document}

%Lorenz' set of differential equations
\def\lorenz{%
  10*(x[1]-x[0])        | %dx/dt
  x[0]*(28-x[2]) - x[1] | %dy/dt
  x[0]*x[1] - 8/3*x[2]    %dz/dt
}%
%
%write timeline file
\newwrite\OutFile%
\immediate\openout\OutFile=lorenz.tln%
\multido{\iLorenz=0+1}{101}{%
  \immediate\write\OutFile{::\iLorenz x0}%
}%
\immediate\write\OutFile{::c,101}%
\multido{\iLorenz=102+1}{89}{%
  \immediate\write\OutFile{::\iLorenz}%
}%
\immediate\closeout\OutFile%
%
\psset{unit=0.155,linewidth=0.5pt}%
\noindent\begin{animateinline}[
  timeline=lorenz.tln,
  controls,poster=last,
  begin={\begin{pspicture}(-36,-13)(36,55)},
  end={\end{pspicture}}
]{10}
  %coordinate axes
  \psset{Alpha=120,Beta=20}%
  \pstThreeDCoor[xMax=33,yMax=33,zMax=55,linecolor=black]%
\newframe
  %attractor segments
  \gdef\initCond{10 10 30}% initial condition
  \pstVerb{/lorenzXYZall {} def} %takes the whole attractor
  \multiframe{100}{rtZero=0+0.25,rtOne=0.25+0.25}{%
    %compute current attractor segment, store it in `lorenzXYZseg'
    \pstODEsolve[algebraic]{%
      lorenzXYZseg}{0 1 2}{\rtZero}{\rtOne}{26}{\initCond}{\lorenz}%
    %empty initial condition --> next \pstODEsolve continues
    \gdef\initCond{}%            from last state vector
    %append segment to the whole attractor stored in `lorenzXYZall'
    \pstVerb{%
      /lorenzXYZall [lorenzXYZall lorenzXYZseg] aload astore cvx def}%
    %plot the current segment
    \listplotThreeD[plotstyle=line]{lorenzXYZseg}%
  }%
\newframe% required between two \multiframe
  %fly-around (whole attractor)
  \multiframe{90}{rAlpha=116+-4}{%
    \psset{Alpha=\rAlpha,Beta=20}%
    \pstThreeDCoor[xMax=33,yMax=33,zMax=55,linecolor=black]%
    \listplotThreeD[plotstyle=line]{lorenzXYZall}%
  }%
\end{animateinline}

\end{document}