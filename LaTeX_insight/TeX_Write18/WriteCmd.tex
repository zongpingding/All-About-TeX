\documentclass{article}
\usepackage[a4paper, total={6.5in, 10in}]{geometry}
\usepackage{graphicx}
\usepackage{listings}
\usepackage{xcolor}
\usepackage{multicol}
\usepackage{gplot}
% \usepackage{FunctionPlot}



% Env setting
\lstset{
    frame=tb,
    aboveskip=3mm,
    belowskip=3mm,
    showstringspaces=false,
    columns=flexible,
    framerule=1pt,
    rulecolor=\color{gray!35},
    backgroundcolor=\color{gray!5},
    basicstyle={\small\ttfamily},
    numbers=left,
    numbersep=5pt,
    numberstyle=\small\color{gray}\ttfamily,
    keywordstyle=\color{blue},
    commentstyle=\color{dkgreen},
    stringstyle=\color{mauve},
    breaklines=true,
    postbreak=\raisebox{0ex}[0ex][0ex]{\ensuremath{\hookrightarrow\space}},
    breakatwhitespace=true,
    tabsize=4,
}
\makeatletter
\lstnewenvironment{bytes}[1][\f@size]           % setting default code fontsize = \normalsize 
{
    \lstset{%
        basicstyle=\fontsize{#1pt}{#1pt}\selectfont\ttfamily, 
    }
}{}
\makeatother




\title{TeX Write18 Config}
\author{Eureka}
\date{\today}

\begin{document}
\maketitle

\section{Start}
Test Echo someting to a \verb|.txt| file
\begin{bytes}
	\immediate\write18{echo Hello2 > ./Test_export.txt}
\end{bytes}
\immediate\write18{echo Hello2 > ./Test_export.txt}

which means that, \TeX{} could interact with the system, and could use the system to do some things.



\section{Powershell Script}
1. Debug Process(Use .ps1 in "Win+R" windows)
\begin{bytes}
	\immediate\write18{pwsh.exe -NoProfile -File "./Scripts/Gplot_3d.ps1" x**2}
\end{bytes}

2. Test Use self Define commanf in \verb|$profile| in PowerShell
\begin{bytes}
	\immediate\write18{pwsh.exe -NoProfile -File "./Scripts/Gplot_3d.ps1" x**2}

	\begin{center}
		\includegraphics[width=.4\textwidth]{./gnuoutput/Function_output_3d.pdf}
	\end{center}
\end{bytes}
% \immediate\write18{pwsh.exe -NoProfile -File "./Scripts/Gplot_3d.ps1" x**2}
% \begin{center}
% 	\includegraphics[width=.3\textwidth]{./gnuoutput/Function_output_3d.pdf}
% \end{center}

3. Make a Command
\begin{bytes}
	% To use it convenient,make it be a command:
	\newcommand{\Gplotz}[1]{%
		\immediate\write18{pwsh.exe -NoProfile -File "./Scripts/Gplot_3d.ps1" #1}
		\includegraphics[width=.4\linewidth]{./gnuoutput/Function_output_3d.pdf}
	}
	% Example
	\Gplotz{x**2 + y**2}
\end{bytes}

% \begin{center}
% 	\Gplotz{x**2+y**2+x+y+x*y}
% \end{center}
It's in Unix, So I can't active with Powershell.
\textcolor{red}{use pwsh is very inconvenient}


\clearpage
\section{LaTeX3 Base:Win and Unix}
Therefore we use unix like command:
\begin{multicols}{2}
	\begin{itemize}
		\item sed, grep, awk 
		\item mv, cp, mkdir, rm
	\end{itemize}	
\end{multicols}


\begin{bytes}
%% 1.counter to identify picture created by Gnuplot
\newcounter{gnu@plot@pic@counter}
\newcommand{\gnu@picture@fullname}{}
\newcommand{\Gplotz}[1]{%
    \sys_shell_now:n {pwsh.exe~ -NoProfile~ -File~ "./Scripts/Gplot_3d.ps1" #1}
	\includegraphics[width=0.4\textwidth]{./gnuoutput/Function_output_3d.pdf}
}
\newcommand{\GplotzNew}[1]{%
    \stepcounter{gnu@plot@pic@counter}
	\sys_shell_now:n {sed~ -i~ "34s|f(x,~ y)~ =~ .*|f(x,~ y)~ =~ #1|"~ ./Scripts/Gplot_3d.gp}
	\sys_shell_now:n {gnuplot~ ./Scripts/Gplot_3d.gp}
    % picture rename
    \renewcommand{\gnu@picture@fullname}{%
		Function_output_3d_\the\value{gnu@plot@pic@counter}.pdf
	}
    \cs_generate_variant:Nn \sys_shell_mv:nn {nx}
    \sys_shell_mv:nx {./Function_output_3d.pdf}{./gnuoutput/\gnu@picture@fullname}
	\includegraphics[width=0.45\textwidth]{./gnuoutput/\gnu@picture@fullname}
}

%% 2.Example 
\begin{center}
	\GplotzNew{x+100y}
	\GplotzNew{x**2-y**2-1000}
\end{center}
\end{bytes}



\begin{center}
	\GplotzNew{x+y}
	\GplotzNew{x**2-y**2-10}
\end{center}


\clearpage
\section{ParamPlot}
ParamPlot is Defined as post:
Becareful the \verb|\[.*\]| in \verb|sed| command , it will cause \verb|\GenericError| Error.

So don't use the following cmd:
\begin{bytes}
	\immediate\write18{sed -i "27s|set yr \[.*\]|set yr [#1]|" ./Scripts/Gplot_2d.gp}
\end{bytes}

use \verb|[#2]| to add the \verb|[]|, instead of in the \verb|sed| cmd...


\begin{bytes}
% 3.param_plot
\newcommand{\GPplotz}[2][-4:4]{%
    \stepcounter{gnu@plot@pic@counter}
    \sys_shell_now:n {sed~ -i~ "42s|splot~ .*|splot~ #2|"~ ./Scripts/GPplot.gp}
	\sys_shell_now:n {sed~ -i~ "36s|set~ zr~ .*|set~ zr~ [#1]|"~ ./Scripts/GPplot.gp}
	\sys_shell_now:n {gnuplot~ ./Scripts/GPplot.gp}
    % picture rename
    \renewcommand{\gnu@picture@fullname}{%
		Param_Function_output_\the\value{gnu@plot@pic@counter}.pdf
	}
    \cs_generate_variant:Nn \sys_shell_mv:nn {nx}
    \sys_shell_mv:nx {./Param_Function_output.pdf}{./gnuoutput/\gnu@picture@fullname}
	\includegraphics[width=0.45\textwidth]{./gnuoutput/\gnu@picture@fullname}
}

%% Example
\begin{center}
	\GPplotz[0:2]{1*cos(u)*cos(v), 2*cos(u)*sin(v), sin(u)}
	\GPplotz[-4.5:4.5]{4*cos(u)*cos(v), 4*cos(u)*sin(v), 4*sin(u)}
\end{center}
\end{bytes}


\begin{center}
	\GPplotz[-.75:.75]{1*cos(u)*cos(v), 2*cos(u)*sin(v), sin(u)}
	\GPplotz[-4.5:4.5]{4*cos(u)*cos(v), 4*cos(u)*sin(v), 4*sin(u)}
\end{center}


\end{document}






