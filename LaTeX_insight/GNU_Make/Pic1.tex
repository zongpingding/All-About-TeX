% 生成最终的图象时把第一个文档类取消注释即可
\documentclass[10pt,varwidth]{standalone}
% \documentclass[12pt]{article}
% 1.必须添加varwidth选项,不然就会报错
\PassOptionsToPackage{quiet}{fontspec}
\usepackage{ctex}
\usepackage{geometry}
% 必须要保证绘图的纸张足够的大
\geometry{a2paper,left=1in,right=1in,top=1in,bottom=1in}
\usepackage{xifthen}
\usepackage{xfp}
\usepackage{xcolor}
\usepackage{amsmath}
\usepackage{pgfplots}
\usepackage{pgfplotstable}
\pgfplotsset{compat=1.16}
% 2.引用的tikz库
\usetikzlibrary {matrix, chains, trees, decorations}
\usetikzlibrary {arrows.meta, automata,positioning}
\usetikzlibrary {decorations.pathmorphing, calc}
\usetikzlibrary {backgrounds, mindmap,shadows}
\usetikzlibrary {patterns, quotes, 3d, shadows}
\usetikzlibrary {graphs, fadings, scopes}
\usetikzlibrary {arrows, shapes.geometric}
\usepgflibrary {shadings}

\tikzset{
    >={Latex[length=6mm, width=2mm]}
}




\begin{document}
\def\R{5em}
\def\addLength{12em}
\begin{tikzpicture}
    % 1.标记 node
    \coordinate (O) at (5em, 0); % 原点坐标
    \node at (O) {\huge 节点$j$};
    \coordinate (node1) at ($(O) + ({-\R*sin(30)}, {\R*cos(30)})$);
    \coordinate (node2) at ($(O) + ({-\R*sin(60)}, {\R*cos(60)})$);
    \coordinate (node3) at ($(O) + ({-\R*sin(90)}, {\R*cos(90)})$);
    \coordinate (node5) at ($(O) + ({-\R*sin(30)}, {-\R*cos(30)})$);
    \coordinate (node4) at ($(O) + ({-\R*sin(60)}, {-\R*cos(60)})$); 
    % \foreach \num in {30, 60, 90, -30, -60}{
        
    % }
    % 绘制目标节点
    \foreach \num in {1,...,5}{
        \node[circle, fill=blue!60, inner sep=2pt] at (node\num) {};
    }


    % 2. 基本框架
    \draw[thick, blue!80] (O) circle(\R);
    \draw[very thick, ->] (10em, 0) -- (25em, 0)node[right]{$e_j$};
    \foreach \num in {1,...,5}{
        \draw[->, thick]%
            ($(O) + ({-\addLength*sin(\num*30)}, {\addLength*cos(\num*30)})$)%
            -- node[pos=.5, above] {$w_{k_\num{}j}$} 
            (node\num);
    }
    \node[circle, draw=black] at %
        ($(O) + ({-14em*sin(30)}, {14em*cos(30)})$) {节点$k_1$};

    

    % 3. 图上标注
    \node at (24em, 6em) {
        \parbox{34em}{
            \begin{align}
                & w'_{k_1, j} = \frac{w_{k_1,j}}{w_{k_1, j}+w_{k_2, j}+w_{k_3, j}+w_{k_4, j}+w_{k_5, j}} \\[2em]
                & e_{k_1,j} = \sum_{j=1}^{n}{e_{k_1}\cdot w'_{k_1,j}},\; k_n \sim n(\mbox{一一对应},\mbox{反向传播到节点} k_1 \mbox{的误差})
            \end{align}
        }
    };



    
\end{tikzpicture}
\end{document}
